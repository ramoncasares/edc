% ufo1.tex (RMCG20040518)


\Section Mi compromiso

El propósito de este ^libro^ es hacer filósofos. El objetivo es que
usted, después de leerlo, sea ^filósofo^. Y lo será, se lo prometo. Aquí
va mi ^compromiso^: si usted lee completamente este libro, obedeciendo
las recomendaciones que se encuentre, entonces cuando lo finalice será
filósofo.

Otro modo de hacerse filósofo es ir a la ^universidad^. Pero esa es una
manera larga y costosa, y los tiempos ya no están para hacer tantos y
tan prolongados esfuerzos. Aquí tomaremos atajos, tantos como sea
menester para alcanzar nuestra meta.

Si a esta ganga se le aplican los precavidos consejos de la abuela, ``lo
barato sale caro'', o los ^refranes^ populares del ^abuelo^, ``no hay
atajo sin trabajo'', es algo que usted mismo tendrá que juzgar. Yo, por
supuesto, soy parcial, y mi recomendación es tajante: lea este libro
para hacerse filósofo. Tenga en cuenta que si la alternativa
universitaria le falla, habrá perdido años irrecuperables de su vida. En
cambio, si lo que le falla es este libro, siempre podrá revenderlo o, en
último término, regalarlo.

Supongo que ahora se estará preguntando usted por qué no le pongo el
tranquilizador mensaje del comerciante seguro: ``si no queda satisfecho,
le devolvemos su dinero''. Bueno, ya está puesto, aunque, claro, como
aparece en una pregunta, no tiene valor legal. Pues si no lo tenía,
ahora lo tendrá: si cumple usted las condiciones del compromiso del
primer párrafo, y no es usted filósofo, le devuelvo su dinero. {\it Mi
^abogada^ me obliga a aclarar que mis responsabilidades pecuniarias sólo
alcanzan a la parte correspondiente a mis ^derechos de autor^, que, y
esto ya lo digo sin el permiso de mi abogada, son exiguos}.


\Section La conspiración de los dioses

Aunque ya esté usted convencido de que, para ser ^filósofo^, es más
fácil y rápido leer este ^libro^ que ir a la ^universidad^, puede seguir
pensando que, aun así, ser filósofo no merece el esfuerzo de leer un
libro, aunque sea corto. Al fin y al cabo, como decíamos antes, los
tiempos no son propicios a los esfuerzos, por menguados que sean.

Los argumentos, sin embargo, son contundentes. Los filósofos somos lo
más excelso y elevado de la raza humana. Nuestro saber es, ¡a la vez!,
el más alto y profundo, no sólo de todos cuantos saberes hay, sino
incluso de los que haber pudiera. La sutileza e importancia de nuestros
razonamientos no tiene igual en las ciencias ni en las artes, y por
encima de todas ellas, como pináculo y epítome de todo ^conocimiento^,
se encuentra la filosofía.

La ^filosofía^ apenas se interesa por el conocimiento mundano, y
desprecia la ^opinión^. Lo que busca es la verdad más profunda, el
misterio que se esconde detrás de la engañosa ^apariencia^. La filosofía
es una tarea de titanes empeñados en desvelar la ^conspiración^ de los
dioses, que se complacen en engañar a los humanos comunes.


\Section El primer principio primordial

Si comparáramos el ^conocimiento^ con un ^edificio^, la ^filosofía^ se
correspondería con la ^estructura^. A los filósofos nos basta saber que
los tabiques podrán sustentarse en el armazón que hemos dispuesto. Nos
resulta irrelevante cómo serán las habitaciones que puedan construirse
con los tabiques, y una pérdida de tiempo considerar el color en el que
se pintarán las paredes. Estas tareas se corresponderían con los
intereses de las ^ciencias^ y las ^artes^. Continuando con esta
comparación, los aspectos más mudables del edificio, como el mobiliario
y la decoración, representarían el conocimiento más cambiante, que es la
^opinión^.

Incluso dentro de la filosofía hay ramas primeras y segundas. A la
primera se le denomina `metafísica'. La ^metafísica^ termina en cuanto
la cimentación del edificio del ^saber^ está dispuesta. Y aun dentro de
la metafísica lo primero, el diseño del edificio, es la epistemología.
La ^epistemología^ es el primero de los principios de la filosofía, que
es el ^conocimiento^ primordial, porque su problema es determinar la
^posibilidad^ del propio conocimiento.

Esto explica que la relación entre nosotros los epistemólogos y el resto
de los filósofos sea similar a la que existe entre los filósofos y el
vulgo. El filósofo no se explica que el profano pierda sus energías en
asuntos mudables, que al poco tiempo pasan de moda y pierden
completamente su valor. El lego reprocha al filósofo dedicarse a asuntos
abstractos, poco útiles para satisfacer las necesidades cotidianas.

En realidad, los filósofos que no son epistemólogos ven absolutamente
^innecesaria la tarea de la epistemología^. Y, aunque están equivocados,
resumiré aquí sus motivos. Piensan que dado que es evidente la
existencia de conocimiento, determinar la posibilidad del conocimiento
es un asunto superfluo y prescindible. Los epistemólogos, en cambio, nos
percatamos del desatino que supone elaborar conocimientos que, a la
postre, podrían mostrarse falaces y faltos de todo fundamento.


\Section Las grietas de la razón

La falta de entendimiento entre los epistemólogos y los que no lo son se
debe a que unos vemos un ^problema^ en donde los otros no ven
dificultad. Para los que no ven un problema, el intento de resolución
del supuesto problema es una actividad sin sentido ni significado, que
nunca podrán entender. Cuando hay diferencias de criterio sobre la
manera de resolver una dificultad, se produce, desde luego,
desentendimiento. Pero cuando no se acierta a adivinar siquiera qué
problema intenta resolver el otro, la incomprensión es total y la
comunicación imposible.

Estas observaciones me señalan el ^atajo^ que voy a tomar para hacerle a
usted ^filósofo^. Porque una de las tareas del filósofo ha de consistir
en buscar grietas en la ^estructura^ del edificio del ^saber^. Es decir,
una de las ocupaciones de la filosofía consiste en el mero enunciado de
problemas fundamentales o ^aporías^. Ojo, no vale plantear problemas de
decoración; han de ser problemas de estructura del ^conocimiento^, o
sea, verdaderas ^paradojas^, o ^grietas^ de la razón.

Por supuesto, otras ^tareas filosóficas^ más laboriosas pretenden la
construcción completa de nuevos edificios del saber. Es a lo que se
dedican los filósofos con más ^coraje^, a saber, a la elaboración de
sistemas filosóficos propios. Pero también estas tareas supremas de la
filosofía deben tomar en consideración los problemas encontrados en los
edificios anteriores, para así no repetir los ^errores^ de aquéllos. De
modo y manera que, incluso si usted me resulta finalmente un filósofo
corajudo, podrá sacar provecho de la instrucción que aquí se imparte,
que le convertirá en un experto en la detección de grietas.


\Section Mi plan

El ^plan^ consiste en revisar el edificio del ^conocimiento^ actual,
buscando en él ^grietas^ y ^problemas^ estructurales. Haremos una
inspección parcial los dos juntos, usted y yo, para que así aprenda a
hacerla por usted mismo.

Las objeciones a mi plan pueden ser varias. Se me ocurren tres, que
rechazaré una a una, explicando mis razones.

Una es que mi juego es ^destructivo^, y no constructivo, porque si
triunfa el plan y encontramos problemas estructurales, entonces lo único
prudente es declarar ruinoso el edificio y derribarlo. Esto puede ser
cierto, pero, en todo caso, es sólo la mitad de la verdad. Porque si el
edificio fuera efectivamente ruinoso, entonces mantener nuestra vivienda
en él sería insensato. Pero si nuestra inspección no encontrara grietas,
y esta es la otra posibilidad, entonces nos permitiría sentirnos más
seguros en él. Luego, en cualquiera de los dos casos, nuestro plan tiene
mérito. Además, para saber construir vigas resistentes, hay que saber
buscarles su ^punto de rotura^.

Otro reparo es que el examen de un único edificio, el actual, producirá,
sin duda, una falta de perspectiva, porque no se puede generalizar a
partir de un único caso. Pero en este ^libro^ no tenemos espacio
suficiente para estudiar las variadas estructuras de conocimiento que se
han ido elaborando a lo largo del tiempo y en los diferentes lugares del
mundo. Además, encontrar grietas en un edificio que ya ha sido demolido
supone un enorme esfuerzo, ya que primero hay que reconstruirlo. Y
siempre quedará la duda de si la ^reconstrucción^ es tan precisa que
incluso ha reproducido fielmente las grietas.

La tercera, y última, objeción es que la exploración debería ser
exhaustiva, y no parcial. A esto digo que escribir la enciclopedia de
las paradojas sería, desde luego, una tarea filosófica de primer rango,
pero que no es el objetivo de este libro. Aquí sólo pretendo que mi
lector, usted, sea capaz de apreciar las paradojas. Porque apreciar una
^paradoja^ es percatarse de un ^problema^ filosófico, es descubrir que
lo obvio no es tal como se ve.

Quien da por bueno lo ^obvio^, quien acepta todo cuanto ve, no es un
^filósofo^. Filósofo es quien gusta de las paradojas. Filósofo es quien
se cuestiona lo obvio. Mi plan consiste en enseñarle a cuestionar lo
obvio.


\Section Descartes

Antes de introducirnos en el ^plan^ para realizarlo, vamos a deternernos
a examinar el propio plan. Porque desde dentro no se ve el ^bosque^, se
ven ^árboles^.

Mi filósofo favorito es \[Descartes]. Seguramente por esto mi plan es
una simplificación del ^método^ cartesiano. El método cartesiano
consiste en dudar de todo cuanto se sabe. La ^duda^ metódica desacreditó
el recurso a la ^autoridad^ como base sobre la que elaborar la
^filosofía^, y como consecuencia fundamentó las ^ciencias^, cuyo juez
definitivo es el ^experimento^ empírico, que se basa en la realización
de ^mediciones^ que son replicables por cualquiera que tenga un
^conocimiento^ suficiente del arte.

Cuando se trata de \[Descartes] tengo que contenerme, porque fácilmente
me extiendo más de lo necesario. El caso es que \[Descartes] no está de
moda porque decretó una diferencia insalvable entre la ^materia^ y el
^espíritu^. Para él todo funciona como un ^mecanismo^, excepto el
^pensamiento^ y el ^habla^. Según la doctrina descartiana, las cosas
materiales que existen realmente son máquinas cuyo funcionamiento está
determinado completamente por las ^leyes^ naturales, y no son, en
consecuencia, ^libres^. Por este motivo, nuestro hablar y pensar
libremente no pueden ser asimilados, de ningún modo, a una máquina. Y,
en razón a esta diferencia irreconciliable, concluyó \[Descartes] que
habían de existir dos ^substancias^: materia y espíritu.

Las ciencias se centraron en el estudio de la materia y descubrieron
muchas leyes naturales. Ha sido tanto, tan completo y absoluto el éxito
alcanzado por las ciencias, gracias al método cartesiano, que la teoría
filosófica hoy predominante, denominada ^materialismo^ por razones
obvias, sostiene que todo es material; y todo es todo, así que también
nos incluye a usted y mi. Para el materialismo la libertad no existe,
y si nos parece que somos libres es porque sufrimos una ilusión.
Podría ser, ya que, como nos enseñó \[Descartes],
las ^apariencias^ engañan.

Los materialistas son ingratos, y a pesar de que se lo deben todo al
método de \[Descartes], proclaman sin verguenza que ^\[Descartes] estaba
equivocado^. Dicen, a quien quiera oirlo y a quien no, que no hay
espíritu, que todo es materia. Que no hay libertad, y que las leyes de
la naturaleza rigen absolutamente todo cuanto acontece, sin excepciones.


\Section Sólo Dios es culpable

El edificio del conocimiento actual es materialista. Le guste a usted o
no, la ^ciencia^ materialista es el sistema de creencias que más
conocimiento abarca, con una colosal diferencia sobre cualquier otro. De
hecho, las críticas más serias al ^materialismo^ provienen, o bien de
ramas del materialismo que aceptan la mayor parte de sus postulados, o
bien de filósofos preocupados por las implicaciones éticas que supone la
negación de la ^libertad^.

Es muy claro que si desterramos completamente la libertad, eliminamos
igualmente la ^responsabilidad^ y la ^culpa^. También el ^mérito^. Si un
malhechor no es libre, entonces no puede evitar hacer lo que hace. Si no
puede evitarlo, si está forzado, entonces no es responsable. La
^justicia^ y la ^ética^ se quedan entonces sin argumentos.

Todo pierde su sentido, porque si todo acontece de acuerdo a las ^leyes^
naturales, sin que ninguna ^voluntad^ pueda desviar los acontecimientos
ni un punto de su curso, entonces, tanto un ^asesinato^, como la
ejecución de un reo, como una ^guerra^, son consecuencias de las leyes
naturales, y nadie es responsable. Sólo quien haya dictado las leyes
naturales es culpable. Solamente ^Dios^ es responsable; usted y yo no.
Y, para aquellos materialistas que no creen en Dios, nunca nadie es
culpable de nada, sino que simplemente ocurre así.


\Section La ley del más fuerte

Un caso concreto, que menciono porque es de los que con más frecuencia
alcanzan los medios de comunicación de masas y porque es un buen ejemplo
de lo que venimos contando, es el de la teoría darviniana de la
^evolución^ de las especies. La teoría de \[Darwin] establece como ^ley^
natural ``la preservación de las razas favorecidas en la lucha por la
^vida^''; éste es literalmente el subtítulo de su obra \(El origen de
las especies por medio de la selección natural).

Según el ^materialismo^ más ortodoxo, la supervivencia de los mejor
dotados en la lucha por la vida es una ley natural que justifica la ley
del más fuerte. Y punto, ya que, para el materialismo, el que doblega
por la ^fuerza^ no es responsable de nada, sino que simplemente la ley
natural es como es.

Algunos materialistas tratan de remediar la situación. Y lo hacen con
\[Wilson] argumentando que la capacidad de supervivencia de un individuo
puede mejorar actuando en grupo, en vez de hacerlo en solitario.

Los materialistas más agudos observan con \[Dawkins] que, bien mirado,
lo que evoluciona es el material genético, es decir, la ^información^
que pasa entre generaciones, de los padres a sus hijos. Y ocurre que, en
determinadas circunstancias, es favorable, para los propios ^genes^,
construir individuos que actúen cooperativamente en grupo. Según esta
rama gene-céntrica del materialismo biológico, los actores son los
genes, ya que lo que ocurre se explica en función de lo que favorece o
perjudica a los genes, y no a los individuos. Los individuos,
construidos de acuerdo a la información genética, son solamente los
vehículos que utilizan los genes para asegurarse su continuidad.


\Section La bomba atómica

Que la profundización gene-céntrica sustituya `la ley del [individuo]
más fuerte' por `la ley de la tribu más fuerte' tampoco mejora la ética
materialista. Si en un caso justifica el ^egoísmo^, en el otro ampara al
^racismo^. Elija usted.

No obstante, repito, la ^ciencia^ materialista es la que nos proporciona
más ^conocimiento^. Y no un poco más, sino, sin exagerar, enormemente
más. Exagerando, pero sólo un poco, infinitamente más. Sólo tiene usted
que pensar en los avances técnicos que disfrutamos, para entender lo que
digo. El ^teléfono^, la ^computadora^, y la ^bomba atómica^ son productos
de la ciencia materialista; sin ella no serían posibles. Y la ciencia
materialista ha alcanzado sus logros postulando que todo es materia, y
que la ^libertad^ es una ilusión.

A dónde quiero llegar con este sí pero no ---se preguntará usted. Lo que
quiero es que entienda que, aunque a usted y a mi nos parezca que la
^ética^ materialista es deleznable, no podemos dejarnos llevar por
nuestras emociones y rechazar, por cuestiones éticas, un postulado
epistemológico. En crudo, aunque rechacemos el materialismo porque
creemos que somos verdaderamente libres y responsables de nuestros
actos, la bomba atómica seguirá explotando delante de nuestras narices,
demostrando, a la vez, la verdad y la eficacia del ^materialismo^.

La verdadera solución al conflicto ético que plantea el materialismo ha
de ser una teoría filosófica que asimile todo el conocimiento científico
y que integre la libertad. No podemos rechazar la ^ciencia^. Al
contrario, hay que trasladar todo el conocimiento científico a otro
edificio y, por lo tanto, este nuevo edificio ha de ser diseñado con el
requisito de que quepa en él toda la ciencia. Y la libertad.


\Section Medir para creer

Sólo veo dos vías para hacer ^epistemología^ hoy. Una es consolidar el
edificio del ^saber^ vigente, que, repito, es la ^ciencia^ materialista.
La otra es tratar de sustituir el edificio actual por otro que salve
todo el conocimiento que contiene éste, pero que también pueda albergar
la ^libertad^. En cualquiera de los dos casos es necesario examinar los
defectos estructurales que presenta el edificio del saber actual, en un
caso para corregirlos y en el otro para evitarlos, y por ello nuestro
plan es legítimamente filosófico.

Desde \[Descartes], para hacer verdadera ^filosofía^ hay que tener
conocimientos científicos. Antes de él, en Occidente, lo imprescindible
era tener conocimientos de ^teología^ cristiana, e interpretar desde
ellos a \[Aristóteles]. Hoy, como venimos diciendo, es necesario tener
una visión científica del mundo. Si usted tiene una formación
científica, entonces mi tarea se facilita enormente, pero, si no es así,
entonces tendré que emplear medicinas paliativas. Y, como usted no puede
contestarme, me pondré en el caso peor, y me aprovecharé de que la
ciencia parte de lo ^obvio^.

La ^medida^ es la obviedad cuantificada, ya que es simplemente la
evidencia comparada con un patrón, también evidente. Y la ciencia acepta
como verdadera la medida, y nada más. Se ciñe a este criterio tan
estrictamente, que ni siquiera reconoce como existente aquello que
sabemos sin necesidad de medir. La ciencia rechaza la libertad porque no
puede verse ni medirse, a pesar de que somos libres, al menos usted, que
podría haber decidido no leer este libro, y yo, que podría haber
decidido no escribirlo.

Este discurso tiene el propósito de acreditar mi ^plan^, que,
recordemos, consiste en enseñarle a cuestionar lo obvio. El plan es una
simplificación legítima del ^método^ cartesiano, y servirá para evaluar
la fortaleza y la debilidad de la ciencia materialista, que es el
edificio del saber vigente.

Por fin, antes de comenzar, una última ^advertencia^, que debería ser
obligatoria en todos los escritos y parlamentos, pero que yo le hago
libremente y a iniciativa propia: éste es un ^libro^ tendencioso. No
presenta hechos objetivos, sino mis opiniones y las opiniones de los
autores citados, que además son ^caricaturizadas^ sin otro límite que mi
propio provecho. También hay obviedades, aunque cuestionables, como le
decía. Pero, basta de prologómenos. Empecemos ya.


\Section El dibujo

Lo ^obvio^ es lo que está delante de los ojos, lo que estamos viendo.
Nada hay más claro y distinto que lo que tenemos frente a nosotros y
podemos ver con buena luz y sin obstáculos. Decimos, de lo que no tiene
dificultad, que es obvio. Ver es ^fácil^, no cuesta esfuerzo.

Una característica de nuestra especie es nuestra ^habilidad manual^.
Podemos manipular objetos con una gran precisión. Pasar la hebra por el
ojo de la aguja es un ejemplo de nuestra pericia coordinando vista y
manos. Y la facilidad para enhebrar la solemos perder antes por falta de
vista que por torpeza.

A pesar de lo fácil que nos resulta ^ver^, ^manipular^ y coordinar ambas
actividades, ^dibujar^ muy bien es algo que solamente alcanzan los más
grandes de los maestros. Para el resto de nosotros, dibujar bien es una
tarea imposible, aunque nos resulte muy fácil garabatear.

Para emponzoñar más el asunto, una ^cámara fotográfica^ es capaz de
producir imágenes con total fidelidad. Y una cámara fotográfica no es,
precisamente, un artefacto complejo. Para fabricar una cámara
fotográfica basta una caja oscura con un pequeño agujero y material
fotosensible. Y cada uno de nuestros ojos es una cámara fotográfica, con
su ^pupila^ y su ^retina^.

Así que ésta es la primera ^aporía^ que le propongo: ¿por qué nos
resulta difícil dibujar?


\Section La línea

Con apenas cuatro rayas un ^caricaturista^ es capaz de dibujarle. Y, si
es bueno, sus amigos podrán reconocerle a usted en el ^dibujo^, aunque
sea una representación muy exagerada de sus rasgos. Esto es también muy
curioso, porque en realidad usted no está formado por líneas.

Si observa una pintura realista, por ejemplo un buen cuadro
^impresionista^, no verá en él ni una sola ^línea^. Tampoco trazan
líneas las cámaras fotográficas. Una ^cámara fotográfica^, o un pintor
^puntillista^ como \[Seurat], compone la imagen anotando el color de
cada punto. Para usted no lo sé, pero para mi ^dibujar^ consiste en
trazar líneas para tratar de representar la situación.

Es obvio que la técnica fotográfica para obtener ^imágenes^ es muy
diferente que el procedimiento de dibujo que utilizo yo. Y, si el método
fácil es el que necesita menos recursos, entonces el que yo empleo es el
difícil. Esto explica por qué me es ^difícil^ dibujar: porque me
complico innecesariamente.

La _cuestión<aporía> es ahora: ¿por qué nosotros dibujamos trazando
líneas?


\Section Una imagen vale más que mil palabras

Yo dibujo mal, lo admito. Pero usted tiene que admitir que un buen
dibujo resume en unas pocas líneas toda una ^imagen^. En realidad no
hace falta que lo admita usted, porque es algo que se puede verificar
empíricamente. Es un hecho informático que una ^fotografía^
(técnicamente un ^mapa de bitios^, en inglés \latin{^bitmap^}) ocupa
mucho más que un ^dibujo^ (técnicamente un ^gráfico vectorial^, en
inglés \latin{^vectorial graphic^}).

Afortunadamente para mis fines inmediatos, es casi seguro que usted ya
ha sufrido, voluntaria o involuntariamente, una inmersión informática de
mayor o menor grado. Me evita tener que explicarle cómo se mide lo que
ocupa una fotografía o un dibujo dentro de una ^computadora^. Es curioso
que fuera de ella podamos medir los cuadros por la longitud de sus
lados, y así en los catálogos encontramos que el lienzo de \[Velázquez]
titulado ``Las ^meninas^'' mide $318 \times 276\, \hbox{cm}^2$. En
cambio, dentro de la computadora se mide en bitios porque es
^información^.

Un ^bitio^ puede tomar dos valores, como {\sc sí} y {\sc no}, o $1$ y
$0$. Con dos bitios se pueden anotar cuatro posibilidades: $11$, $10$,
$01$ y $00$. Con tres las posibilidades son ocho ---$111$, $110$, $101$,
$100$, $011$, $010$, $001$ y $000$--- y cada nuevo bitio dobla el número
de posibilidades. De manera que con $n$ bitios el número de
^posibilidades^ es de $2^n$, o sea, dos multiplicado por dos
multiplicado por dos, y así hasta $n$ veces.

En un ^mapa de bitios^ la imagen se segmenta como con una rejilla
rectangular, y se anota el color predominante en cada celda. Cada una de
estas anotaciones se denomina ^pixel^, y se necesitan unos diez mil
($100\times100$) por centímetro cuadrado para que nuestro ojo no los
distinga. Para cubrir una página de este ^libro^ se necesitan, por lo
tanto, unos tres millones de píxeles. Y, dadas las peculiares
características del ^ojo^ humano, para la visión en ^color^ basta con
componer tres colores básicos. De manera que, si es suficiente una
graduación de doscientos cincuenta y seis valores, entonces necesitamos
ocho bitios ($2^8=256$) por color básico, lo que hace veinticuatro
bitios por pixel, que producen los más de dieciseis millones de colores
($2^{24}$) que se denominan color verdadero (\latin{true color}) en
informática. Y, en resumen, un mapa de bitios de alta calidad, en color,
y del tamaño de esta página ocupa unos setenta y dos millones de bitios.

Un ^gráfico vectorial^, en cambio, es la descripción escrita de la
imagen, y usa objetos. Cada ^objeto^ es de un tipo de entre varios
posibles ---puede ser una línea recta, o curva, o un polígono, o un
óvalo, u otro tipo--- y lleva asociadas propiedades. Por ejemplo, un
triángulo es un tipo de objeto que tiene definidas las siguientes
propiedades: la ubicación de los tres vértices con sus coordenadas
cartesianas $(x,y)$, el grosor y el color de los lados y el color del
interior. Y esto ocupa aproximadamente tantos caracteres como me ha
ocupado a mi la descripción anterior, o sea, pongamos unos cien.
Empleando un alfabeto de ciento veintiocho signos, como el viejo código
{\sc ^ascii^}, que puede incluir todas las letras minúsculas, todas las
letras mayúsculas, los diez dígitos, todos los signos de puntuación, y
otros caracteres especiales, bastan siete bitios ($2^7=128$) para
representar cada signo. De modo que cada objeto puede ocupar unos
setecientos bitios, y un gráfico con cien objetos, que ya es muy
complejo, ocupará unos setenta mil bitios.

Con estos números resulta que el mapa de bitios ocupa unas mil veces más
que el gráfico vectorial, y puede usted utilizar, como recurso
mnemotécnico, el ^dicho^ de que una imagen (o mapa de bitios) vale más
que mil palabras (o gráfico vectorial).


\Section La doble imagen

Voy a resumir las dos pistas que tenemos, para ver si ya podemos
alcanzar alguna conclusión. Primera pista: la manera más ^fácil^ de
reproducir una ^imagen^ es anotar el color de cada punto. Segunda pista:
la manera más económica de almacenar una imagen es quedarse con sus
trazos. Pruebe usted primero: ¿qué le sugiere esto?

Yo ya tengo una conclusión. Yo sospecho que las cosas suceden del
siguiente modo. Primero mis ojos recogen la imagen en la retina de la
manera fácil, esto es, como una ^fotografía^, anotando el color de cada
punto pixel a pixel. Pero esta imagen es demasiado costosa en términos
de información, así que los procesos ^perceptivos^ posteriores, la
^visión^ en este caso, extraen los trazos de la imagen, y esta imagen
más ^económica^, que es como un ^dibujo^, es la que se almacena, y es
con la que trabajan todos los demás procesos cognitivos. Todos.

Al dibujar, aunque tenga el modelo frente a mi, operan los mismos
procesos. De modo que al dibujar, o al pintar, se produce una
^interferencia^. Lo inmediato es recuperar la imagen a la que tengo
acceso directo, que no es la fotografía retinal, sino el dibujo mental
elaborado por mis procesos cognitivos. Si se hace bien, como hace el
buen ^caricaturista^, entonces las cuatro rayas son suficientes para
reconocer al caricaturizado. Pero, si no se tiene su talento, como es
tristemente mi caso, las líneas del dibujo que voy haciendo van
interfiriendo, en cuanto las voy trazando, con el mapa de bitios que a
modo de fotografía del modelo recogen mis ojos. Yo soy un mal dibujante
porque no sé deshacer este enredo, porque en mis representaciones mezclo
malamente mi imagen mental con mi imagen retinal.


\MTbeginchar(84pt,84pt,0pt);
 \MT: save u; u = w/6;
 \MT: y2 = y3 = h - y1 = u;
 \MT: x1 = w/2;
 \MT: z1 - z2 = (z3 - z2) rotated 60;
 \MT: pickup pencircle scaled 0.5pt;
 %\MT: draw (0,0)--(w,0)--(w,h)--(0,h)--cycle;
 \MT: z0 = (w/2, 1/3[y3,y1]);
 \MT: z6 - z0 = z0 - z1;
 \MT: z5 - z0 = z0 - z2;
 \MT: z4 - z0 = z0 - z3;
 \MT: draw z4 -- z5 -- z6 -- cycle;
 \MT: for i := 1 upto 3:
 \MT:  fill fullcircle scaled (2u) shifted (z[i]);
 \MT: endfor
 \MT: unfill z1 -- z2 -- z3 -- cycle;
\MTendchar;

\Section Las ilusiones

Hay un tipo de ^ilusión óptica^ que permite desenmascarar las rayas que
trazan nuestros procesos cognitivos; esas que no capta la ^cámara
fotográfica^, pero que yo dibujo. Mire la figura siguiente. Al mirarla
es imposible evitar ver un triángulo blanco y brillante que no está
dibujado.
$$\box\MTbox$$

Estos ^contornos subjetivos^ prueban que tendemos a trazar ^líneas^, es
decir, prueban que nuestra representación mental añade líneas a la
representación retinal.


\Section El punto ciego

Pero la más sorprendente de todas las ilusiones ópticas es una que no
puede verse. En la región de la ^retina^ del ^ojo^ de la que parte el
nervio óptico hacia el cerebro no hay receptores lumínicos. Esta región
de la retina se denomina ^punto ciego^ porque no capta luz. De manera
que la imagen retinal tiene un hueco. Tiene usted dos agujeros, uno por
cada ojo, de tamaño no despreciable en medio de su campo visual.

Ante esta información que acabo de darle, la única postura legítima que
puede adoptar un ^filósofo^ es mostrar una actitud de franca
incredulidad. No es posible que un fenómeno tan notorio, que afecta a
una función tan utilizada, pueda pasar completamente desapercibido
durante tanto tiempo. Si usted me ha creido, y no tenía noticias previas
de la existencia del punto ciego, entonces es que todavía no es
filósofo.

\MTbeginchar(\the\hsize,84pt,0pt);
 \MT: y1 = y2 = h/2;
 \MT: x2 - x1 = 8cm; x1 + x2 = w;
 \MT: z10 = (1truecm,0);
 \MT: z11 = z10 rotatedaround(z1,45);
 \MT: z12 = z10 rotatedaround(z1,135);
 \MT: z13 = z10 rotatedaround(z1,225);
 \MT: z14 = z10 rotatedaround(z1,315);
 \MT: pickup pencircle scaled 0.4pt;
 %\MT: draw (0,0)--(w,0)--(w,h)--(0,h)--cycle;
 \MT: pickup pencircle scaled 2pt;
 \MT: draw z11 .. z13; draw z12 .. z14;
 %\MT: fill fullcircle scaled (2truecm/1.3) shifted z2; % para compensar
 \MT: fill fullcircle scaled 2truecm shifted z2;
\MTendchar;

En cualquier caso, puede usted comprobar que ocurre efectivamente lo que
le he contado con el siguiente experimento. Cierre su ojo izquierdo y
mire fijamente con su ojo derecho a la `x'. Hay a una cierta distancia,
que es variable para cada persona, pero que suele ser de una cuarta
aproximadamente, a la que dejará de ver el círculo.
$$\box\MTbox$$

Yo no sé qué le parece a usted esta ^ilusión óptica^, pero para mi es
sugerente y sorprendente; es el más perfecto ejemplo de cuestionamiento
de lo ^obvio^. El círculo negro tiene dos centímetros de diámetro, y
deja de verse a una cuarta de distancia. ¿Cómo puede ser que no nos
percatemos del punto ciego, cuando sus efectos son tan aparentes en las
condiciones del experimento?

Por lo que estamos comprobando, la imagen retiniana no tiene líneas y
tiene un agujero notable. Y, sin embargo, vemos las líneas y no nos
percatamos del agujero.


\Section Qué es ver

Vamos a intentar cuadrar los datos que hemos ido recopilando. La
representación retinal sólo contiene las anotaciones del color de cada
punto del espacio visual, espacio que tiene dos agujeros debidos al
punto ciego de cada ojo. La representación mental simplifica, traza
líneas donde los cambios de color son más notables, y trata como un todo
indistinguible la región cerrada por una línea.

Cada agujero del punto ciego forma una zona sin color y, por lo tanto,
sin cambios de color, por lo que queda incorporada a sus alrededores. Si
ya va creciendo en usted el espíritu escéptico del verdadero ^filósofo^,
intentará comprobar esta teoría. Para ello le bastará preparar una
figura similar a la utilizada antes, pero cambiando el color del fondo,
que en este ^libro^ es blanco, y que puede usted variar a voluntad.

Yo he hecho este nuevo experimento y confirma la teoría, pero no debe
usted fiarse de mi. ¡Haga el experimento! Se preguntará usted por qué no
le ayudo: ¿por qué no aparece la figura coloreada en el libro? Podría
decirle que es para comprobar su actitud, pero la razón es más simple.
El libro resulta más barato si se imprime en blanco y negro, y haciendo
un círculo blanco sobre fondo negro la página quedaría demasiado
tintada. ¿Y en gris? En gris no es tan espectacular como en rojo.
Además, usted no necesita hacer un dibujo. Sabiendo como funciona el
experimento, le bastará poner una ^moneda^ pequeña, de menos de dos
centímetros de diámetro, sobre una superficie de color uniforme en la
que quede una marca, a modo de `x', a unos ocho centímetros de distancia
de la moneda.

Si ya lo ha hecho, le será más fácil admitir mi nueva definición: ^ver^
es el proceso cognitivo que construye la imagen mental a partir de la
retinal, y consiste en simplificar la ^imagen^ ^fotográfica^ captada en
la retina sustituyéndola por un ^dibujo^ en el que sólo cuentan los
trazos que compartimentan la escena.


\Section El objeto es obvio

Tome una cosa cualquiera, digamos una ^piedra^, y póngasela frente a
usted. ¿Qué ve? Vaya tontería ---se dirá usted--- pues veo una piedra.
Efectivamente, si yo le digo que tome una piedra y la ponga delante de
sus ojos, lo que verá es una piedra. Cuestionar obviedades es lo que
tiene, que se parece mucho a una ^tontería^. Si le pido, en estas
circunstancias, que me dibuje lo que está viendo, y no es mejor
dibujante que yo, trazará una línea cerrada en el centro de la hoja que
represente el contorno de la piedra y, si la piedra es oscura,
oscurecerá el interior de la línea. Finalizada esta operación dará por
terminada su tarea. Vale, yo haría lo mismo.

Si hubiéramos tomado una ^fotografía^ de lo que estaba viendo, y la
comparáramos con su ^dibujo^, encontraríamos muchas discrepancias,
tantas que no sería divertido jugar a buscar las siete diferencias. La
mayor discrepancia concerniría al ^fondo^, es decir, a todo cuanto veía
y que no era la propia piedra. El fondo sí aparecería en la fotografía,
pero no en el dibujo. Al dibujar se ha fijado en un único ^objeto^, la
piedra, y ha ignorado todo lo demás.

Etimológicamente, `^obvio^' y `^objeto^' son sinónimos. `Obvio' viene de
{\em vía}, o camino, y de {\em ob-}, que es un prefijo que alude a lo
que nos encontramos de frente, por lo que obvio viene a ser lo que se
nos presenta en el camino. `Objeto' comparte el prefijo {\em ob-} con
obvio, pero aquí se aplica a `echar', o `yacer', por lo que objeto es
cualquier cosa que está puesta frente a nosotros.

Según los diccionarios, `^objeto^' es `todo lo que puede ser materia de
^conocimiento^ o sensibilidad de parte del ^sujeto^, incluso este
mismo'. ¿Contradice la definición del diccionario a la etimológica?


\Section La triple imagen

Vamos a llamar ^objeto^, como manda la definición del diccionario, a lo
que conoce usted de la piedra que ha dibujado. Como la ha tomado en su
mano y la ha dibujado, sabe que en una zona del espacio que se encuentra
justo frente a usted hay un objeto de determinada forma y color, que
tiene cierto peso, y que es más o menos duro y rugoso. Sabemos que las
propiedades visuales del objeto, su forma y color, son simplificaciones
extremas de la imagen retinal, que convierten su fotografía en un dibujo
esquemático que ignora el fondo.

En cambio, el objeto etimológico está fuera, porque es lo que está
frente a nosotros. La nueva _cuestión<aporía> puede parecerle absurda,
pero es una consecuencia {\em obvia} de nuestras especulaciones: ¿está
la piedra dentro o fuera de usted?

Para poder mantener nuestros razonamientos sanamente y sin enloquecer,
vamos a distinguir la ^ob-piedra^, que estaría frente a nosotros, o sea,
fuera de nosotros, de la ^sub-piedra^, que sería nuestro conocimiento, o
imagen mental, y que estaría dentro de nosotros los sujetos. Para
completar la situación, podemos también definir la ^inter-piedra^, que
se correspondería con la imagen retinal, o fotografía, y que se
encontraría entre las otras dos.
$$\hbox{ob-piedra}\leftrightarrow
  \hbox{inter-piedra}\leftrightarrow
  \hbox{sub-piedra}$$

Con estas tres definiciones la pregunta absurda queda reformulada. Ahora
se trata de determinar cuál es la verdadera piedra, la ob-piedra, la
inter-piedra, o la sub-piedra.


\Section Yo soy subjetivista

Conocemos la relación que existe entre la ^inter-piedra^ y la
^sub-piedra^. Sabemos también que la inter-piedra es como una
^fotografía^ y que la sub-piedra es como un ^dibujo^. En cambio, sobre
la ^ob-piedra^, que es la piedra obvia, la que está fuera, apenas hemos
dicho nada.

La teoría ^objetivista^ postula que la ob-piedra es la verdadera piedra,
y que la ob-piedra es la causante de que en la retina se forme la
inter-piedra. Según esta teoría, la sub-piedra es la representación
mental de la ob-piedra, y la ^verdad^ consiste en la igualdad de la
ob-piedra con la sub-piedra.

No voy a engañarle, yo soy ^subjetivista^. Creo que no hay objetos
fuera. O sea, que no hay tal ob-piedra. Para dejar las cosas claras, sí
que creo que la inter-piedra tiene un origen ^exterior^. Lo que ya no
suscribo es que el exterior esté segmentado. Déjeme que me explique más
despacio.

Lo primero que quiero que observe es que, si se fija bien, en la imagen
retinal no hay ninguna piedra. Lo único que hay son puntos de color.
Pero esto nunca nadie lo ha puesto en duda, es decir, nadie piensa que
la inter-piedra sea una verdadera piedra. Supongo que usted tampoco.

Lo segundo que debe advertir es más sutil. Fíjese que todo cuanto
sabemos de la piedra, toda la ^información^ que tenemos sobre ella, es
precisamente lo que hemos llamado sub-piedra. La ob-piedra queda fuera
de nosotros, fuera de nuestro alcance, inaccesible. Lo único que nos
llega de fuera es lo que captan nuestros sentidos, la inter-piedra. Todo
cuanto podemos decir y pensar de la piedra se refiere necesariamente a
la sub-piedra. Porque la sub-piedra es todo el ^conocimiento^ que
nosotros tenemos de la piedra.


\Section La cadena causal

Las propiedades del ^objeto^ son ^información^ sobre el objeto, y forman
parte de la representación mental. Así, cuando hablamos del color o de
la dureza de la piedra nos referimos a propiedades de la sub-piedra. El
color es el resultado del procesamiento visual de la imagen retinal, y
la dureza del correspondiente proceso tactil.

Solamente tenemos acceso directo a la representación mental de la
piedra, que venimos llamando ^sub-piedra^. Pero, suponiendo que el
objetivismo fuera correcto, habría una cadena causal desde la
^ob-piedra^ hasta la sub-piedra, a través de la ^inter-piedra^.
Entonces, aunque nuestras afirmaciones sobre la piedra usarían la única
información disponible, esto es, la sub-piedra, nuestra intención sería
referirnos, en última instancia, a la verdadera piedra, que para los
objetivistas es la ob-piedra.

A mi, repito, no me vale el ^objetivismo^. Los ^objetos^ son regiones
del espacio que tratamos mentalmente como una unidad a la que atribuimos
algunas propiedades. Con estas propiedades resumimos toda la enorme
cantidad de información que nuestros sentidos captan del área en
cuestión. La razón por la cual nuestra representación mental utiliza
objetos es porque así se _simplifica<comprimir> el enorme caudal de datos
^percibido^. Es tarea del objetivismo explicar por qué, si la
representación mental es sencilla y remeda al objeto exterior, ocurre
que la imagen retinal intermedia es tan compleja. Y es su tarea, porque
es el objetivismo el que propone la cadena: objeto exterior causa imagen
retinal causa representación mental.
$$\hbox{objeto exterior}\Rightarrow
  \hbox{imagen retinal}\Rightarrow
  \hbox{representación mental}$$


\Section El bosque

Me gusta el ejemplo del ^bosque^. Súbase a un monte, o más fácil,
busque en un libro la fotografía de un bosque y respóndame,
¿es el bosque un ^objeto^? Para mi sí, sin problemas, porque puedo
verlo. Pero para un objetivista la cuestión es
más difícil. Si la fotografía del bosque lo ha retratado completo desde
la lejanía, entonces es fácil fijar sus límites. Pero todos sabemos que,
al acercarnos, la ^frontera^ del bosque se hace difusa. De algunos
árboles de la periferia es controvertido afirmar tanto que pertenecen al
bosque como lo contrario. Y, lo que es peor, desde dentro del bosque no
se ve el bosque, se ven árboles. Así que traslado la pregunta del bosque
al árbol, ¿es el ^árbol^ un objeto?

La ciencia nos explica que con los árboles sucede exactamente lo mismo
que con los bosques. Así como un bosque es un conjunto de árboles, un
árbol es un conjunto de células, y una ^célula^ es un conjunto de
átomos. La diferencia es de tamaño. Así que las dificultades
objetivistas del bosque, del árbol, de la célula y del ^átomo^ han de
ser las mismas, aunque a distinta escala.

Si las cosas fueran como nos las cuenta el ^objetivismo^, entonces al
acercarnos al ^cerco^, éste crecería de tamaño y lo veríamos mejor. Pero
ocurre justo lo contrario. Si el ^límite^ es un artefacto causado por
los procesos ^perceptivos^ para simplificar la imagen retinal, entonces
éste aparecerá solamente cuando sea preciso para segmentar la imagen. Y
esto sí que explica por qué desde dentro del bosque no se ve el bosque;
porque al ocupar el bosque la imagen completa, distinguir al bosque del
no-bosque no simplifica nada.

No hay límites ahí fuera. Las líneas son artefactos perceptivos que
acomodan el exterior a nuestros escasos recursos cognitivos. Y si fuera
no hay cercos, tampoco hay objetos.


\Section Una situación extraña

Tal vez le parezca a usted que las conclusiones que vamos alcanzando
configuran una situación extraña. Tiene razón. La ^percepción^ y los
sentidos tienen el propósito de darnos ^información^ sobre lo que sucede
fuera. Y, sin embargo, yo le insisto en que la piedra no está fuera,
sino dentro.

Al hablar de la piedra queremos referirnos al ^exterior^, es cierto, lo
concedo porque así lo creo. Ocurre que hay ciertas situaciones
exteriores que hacen que usted y yo veamos una ^piedra^. Ver una piedra
significa que en la mente se dibuja una piedra, la ^sub-piedra^, pero no
implica que fuera haya una, la ^ob-piedra^. Esta última hipótesis es
innecesaria y engañosa. Y, cuando vemos una piedra, es legítimo decir
`veo una piedra', que, si nos interesa, podemos traducir al lenguaje
objetivista como `hay una piedra frente a mi'.

Por otro lado, no podemos hacer otra cosa. En las condiciones en las que
vemos la piedra, la vemos. Igual que no podemos evitar formar el
triángulo en la ilusión de los ^contornos subjetivos^, ni podemos evitar
que en determinadas circunstancias el agujero del ^punto ciego^ haga
desaparecer el círculo de dos centímetros, tampoco podemos evitar ver la
piedra. Es más, toda la información que disponemos del exterior está
objetivada, o sea, compartimentalizada en ^objetos^, y, por eso, al
hablar sobre el exterior tenemos que hablar de objetos. Para hablar de
lo que está fuera tenemos que hablar de objetos, aunque fuera no hay
objetos.

Así que tenía usted mucha razón: esto es muy extraño. Aun así, nosotros
no somos los primeros en percatarnos de nuestra extraña situación con
respecto al exterior. Ya lo advirtió, a finales del siglo {\sc xviii},
otro de los grandes filósofos: \[Kant]. Pero antes, déjeme que le hable
de \[Hume].


\Section Hume

Si no fuera porque escribió sus mejores obras en la primera mitad del
siglo {\sc xviii}, parecería que \[Hume] se había hecho ^filósofo^
leyendo este ^libro^. ¡Qué más quisiera yo! El caso es que el logro
principal de este filósofo escocés consistió en cuestionar el fundamento
de la ^ciencia^: la experiencia.

\[Hume] se preguntó por qué aprendemos de la experiencia. Respuesta:
porque la naturaleza se repite, de modo que si observamos que, dadas
ciertas condiciones, se siguen siempre las mismas consecuencias,
entonces prescribimos que las condiciones son las ^causas^ de las
consecuencias, que son sus efectos. El ^experimento^ replicable que
utiliza la ciencia para sancionar las ^leyes^ naturales se basa en este
mecanismo de ^inducción^.

Pues bien, \[Hume] nos hizo ver que ``si hubiera sospecha alguna de que
el curso de la naturaleza pudiera cambiar y que el pasado pudiera no ser
pauta del futuro, toda experiencia se haría inútil y no podría dar lugar
a inferencia o conclusión alguna. Es imposible, por tanto, que cualquier
argumento de la experiencia pueda demostrar esta semejanza del pasado
con el futuro, puesto que todos los argumentos están fundados sobre la
suposición de aquella semejanza''.

Lo repito, pero ahora con mis palabras del siglo {\sc xxi}. Las leyes
científicas se basan en que el ^futuro^ será como ha sido el ^pasado^, y
de ese modo sacan partido de la experiencia pasada. Pero, precisamente
la proposición `el futuro será como ha sido el pasado', no puede
fundarse en la experiencia, porque se cerraría un ^círculo vicioso^. Y,
al fallar el fundamento de la base de la ciencia, la ciencia se queda
sin cimientos.

Para \[Hume] esto significa que el ^conocimiento^ humano no tiene su
fundamento en la razón sino que, con sus propias palabras, ``se trata de
una operación del ^alma^ tan inevitable cuando estamos así situados [en
posición de prever] como sentir la pasión de amor cuando sentimos
beneficio, o la de odio cuando se nos perjudica. Todas estas operaciones
son una clase de ^instinto^ natural que ningún razonamiento o proceso de
^pensamiento^ y ^comprensión^ puede producir o evitar''.


\Section Kant

A \[Kant] debió de parecerle pobre la solución escéptica de \[Hume]. Lo
que es seguro es que sus bien argumentadas dudas le escocieron. Por eso,
el ^filósofo^ alemán dedicó muchos esfuerzos a investigar las
``operaciones del ^alma^'' a las que se refería el escocés. Fruto de
estas investigaciones, \[Kant] descubrió que la mente se ve obligada a
clasificar los datos de la experiencia en cajas o compartimentos que
denominó ^categorías^. Y la relación de ^causa^ a efecto, que preocupaba
a \[Hume], es una de las doce categorías que identificó.

\[Kant], además de profundizar en el funcionamiento de la razón, rebatió
el ^objetivismo^. Nos mostró que no podemos siquiera pensar en el
^objeto^ en sí, que denominó \latin{^Noumenon^}. Este \latin{Noumenon}
sería el objeto tal cual es, es decir, sin moldear por las categorías y,
por lo tanto, exento de toda deformación subjetiva. La cuestión es que,
haya o no tal \latin{Noumenon}, los sujetos no podemos alcanzarlo.
Podemos expresarlo a la manera más rotunda de una ^tautología^: no
podemos pensar el objeto no pensado.

Esta revisión histórica nos enseña que la extraña situación de tener que
utilizar ^objetos^, que no existen fuera, para referirnos al ^exterior^,
no es cosa de hoy.


\Section El mito del restaurante

\[Platón] concibió hace dos mil quinientos años el mito de la ^caverna^
para explicar la situación del hombre frente al ^conocimiento^. Ya es
hora de modernizarlo.

Suponga usted que unos hombres han vivido toda su vida en el comedor de
un ^restaurante^. Allí han nacido y nunca han salido de allí. Ignoran
todo cuanto ocurre fuera del comedor, incluso en la cocina.
Afortunadamente, la carta es larga y variada y pueden ordenar a su
antojo al camarero. Eso sí, el camarero es de muy cortas luces, no sabe
hablar, sólo es capaz de entender los nombres de los platos que le
piden, y no puede proporcionar ninguna información a los comensales.

Lo principal es que toda la ^información^ que estos desdichados reciben
del mundo exterior les llega en los platos que solicitan. Dado que la
carta del restaurante es extensa, podrían llegar a acumular muchos
conocimientos de ^biología^. Vamos a suponer que disponen de
instrumental de investigación, como microscopios, que son personas
inteligentes y curiosas, y que efectivamente adquieren mucha sabiduría,
a pesar de su infortunio.

Pues, seguramente, la ^ley^ fundamental que podrían enunciar los sabios
comensales sería que el universo exterior está segmentado, de modo tal
que no hay cosa alguna en él que sobrepase el tamaño de un plato.
Seguramente sería así, porque la mayor de las regularidades que
observarían sería que todo cuanto les traen, absolutamente todo, se lo
traen en un plato.

A nosotros nos es muy fácil ver que están equivocados. Nosotros sabemos
que el alimento se trozea y se raciona para adecuarlo a la dieta del
comensal. Como dijo el sabio \[Protágoras], ``el hombre es la medida de
todas las cosas''.

Llegados a este punto, usted, con su proverbial perspicacia, ya habrá
descubierto mis aviesas intenciones, que no son otras que ridiculizar a
los ^objetivistas^. Pues sí, como ha adivinado, los comensales que no
sospechan que sus viandas están preparadas son los objetivistas, que no
tienen en cuenta que los objetos, tal como los conocemos, son productos
elaborados. Son necios que no sospechan, ni saben, que el propio
^objeto^ es el resultado de una limitación que impone el ^sujeto^, y,
por lo tanto, tampoco sospechan, ni saben, que el objeto es un artefacto
de la ^percepción^.


\Section Qué hay ahí fuera

Saber lo que se cuece en la cocina no proporcionaría una visión directa
del mundo ^exterior^ a los cautivos del comedor del ^restaurante^, pero,
por lo menos, les sirviría para corregir algunos errores de bulto. A
nosotros nos sirve para dudar de la existencia de objetos fuera. Pero si
fuera no hay objetos, ¿qué hay ahí fuera?

Si la alegoría del restaurante es correcta, entonces el conocimiento
está troceado, cocinado y puede llevar aditamentos, pero lo fundamental,
la carne, viene de fuera. Podemos decir que la carne traspasa la cocina,
o sea, atraviesa la frontera y trasciende al interior del comedor.

Hay que dudar de todo, como nos enseñó \[Descartes], y gracias al
restaurante podemos expresar algunas nuevas dudas. Lo primero es dudar
de la propia analogía, ¿será válida? Hay que preguntarse si habrá algo
exterior que traspase los sentidos y la ^percepción^ y alcance el
conocimiento. En el lenguaje epistemológico, la cuestión parece, ¡a la
vez!, más elevada y profunda: ¿hay algo trascendente en el conocimiento?

Yo pienso que sí, y voy a explicarle mis dos razones. La primera es que
si nada pudiera pasar del exterior al interior, entonces la
^comunicación^ sería imposible, y yo no podría comunicarme con usted.
Francamente, no escribiría si no pensase que escribiendo puedo influir
en usted. La segunda es que pienso que el ^conocimiento^ nos sirve
precisamente para interactuar con el exterior. No veo qué sentido
tendría el conocimiento si no fuese así; aunque esto no prueba nada y
únicamente confirma mi falta de imaginación. En fin, la opinión
negativa, que es la del ^solipsismo^, simplemente no da más juego.

Si, como sostengo yo, hay efectivamente algo trascendente en el
conocimiento, entonces se plantea inmediatamente la segunda cuestión:
¿qué es lo que trasciende al conocimiento?


\Section Todo es información

Buscamos algo que, como la carne en el ^restaurante^, tenga su origen
fuera y pase, más o menos alterado, al interior. Algo que esté en el
^conocimiento^ y fuera. Algo que traspase los sentidos y la percepción.

La ^percepción^, como vimos, es un proceso que simplifica lo que captan
los sentidos: _condensa<comprimir> la ^información^ recibida. De modo
que la percepción elabora la información recogida, tal como el cocinero
condimenta y prepara la carne comprada en el mercado. El conocimiento es
información, más o menos elaborada, así que, si estamos en lo cierto, lo
que hay fuera y dentro es información.

Esta conclusión ya es muy contraria al postulado materialista, puesto
que supone negar la primacía de la materia, de la ^substancia^. La
^masa^, o aun mejor, según \[Einstein], la ^energía^, puede ser una
unidad de cuenta interesante, pero no es de lo que está hecho el
universo. El ^universo^ está hecho de información. Todo es información.
Ahora la pregunta es: ¿qué es la información?


\Section Darwin

El siglo~{\sc xix} no produjo grandes filósofos. Mi parcial y
tendenciosa interpretación es que se reconoció el inmenso logro
alcanzado por \[Kant], pero que nadie fue capaz de asimilar el
^subjetivismo^ resultante de su teoría. Como consecuencia nos podemos
ahorrar el estudio de los filósofos alemanes de la H, tanto del
siglo~{\sc xix} como del~{\sc xx} ---\[Hegel], \[Husserl] y
\[Heidegger]---, que intentaron, de uno y otro modo, salvar el
insalvable ^objetivismo^.

En cambio, las ^ciencias^ en el siglo~{\sc xix} avanzaron en el
esclarecimiento del concepto de ^información^, sobre todo merced a dos
teorías. Una de ellas es la teoría de la ^evolución^ de las especies por
^selección natural^ de \[Darwin]. Aunque no quiero decir que la teoría
de la evolución trate de la información; o tal vez sí. Veamos.

La teoría de \[Darwin] se basa en una previa de \[Malthus] sobre la
^población^ de las naciones. \[Malthus] observó, finalizando el
siglo~{\sc xviii}, que sin limitaciones de alimento ni de otro tipo, la
población aumenta de forma geométrica, esto es, por multiplicación, como
en la secuencia 1, 2, 4, 8, 16, 32, 64. Que, por otro lado, no es
posible aumentar la producción de alimento más que de forma aritmética,
o sea, por suma, como en 1, 2, 3, 4, 5, 6, 7. Que, como consecuencia del
distinto modo de crecimiento de estas dos secuencias, la ^miseria^ no
puede combatirse con la ^caridad^, y que la única solución para evitar
el ^hambre^ es contener el aumento de la población.

Leyendo a \[Malthus], \[Darwin] se percató de que la limitación de los
recursos y la fecundidad de la vida resultan en un proceso continuado de
^selección natural^. La mayoría de los organismos mueren, y sólo
^sobreviven^ los mejor adaptados, ya sea porque son capaces de acceder a
más recursos que sus congéneres, o porque son más resistentes, o más
económicos, o por cualquier otra causa que les dé ventaja. Como, además,
la reproducción no es perfecta, de modo que los hijos no son réplicas
exactas de sus padres, las especies irán evolucionando de manera
adaptativa al medio en el que se desenvuelven.

La teoría darviniana de la ^evolución^ de las especies por selección
natural explica cómo es posible diseñar sin diseñador. El ^diseño^ es
otra operación en la que interviene la ^información^.


\Section Aristóteles

Un hecho curioso, que todo ^filósofo^ ha de saber aprovechar, es que
cada ^ciencia^ acumula enormes, yo aun diría más, ingentes cantidades de
conocimientos que explican los asuntos más variados y recónditos y, sin
embargo, suele carecer de una respuesta adecuada para la ^pregunta^ más
básica, que es la pregunta acerca de lo ^obvio^. Por ejemplo, la física
no puede contestar qué es la ^materia^, o qué es la ^energía^. En el
caso de la ^biología^, la pregunta sin respuesta es: ¿qué es la ^vida^?

Debo prevenirle a usted, filósofo en ciernes, para que no abuse de este
arma contra los científicos, a no ser que tenga preparada una buena
defensa para neutralizar el contraataque. Porque la ^filosofía^ adolece,
no solo de éste, sino también del problema inverso. La filosofía produce
definiciones con demasiada facilidad, y tiene dificultades para
proporcionar explicaciones informativas y adecuadas.

Lo cierto es que la definición de la vida ---qué es la vida--- no la
proporcionan los biólogos, que son quienes más saben del tema, sino
nosotros los filósofos. Se trata, al fin y al cabo, de una obviedad, ya
que todo el mundo conoce la diferencia entre lo que está vivo y lo que
no. En este caso, el primero que abordó seriamente la cuestión fue
\[Aristóteles], que fue filósofo y biólogo, y discípulo de \[Platón]. Se
preguntó por la vida comparando un cuerpo vivo con otro recién muerto.
La única diferencia entre ellos es precisamente la vida, pero en qué
consiste esta diferencia.

\[Aristóteles] utilizó la distinción entre ^acto^ y ^potencia^. El
cuerpo puede estar actualmente muerto y, por lo tanto, el cuerpo sólo
tiene vida en potencia. La vida es, entonces, aquello capaz de hacer
actual lo que sólo es potencial. Para \[Aristóteles], a quien le gustaba
distinguir la ^forma^ de la ^materia^, la vida es forma, ya que es mera
capacidad y no pesa. En términos modernos, la vida de \[Aristóteles] es
lo que llamamos ^información^. La información que dirige al cuerpo vivo
en busca de ^alimento^, refugio o pareja, es lo que le falta al cuerpo
muerto y lo convierte en un ^cadaver^ inerte. La vida es información.


\endinput
