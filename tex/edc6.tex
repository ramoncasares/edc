% edc6.tex (RNMCG20050405)


\Section La autobiografía

Por las razones que venimos sopesando, cada uno de nosotros es un
^sujeto^ que va acumulando a lo largo de su vida ingentes cantidades de
^información^, tanto consciente como inconsciente. Ahora nos
preocuparemos únicamente de la información consciente, que a veces es
del tipo `uno más uno son dos', y otras es como `está lloviendo'. Para
facilitar el manejo de tanta información, el sujeto va clasificándola en
diferentes apartados de ^conocimiento^ y va elaborando teorías que la
ordenan y la ^comprimen^. La primera operación es una discriminación
binaria que atribuye una parte del conocimiento a su propio hacer y
pensar, a su ^yo^; del resto no se ^responsabiliza^.

Así que el yo de cada sujeto humano se compone de esta información
verbal conjuntivamente acumulada, que es en buena parte histórica, que
es distinta para cada persona, y que el sujeto se atribuye a sí mismo.
De igual manera que no debe usted confundir al ^presidente^ con la
corporación que dirige, tampoco debe confundir al yo con el sujeto. El
yo es como se ve a sí mismo el sujeto. El yo es un modelo consciente que
se nutre únicamente de datos conscientes, o sea, de palabras. El ^yo^ es
la ^autobiografía^ siempre inacabada de cada sujeto. Somos nuestras
^palabras^.

Yo ya le he prevenido severamente en contra de las autobiografías, de
modo que, si le han aprovechado a usted mis consejos, ya habrá adivinado
que el yo es una novela que embellece algunos de los sucesos que le ha
tocado vivir a cada uno, y que ignora el resto; créame, su yo y el mío
también son de esta aparente y falsa guisa.

Por eso, una tarea de la ^psicología^ consiste en obedecer el viejo
mandato ^délfico^: ``conócete a ti mismo''. O sea, consiste en intentar
que esa autobiografía que es su yo, de usted o de cualquier otro, sea lo
más parecida posible a la biografía no autorizada que escribiría de
usted su mayor enemigo, si éste hubiese estado con usted en todo
momento. Bueno, creo que he exagerado algo la situación para hacerla más
memorable, que es un recurso didáctico discutible, porque lo que se
recuerda no es exacto y puede confundir; así que ponga, por favor, a un
historiador ecuánime en vez de a su mayor enemigo.


\Section Yo soy libertad para no morir

Los ^sujetos^ somos aquellos resolutores de un ^problema aparente^ que
somos capaces de resolver por ^traslación^ de resoluciones. Para poder
trasladar resoluciones es necesario un ^motor sintáctico^. Los lenguajes
asociados a los motores sintácticos son los simbólicos, que se
caracterizan por su ^sintaxis recursiva^ y por sufrir necesariamente de
paradojas. Si algo de todo esto le suena extraño, debe repasar lo leido,
porque me estoy limitando a recapitular.

Sigo, con un inciso también ya visto. Una consecuencia de que un
problema sea aparente es que la ^información^ que puede obtenerse de él
se conforma a la ^ley de la información incesante^, que, cuando la
información puede ser acumulada, se asimila a la ^ley de la información
creciente^. Tiempo y libertad, ¿se acuerda?

Los sujetos vivos somos aquellos resolutores del ^problema aparente de
la supervivencia^ que somos capaces de resolver por traslación de
resoluciones. Los hombres, y me refiero tanto a las hembras como a los
machos de la especie \latin{^homo sapiens^}, somos sujetos vivos.
Nuestro ^lenguaje simbólico^ es el ^habla^, que tiene diversos
dialectos, como el ^castellano^, el ^inglés^ y el ^japonés^; también el
^gallego^. Nuestro habla es analítica, y con esto quiero decirle que
componemos la información sintáctica por acumulación conjuntiva.

Los sujetos, y solamente los sujetos, tenemos ^consciencia^,
^autoconsciencia^ y ^empatía^. Como somos conscientes nos representamos
el mundo como un ^enigma^, como un ^problema^ a resolver. Como somos
autoconscientes nos representamos a nosotros mismos como resolutores de
ese problema que es el mundo. Como tenemos empatía también nos
representamos a los demás seres vivos como resolutores enfrentados al
problema de la supervivencia.

Con todos estos repasos, y siendo que el ^sujeto^ no se ve
conscientemente a sí mismo como sujeto, sino como yo, llego a mi
definición concisa del ^yo^. {\em Yo soy libertad para no morir}. Esta
definición es breve pero suficiente, porque es densa. Que sea
`^libertad^', pero no completa ni absoluta, sino condicionada, implica
que el yo se identifica con un ^problema^. Y `para no morir' indica que
el sujeto tiene como meta la resolución del ^problema de la
supervivencia^, o sea, que se ve introspectivamente como ^resolutor^ del
problema de la supervivencia.

Es decir, que así como la definición larga del yo era concreta e
individual, o sea, autobiográfica, en cambio la corta es genérica, y
vale para un diccionario.


\Section Una afirmación temeraria

Son muchas las consecuencias que se pueden extraer de asimilarnos a
sujetos del problema aparente de la supervivencia según la ^teoría del
problema^, pero no voy a seguir por ese camino porque antes de finalizar
este libro es necesario afianzar la ^teoría de la información^, y veo
que aún le quedan algunas dudas sobre asuntos básicos. Además, así dejo
materia para escribir otro libro.

Desde luego, si es cierto, como yo le aseguro, que el ^problema de la
supervivencia^ es el único origen de toda la ^información^, entonces
todo nuestro ^conocimiento^ se referirá directa o indirectamente a él.
Esta afirmación puede parecerle temeraria, si piensa que hay datos que
no son en absoluto problemáticos. Qué conexión puede existir entre la
^torre de Hércules^ y el problema de la supervivencia.

La torre de Hércules es un faro que sirve de guía a los barcos en la
obscuridad de la noche. Ha salvado la vida a muchos marineros, para los
que la relación entre la torre y la vida está clara. Vale, vale, ya sé
que usted no es marino, y que tampoco le salva la vida a nadie el dato
de que la construcción primitiva de la torre de Hércules se remonta a
los romanos.

Mi opinión es que la recogida de datos históricos confiere ventaja a los
que la practican, y la ^evolución^, como suele ocurrir en tales
circunstancias, ha seleccionado este ^comportamiento^. Es un hecho que
acumulamos la ^información^, ya sea histórica o de cualquier otro tipo.
Podría ser de otra manera, pero es así, y un ^filósofo^ ha de
preguntarse por qué. Para mi, ya lo vimos, se trata de una manera de
hacer frente a la ^ley de la información incesante^.

Por otro lado, tomemos, por ejemplo, a las ^hormigas^ y respóndame: ¿no
le parece que todo cuanto hacen y dejan de hacer las hormigas tiene el
objetivo único de sobrevivir? En el caso de las hormigas, podemos
ponernos de acuerdo en que toda su actividad tiene como objetivo la
supervivencia de la colonia. Que sea la supervivencia de la colonia y no
la individual es un asunto interesante, pero que no viene al caso.

¿Qué pasa con los animales más parecidos a nosotros? Si elegimos de
nuevo a su ^perro^, puede usted argumentarme que no siempre se dedica a
sobrevivir, porque no sólo come, sino que también juega. En la vida
perruna ya se distingue el ^ocio^ del ^negocio^. Pero, para que esta
distinción me contradijera, tendría que suceder que el ocio no tuviera
significado evolutivo alguno. Si, por ejemplo, se demuestra que el
^juego^ es un entrenamiento que sirve de preparación para enfrentarse a
situaciones que pueden encontrarse posteriormente en el negocio de la
^vida^, entonces también se juega para sobrevivir.

Yo creo que también el ocio, en todas sus modalidades, sirve, en último
término, para sobrevivir. Piense que todas las actividades pasan
necesariamente por la criba de la ^selección natural^, simplemente
porque si un animal se descuida, es presa de otro.


\Section La soberbia

Y, ¿qué hago yo escribiendo desde hace años libros que no me reportan
beneficio alguno? ¿No es este mismo ^libro^ que está usted leyendo, y
que trata de asuntos sin ninguna importancia práctica, una prueba
irrefutable de que no todas nuestras actividades tienen relación con el
^problema de la supervivencia^?

Ya hemos hablado antes del ^yo^ como ^autobiografía^, y de como uno
tiende a autoengañarse para embellecerse ante sí mismo. Es decir, que
aunque mi situación como autor sea así de triste, esto no impide que
{\em yo} siga esperando alcanzar la ^fama^ y la ^gloria^, y hasta
dinero, escribiendo abstrusas filosofías. Es el yo el que toma las
decisiones conscientes que, por supuesto, teniendo una opinión tan
indulgente de sí, pueden fácilmente estar equivocadas. Y con mucha
frecuencia lo están.

Esta circunstancia individual se traslada, punto por punto, a nuestra
opinión sobre nosotros mismos como especie. Nos creemos que somos los
hijos predilectos de ^Dios^, y pensamos que somos el centro mismo del
universo y que todo gira alrededor de nosotros. Nos creemos ángeles
miríficos entregados a sublimes hazañas, o incluso taimados y astutos
^demonios^ que montan trampas arteras e ingeniosas, antes que bajos,
vulgares y prosaicos animales, que es lo que somos.

En fin, yo creo que todo nuestro diseño como ^especie^ ha sido tamizado
por la ^evolución^ darviniana. Y esto quiere decir que nuestro propósito
final es sobrevivir. Sea humilde, abandone por un momento su soberbia, y
le será fácil estar de acuerdo conmigo.


\Section La muerte es una estrategia de la vida

Y el ^suicidio^, ¿no refuta el suicidio la tesis de que todo lo hacemos
para sobrevivir? Pues no, sino que es justo al revés, porque el suicidio
prueba mi teoría, y además fácilmente.

El ^problema de la supervivencia^ es el problema al que se enfrenta la
^evolución^ darviniana, de modo que es el problema de toda la ^vida^
considerada como un ente único, como nos enseña \[Lovelock]. Entonces,
al igual que se entiende que, en ocasiones, es necesario amputar un
miembro para salvar a un individuo, también debe comprenderse que, a
veces, es necesario que muera un individuo por el bien de todos. Si un
individuo determina que él mismo es un estorbo para su colectivo, o
simplemente que su muerte hace más bien que mal a su comunidad, puede
acabar matándose. Esto explica los suicidios altruistas.

El ^envejecimiento^ está relacionado con estas consideraciones y es muy
interesante. La ^evolución^ darviniana descubrió que es conveniente que
todos los individuos envejezcan, de tal suerte que, incluso aquellos que
consigan evitar los accidentes y sean inmunes a las enfermedades,
terminen por morir. Las especies que no emplean esta estrategia no
evolucionan con la velocidad suficiente, y terminan por extinguirse al
competir contra otras especies que están mejor adaptadas porque sus
individuos sí que se mueren. Fíjese que, según este razonamiento, en el
que he vuelto a pecar de ^diacronismo^, la ^inmortalidad^ individual es
un lastre para la ^especie^, y no el ^suicidio altruista^. Ya tengo otra
sentencia aparentemente paradójica: la ^muerte^ es una estrategia de la
^vida^.

He explicado demasiado bien el suicidio, tanto que lo que ahora parece
amenazar mi teoría es, por el contrario, el anhelo humano de alcanzar la
inmortalidad individual. ¿De verdad tengo que explicarle por qué no
queremos morirnos? Vale, vale, si es muy sencillo. El ansia de
inmortalidad no es más que la racionalización del ^instinto de
supervivencia^ que los humanos heredamos de nuestros antepasados.
Recuerde que somos unos sujetos diseñados por ^acreción^, y vea que una
especie cuyos individuos carecieran de instinto de supervivencia se
extinguiría, ¿o no?


\Section La vida es absoluta

He de matizarle la explicación del ^suicidio altruista^. Explica que la
^evolución^ puede diseñar seres altruistas al punto de llegar a matarse,
pero no específicamente a suicidarse. Deténgase para discernir
`^suicidarse^' de `^matarse^'. Matarse es causarse la muerte a uno
mismo. Suicidarse es resolver que la solución es matarse, y hacerlo.
Suicidarse exige que un individuo razone sobre su propia vida, de modo
que solamente puede suicidarse un ^sujeto^. Y el sujeto no tiene acceso
al ^problema de la supervivencia^, sino a su interiorización, que es el
^problema del sujeto^.

Esto me recuerda que, antes de continuar, he de informarle sobre una
seria dificultad que afecta a toda mi teoría. Efectivamente, yo tampoco
soy otra cosa que un ^sujeto^, así que tampoco yo tengo acceso al
problema de la supervivencia, y, en consecuencia, debería ser más
prudente y no hablar tanto sobre aquello de lo que nada sé.

Esta objeción es legítima, y ante ella sólo tengo una endeble pero
franca defensa: simplemente que, llegados hasta los fundamentos primeros
de cualquier teoría, se advierte necesariamente que ellos mismos sobre
nada se sostienen, o no serían los primeros. La validez de los
fundamentos primeros se mide por el peso que pueden llegar a soportar;
no hay otro modo.

En conclusión, el fundamento primero de mi teoría es la ^vida^. La
consecuencia es que no puedo definir la vida, que es simplemente
absoluta en sí misma porque no puede referirse a ninguna otra cosa. Se
explica para vivir, pero vivir no tiene explicación.


\Section El suicidio

En fin, retomando el tema del ^suicidio^, los sujetos no podemos
alcanzar el problema contra el que nos enfrentamos ---el ^problema de la
supervivencia^--- sino que accedemos a nuestra representación de él
---el ^problema del sujeto^. Y como cada uno nos lo representamos de un
modo diferente, resulta que cada cual tiene su idea de cómo es la ^vida^
y, consecuentemente, de cómo debe ir resolviéndola.

Algunos creen que la muerte es meramente una traslación de una vida a
otra. Éstos darán por hecha la ^inmortalidad^ individual y no tendrán
reparo en suicidarse, siempre y cuando lo vean necesario y venzan a su
^instinto de supervivencia^.
%
Para otros la vida individual está supeditada a entidades mayores: la
^familia^, la ^tribu^, la ^patria^, la ^iglesia^, la ^raza^, la
^especie^, o la ^vida^ entera. También éstos podrían suicidarse por sus
ideales ---inmolarse--- si la ocasión lo requiriera y vencieran a su
instinto de supervivencia.
%
Hay más posibilidades. Ante un ^dolor^ insoportable, un sujeto puede
resolver que no le compensa padecer tanto. Con mayor razón si el dolor
se debe a una enfermedad incurable y mortal.

He dejado para el final el caso más interesante. Los sujetos, como le
vengo diciendo, somos inquisitivos y, por diseño, buscamos las razones y
las causas de todo cuanto ocurre. Y la razón que nos atañe más
directamente es la que se refiere a nosotros mismos, es el sentido de
nuestra propia existencia: ¿por qué existo? Para un ^materialista^ la
existencia no tiene sentido. Lo digo porque el materialista cree que
todo funciona como un enorme ^mecanismo^, sin que nadie, pero, sobre
todo, sin que él mismo tenga la ^posibilidad^ de variar voluntariamente
el curso de los acontecimientos ni un punto, nada de nada, cero, o sea,
sin ^libertad^ ni ^voluntad^ ni culpabilidades ni responsabilidades.
¿Para qué habría de vivir así, como un ^autómata^? Además, supongo que
el materialista, una vez que empieza a pensar en ello, acabará aceptando
que suicidarse es inevitable, reforzando así sus creencias previas
---todo es inevitable y la ^voluntad^ es una ilusión, porque no hay
libertad. Afortunadamente para los materialistas, también ellos tienen
que vencer a su instinto de supervivencia para suicidarse.

Cualquier ser vivo puede matarse, ya sea por debilitamiento del instinto
de supervivencia, o por un funcionamiento inadecuado de alguno de sus
órganos, o por una predicción equivocada. Incluso puede estar
genéticamente programado tal comportamiento, como en el caso de los
^salmones^, que se matan tras procrear. Pero sólo los ^sujetos^ podemos,
además de matarnos, suicidarnos. Disponer de esta capacidad adicional
nos proporciona una ventaja dudosa, pero de la que puede extraerse una
consecuencia. El suicidio muestra hasta donde puede alcanzar nuestra
libertad.


\Section Las palabras pueden matar

Para dejar finalmente estos macabros asuntos, y evitar de esta manera
tener que volver a tratarlos más adelante, extraeremos aquí las
conclusiones definitivas. El ^suicidio^ muestra que las palabras de un
razonamiento pueden ser letales para el que las produce.

El suicidio es el ejemplo máximo de que las palabras son peligrosas,
incluso para el que las maneja. Sin llegar a tal extremo, los ^insultos^
muestran que las palabras también pueden herir al que las escucha.
También en este caso podemos llegar al extremo. Valen ahora de ejemplo
los casos verídicos en los que el caudillo carismático de una ^secta^
persuade retóricamente a todos sus seguidores para que se suiciden con
él. La palabra también puede matar a terceros, como ocurre en las
guerras santas. En una ^guerra^ santa se mata, y se muere, por razones
estrictamente ^religiosas^. Más concretamente, en una guerra santa es
lícito y justo matar a cualquiera, con la única condición de que a ese
cualquiera se le pueda aplicar el ^adjetivo^ calificativo `infiel'. Que
en todos estos casos haya ^mentiras^ interesadas y embustes arteros no
debilita nuestro caso, sino que lo refuerza: las palabras pueden matar.

Supongo que a usted le parece superfluo explicar que las ^palabras^ no
son inocuas, porque es sabido desde siempre que las palabras pueden
herir. Pero es que esto no va por usted, sino por quienes opinan que es
falso. Creo que la mayoría de la gente es como usted, es decir,
objetivista y no materialista, o, dicho en contrario, no subjetivista y
dualista. ^Objetivista^ porque usted cree que los objetos están y
existen fuera, y ^dualista^ porque usted cree que hay dos substancias
---materia y espíritu--- que se afectan mutuamente. Pero los
materialistas tienen, por una parte, una objeción genuina al dualismo,
y, por la otra, abundantísimos argumentos en favor de su propio caso.


\Section La epífisis

Empecemos examinando la objeción. Si lo espiritual afecta a lo material,
ha de hacerlo en un lugar real preciso y en un momento real determinado,
de manera que el materialista le pide al dualista que le muestre ese
fenómeno de ^interacción^ entre la ^materia^ y el ^espíritu^. De esta
manera el materialista le pasa la carga de la prueba al dualista que,
hasta hoy, no ha podido satisfacer la demanda. El bueno de \[Descartes]
fijó el punto de contacto en la ^glándula pineal^, o ^epífisis^, pero
nunca nadie ha visto nada espiritual allí. Ni tampoco en ningún otro
sitio. Mi opinión es que, con su objeción, el materialista está llevando
la pugna a su propio terreno, y allí sí que gana, pero que no vencería
en terreno neutral.

A estas alturas usted ya sabe que lo espiritual ---llámese teórico, o
sintáctico, como los problemas y la libertad--- simplemente no se puede
percibir. Es decir, nuestros sentidos no pueden darnos noticia del mundo
espiritual. Por lo tanto, el materialista está pidiendo un imposible
cuando nos pide una prueba perceptible del espíritu. Pero esto sólo
prueba que nuestros sentidos están limitados y, en consecuencia, que la
evolución diseña torpemente. Nosotros ya sabíamos que el diseño por
^acreción^ tiene sus inconvenientes, y éste es uno. Nada nuevo, pues.

Pero, si el materialista abandona sus trincheras, y se viene conmigo al
terreno neutral, puede {\em casi} ver los efectos de la ^libertad^. Yo
le sugiero que observe conmigo concretamente el probabilismo esencial de
los fenómenos cuánticos. Ya {\em sólo} tiene que aceptar con nosotros
que la libertad es una de las fuentes de la probabilidad, e interpretar
subjetivamente las probabilidades cuánticas.

Sé que ahora soy yo el que estoy llevando las aguas a mi molino, pero
creo sinceramente que el materialismo no se sostiene. A la postre, si el
materialista no quiere verlo subjetivamente como yo, entonces es su
obligación explicarse a sí mismo las ^paradojas cuánticas^, porque
solamente son paradojas para un ^objetivista^. Y, si es riguroso consigo
mismo, como debe, no le valdrá que se le cuente, siguiendo a \[Bohr],
que la ^física cuántica^ no es racional, porque lo es. La prueba de que
la física cuántica es racional es que ha sido descubierta y está
enunciada con medios exclusivamente racionales. ¿O es que, acaso, la
física cuántica no es un producto de la razón? Lo repito, la física
cuántica es racional, pero no es objetiva, sino subjetiva.


\Section Una función de marionetas

La objeción del materialista es legítima si se considera que la materia
y el espíritu son dos substancias absolutamente incomparables, como
sostiene el dualismo. Porque si una ^substancia^ no tiene nada en común
con la otra, entonces es imposible que puedan entrar en contacto y que
una influya en la otra. Al menos es inimaginable. Por esta razón el
^dualismo^ es lógicamente inconsistente. Es lamentable que usted sea
presa de tan absurda incoherencia, pero tenía que decírselo. Además, no
está usted solo, porque tengo bronca para todos; ahora voy a por los
materialistas.

Ya que no es posible negar la evidencia ---es evidente lo que se ve, lo
obvio, lo perceptible, o sea, la materia---, la manera materialista de
evitar explicar la dolosa interacción entre la materia y el espíritu
consiste en negar el espíritu. Para el ^materialismo^ sólo existe lo que
puede ser medido, que es la ^materia^, y, lo que no se puede medir, ni
existe ni puede influir en lo que existe. La conclusión materialista es
impecable ---si no hay espíritu, no hay interacción que explicar--- pero
sus consecuencias son extrañas: la ^libertad^, la ^voluntad^ y la
consciencia son ilusiones, no hay problemas sino que todo sigue su curso
imperturbable, las palabras no hieren ni animan, y nadie es culpable ni
responsable de nada. Para el materialismo somos unos ^autómatas^ que
vemos una función de ^marionetas^ manejadas por autómatas. ¿No es
inquietante?

Para mi es obvio que yo soy libre de escribir esto, u otra cosa, o
ninguna. Es decir, me parece que los materialistas están negando una
evidencia al no aceptar como verdad que `somos libres'. Sin embargo,
aquí hay un matiz que le conviene apreciar, por si alguna vez tiene que
debatir seriamente con un materialista, así que le prevengo. Para el
materialista la proposición `somos libres' no es verdadera, pero tampoco
es falsa. Esto no es tan extraño como puede parecer a primera vista. Por
ejemplo, la ^paradoja^ `esta frase es falsa' tampoco me parece a mi, y
supongo que a usted tampoco, ni verdadera, ni falsa. Y lo mismo me
ocurre con otras oraciones surrealistas como `la luz, cuanto más larga
es, más salada sabe', que tengo por absurda.

El caso es que los materialistas disponen de un método, ideado por
\[Popper], para distinguir las proposiciones con sentido de las que no
lo tienen. Para ellos, una proposición tiene sentido {\em si y sólo si}
es falsable. Y una proposición es ^falsable^ {\em si y sólo si} existe
un experimento físicamente realizable y un resultado posible de tal
experimento que, de ocurrir, refute la proposición. Es un método
genuinamente materialista, pues solamente deciden las pruebas medibles.

Y, a lo que iba, según este método la proposición `somos libres' no
tiene sentido. No tiene sentido porque no hay ningún experimento físico
que pueda refutar ni la proposición `somos libres' ni su contraria `no
somos libres'. La cuestión es que, aun admitiendo que los resultados de
algunas mediciones fueran impredecibles, como por ejemplo el destino aún
no decidido de sus próximas ^vacaciones^, eso no haría verdadera la
proposición `somos libres' por la sencilla razón de que cualquier dato
impredecible puede atribuirse al ^azar^ o al ^desconocimiento^ antes que
a la ^libertad^. Resulta entonces que ningún resultado de un experimento
físico permite discriminar si lo que ocurre es que `somos libres' o,
simplemente, que `somos ignorantes'. En mi opinión somos libres e
ignorantes, pero la conclusión materialista es que la proposición `somos
libres' no se puede falsar y, por consiguiente, no tiene sentido.


\Section Ni dos ni una, ninguna

Mi conclusión es que el ^objetivismo^ no combina bien ni con el
^materialismo^ ni con el ^dualismo^. A estas alturas no creo que vaya a
sorprenderle mi solución del problema: el subjetivismo. El
^subjetivismo^ termina a la baja con la discusión sobre el número de las
substancias. No hay ni dos, como nos juran ustedes los dualistas, ni
una, como nos garantizan los materialistas, porque hay cero substancias.
No hay ninguna ^substancia^, lo que hay es ^información^.
$$\hbox{Substancias}
  \llave{\hbox{Objetivismo}%
         \llave{\hbox{Dualismo $\cdot$ 2}\cr
                \hbox{Materialismo $\cdot$ 1}}\cr
         \noalign{\vskip2\jot}%
         \hbox{Subjetivismo $\cdot$ 0}}$$

La ^información^ unifica la ^materia^ y el ^espíritu^. La materia y el
espíritu son ambos información. Así la ^interacción^ entre la materia y
el espíritu no es imposible, sino inevitable. Y se produce como nos
enseñó \[Turing] ---ni \[Gödel] ni \[Church], \[Turing].


\Section Viva la diferencia

Los ^lenguajes simbólicos^ son más expresivos que los semánticos y, sin
embargo, no se puede distinguir por el comportamiento a un individuo
simbólico de otro que no lo es. Recuerde que cualquier posible jugada de
^ajedrez^ puede ser expresada en un ^lenguaje semántico^, aunque tal
lenguaje no permita expresar problemas ajedrecísticos. Por eso, no es
posible discernir si detrás de una jugada de ajedrez hay un jugador
simbólico o uno que no lo es. No hay ningún acto de un ^sujeto^ libre
que sea perceptiblemente diferente de un acto que podría realizar un
autómata lo suficientemente complejo.

Hay diferencias, sin embargo. Una diferencia es la ^consciencia^. Pero
la consciencia no se puede medir, ni percibir, aunque sí somos
conscientes de nuestra consciencia. Otra diferencia es la ^libertad^,
pero tampoco se puede percibir, aunque seamos plenamente conscientes de
ser libres y de actuar en libertad, ¿o no? El ^suicidio^ es otra
diferencia más. Pero sin ^empatía^ es imposible distinguir
matarse de suicidarse, y la propia empatía no se ve con los ojos.

La conclusión es sencilla: existen cosas que no son perceptibles pero
que afectan. No pueden medirse, es decir, son etimológica y literalmente
^inmensas^, y, no obstante, influyen en las medidas. Es el mundo
sintáctico, la ^teoría^. Así como hay sentidos que nos dejan ver el
mundo perceptible real, hace falta un ^motor sintáctico^ para ver el
mundo teórico espiritual. Fue \[Turing], sobre los hombros de \[Gödel],
quien nos mostró como construir un motor sintáctico, es decir, quien
demostró que la sintaxis puede tener efectos medibles. De este modo,
\[Turing] derribó el postulado ^materialista^. Posiblemente sin
quererlo, pero esa es otra historia. En cualquier caso, no es en
la epífisis de \[Descartes], sino en el motor sintáctico de \[Turing],
en donde la materia y el espíritu interactúan.

Se lo explicaré de otra manera. Tanto la ^realidad^ como la ^teoría^ es
^información^ cuyo origen es el ^problema de la supervivencia^. La
distinción entre la información real y la información teórica es
producto de la historia evolutiva, por lo que puede ser meramente
circunstancial. Porque, aunque la ^teoría del problema^ establece una
diferencia entre representar soluciones y representar resoluciones que
explica la distinción entre la realidad y la teoría, es posible que la
propia teoría del problema tenga la forma que tiene porque es un
producto, también, de la ^evolución^ darviniana.

Sea circunstancial o bien fundada, para nosotros la distinción entre la
realidad y la teoría está escrita en nuestros ^genes^, por lo que no
podemos evitarla. Aunque ambos sean información, podemos percibir una
^piedra^ pero no podemos percibir la ^libertad^. Podemos, eso sí, ^ver^
la libertad introspectivamente en nosotros mismos y empáticamente en los
demás.


\Section El relativismo

Mi manera de unificar la materia y el espíritu consiste en reducirlos
ambos a información. Pero, como vimos casi al principio del libro, la
^información^ mide la ^incertidumbre^ despejada, de manera que la
información depende del ^conocimiento^ previo del ^sujeto^ que recibe el
^dato^ y del uso que le dé. Recuerde que, como usted ya sabía que estaba
^lloviendo^, oir la oración `está lloviendo' no le aportó ninguna
información meteorológica, cero bitios, o, mejor, prácticamente cero,
mientras que para mi, ignorante del estado atmosférico, significó un
^bitio^ de información. Y acuérdese también de que, aunque no
meteorológica, a usted el mensaje sí que le proporcionó información
sobre la veracidad de la mensajera. Veo que este ^subjetivismo^ le
inquieta.

Desgraciadamente, no puedo complacerle: mi teoría es una teoría
subjetivista del sujeto. Pero, tal vez, le consuele saber que,
contrariamente al subjetivismo clásico de los ^sofistas^, mi
subjetivismo no es relativista. Cuando el sofista \[Protágoras]
declaraba perspicazmente que ``el hombre es la medida de todas las
cosas'', \[Sócrates] se escandalizaba pensando en las consecuencias
éticas de tal proclama. Según los sofistas, la ^verdad^, el bien y la
virtud eran asuntos de cada uno, de modo que la ^justicia^, y el bien
común sobre el que se sostiene, quedaban inhabilitados. \[Sócrates] se
dejó ajusticiar para mostrar cuanto le repugnaba el ^relativismo^
---`todo es relativo'--- de los sofistas.

\[Platón], que era un joven discípulo que asistió a su viejo maestro
\[Sócrates] cuando éste bebió la cicuta, quedó deslumbrado por tan bello
y convincente gesto. Tanto que buscó absolutos mucho más allá de lo
prudente. En mi opinión fue un error que, además, tuvo repercusiones muy
duraderas. Por eso le dije, aun más al principio del ^libro^, que no
debemos tomar una postura epistemológica, esto es, relativa al origen
del conocimiento, por razones éticas. De hacerlo repetiríamos la
equivocación de \[Sócrates] y \[Platón].

Si construyéramos la ^epistemología^ sobre la base de la ^ética^,
entonces todo el ^conocimiento^ quedaría viciado desde el principio. Es
decir, no podríamos saber si nuestro conocimiento era verdadero, o
simplemente era bueno, como consecuencia de nuestra ética. Y, en el caso
del conocimiento ético, la situación sería paradójica como resultado de
un ^círculo vicioso^. ¿Cómo buscaríamos las verdades éticas, si
hubiéramos basado la verdad en la ética? No podríamos, y la ética no
sería racional.


\Section El sermón

De todos modos, si yo mantengo tan ufano mi opinión sobre la
preeminencia del conocimiento verdadero sobre la buena acción es porque
la ^ética^ que resulta de mi ^epistemología^ no es relativista. De ser
de otro modo, seguramente también yo intentaría cualquier pirueta
racional, por incoherente que fuera, que casara mi epistemología con mi
ética. Por cierto, es muy divertido observar las acrobacias que tienen
que realizar por este motivo los más sinceros y consecuentes de los
materialistas, como \[Dennett].

Mi ^subjetivismo^ no es relativista porque, para mi, la ^vida^ es
absoluta. Los sujetos individuales solamente somos una parte minúscula
de la vida; usted y yo también. En concreto, nuestra especie
\latin{^homo sapiens^}, y con ella todo el género \latin{homo}, podría
extinguirse y la vida proseguiría. No somos imprescindibles; usted y yo,
menos.

Sin embargo, los sujetos somos muy peculiares. Y nosotros somos aun más
particulares porque nuestra especie es la única que alcanza el nivel de
sujeto; somos una especie singular. La ^libertad^ es la peculiaridad del
^sujeto^ con más consecuencias éticas. Como sujetos vivos,
interiorizamos la libertad del ^problema de la supervivencia^ y la
gestionamos en la multitud de subproblemas que resultan de su
resolución. Para mi, la ^vida^ es un problema genuino y absoluto ---el
problema de la supervivencia es {\em el} problema--- y su libertad es
consecuentemente genuina y limitada. ¿Limitada? Sí, limitada por la
^condición^ del problema, que en este caso es desconocida porque el
problema de la supervivencia es un problema aparente. No se confunda,
todo nuestro ^conocimiento^ es la expresión de la condición del problema
del sujeto, que no es el problema de la supervivencia sino su
interiorización.

Con esto quiero decir, sencillamente, que los sujetos somos libres. El
^futuro^ no está escrito, sino que es abierto. Podemos, y si no
podremos, acabar con la ^vida^ toda, pero también podemos, y si no
podremos, extenderla por todo el universo. No me gusta sermonear, así
que me he limitado a reseñar los dos datos de la cuestión ^ética^: que
la ^vida^ es absoluta ---nosotros no--- y que nosotros somos libres
tanto para beneficiar como para perjudicar a la vida. Le doy una pista,
aunque ya sé que no la necesita: si dañamos a la vida, nos herimos a
nosotros mismos.


\Section Non plus ultra

Si la ^vida^ es absoluta, se preguntará usted cómo es que podemos, o
podremos, terminar con ella. Simétricamente querrá saber cómo pudo
aparecer la vida de la no vida.

Me ha pillado. La verdad es que no sé si seremos capaces de terminar con
toda la vida, o no. Y, siendo la ^evolución^ darviniana un ^conocedor^
combinatorio, es imaginable que antes hubiera otros procesos más
sencillos, pero sobre el origen mismo de la vida nada cierto puedo
añadir. Es decir, podría contarle mis fantasías, pero sería confundirle,
así que cada cual con las suyas.

Siento defraudarle, pero mi teoría no lo explica todo. La vida es
absoluta y, en consecuencia, el ^conocimiento^ es relativo. Toda la
^información^ que captamos se refiere a la resolución del ^problema de
la supervivencia^, y el conocimiento, que es la información acumulada,
ordenada y _comprimida<comprimir>, es la mejor herramienta que tenemos
para resolverlo. El conocimiento es una herramienta de supervivencia.

La ^ley de la información incesante^ es otra manera de decir que el
^problema de la supervivencia^ es aparente, de manera que no es posible
cerrarlo definitivamente, y que la ^vida^, la ^libertad^, la
^información^ y el ^tiempo^ están ligados inextricablemente. Ahora viene
la prueba de si está usted adquiriendo una sensibilidad subjetiva o no.
Dígame la verdad, sí o no, ¿a que a usted le parece harto improbable que
antes de la vida no hubiera tiempo? Si le parece, no improbable, sino
aberrante la mera consideración de semejante posibilidad, entonces es
que no he conseguido atraerle al ^subjetivismo^. La cuestión es que el
tiempo, tal como nosotros lo entendemos, es un concepto que tiene el
propósito final de facilitar nuestra supervivencia. Preguntarse si el
tiempo, purificado y desembarazado de ese propósito utilitario, podría
mantener o no alguno de sus rasgos es pura conjetura.

Reconozco que es interesante hacerse preguntas, e incluso inevitable, ya
que somos inquisitivos por diseño. Por eso es conveniente saber hasta
dónde es razonable contestarlas y no ir más alla ---en latín es el
\latin{non plus ultra} de las ^columnas de Hércules^; otra vez Hércules y
sus columnas. Se lo repito, explicamos para vivir, pero vivir no tiene
explicación.


\Section La materia

Estoy refutando las dos variedades del ^objetivismo^: el ^dualismo^ y el
^materialismo^. El dualismo fracasa porque es incapaz de ligar el
^espíritu^ con la ^materia^. El materialismo falla porque, al despreciar
la ^libertad^ y la ^consciencia^, es más restrictivo de lo necesario. El
^subjetivismo^ tampoco es completo ---no explica la ^vida^, como
acabamos de ver--- pero abarca más que el materialismo. Esto es lo que
quiero demostrarle ahora.

Para el materialismo lo absoluto es la ^materia^. Esto significa que en
el materialismo objetivista lo más básico, lo primero, es la
^ontología^, que trata de los objetos materiales. Estos objetos, que
existen absolutamente, generan datos en todo instante, el ^sujeto^ capta
parte de esos datos y de esta manera alcanza el ^conocimiento^ de los
objetos. Según la ^epistemología^ objetivista, los sujetos intentan
reproducir fielmente en su interior los objetos que existen fuera, y la
^verdad^ es la igualdad entre la reproducción interior y el ^objeto^
exterior.

El ^positivismo^, que viene a ser el desarrollo científico del credo
materialista y objetivista, ensancha el concepto de ^materia^, que
merced a la ^teoría de la relatividad^ de \[Einstein] se iguala a la
^energía^, y postula que ha de haber una cadena causal de fenómenos
físicos que enlace el objeto externo con la representación interna.
Todos los fenómenos de la cadena son físicos, de manera que son todos
ellos medibles y, por consiguiente, asequibles a la ciencia positiva.

Atento ahora, porque voy a relacionar el conocimiento positivo, esto es,
el conocimiento obtenido por la ciencia objetivista y materialista, con
mi punto de vista subjetivo. Desde la interpretación positiva, lo que yo
llamo ^objeto^ es la representación mental del objeto material, y la
^realidad^ subjetivista es la representación mental de la realidad
exterior. Como, incluso desde la interpretación objetivista, hay que
reconocer que nuestro pensamiento sólo tiene acceso a las
representaciones mentales, resulta que el objeto material y la realidad
exterior son referencias indirectas. Esto significa que, todavía desde
la interpretación positivista, la ^ciencia^ positiva se refiere
indirectamente a la realidad exterior objetivista, y directamente a la
realidad subjetivista.

La conclusión del razonamiento anterior es muy importante porque
satisface un requisito que nos habíamos impuesto al comenzar esta
nuestra tarea filosófica, y porque transfiere a nuestro ^subjetivismo^
cada uno de los abundantísimos argumentos que el ^positivismo^ tiene a
su favor. La conclusión es que en el subjetivismo cabe toda la ^ciencia^
positiva, incluso mejor que en el propio positivismo. Cabe toda y mejor.

Cambia, eso sí, el punto de vista. Las ^leyes^ de la naturaleza no
describen los cambios que acaecen en el ^exterior^, ni sirven para
predecir el futuro estado de los objetos externos, como dice el
^positivismo^. Para el ^subjetivismo^, las leyes de la naturaleza
describen como varía nuestro conocimiento sobre el exterior, y nos
permiten predecir las ^mediciones^ futuras.

Aunque no haya cosas reales ahí fuera, la ciencia positiva sirve para
explicar nuestra ^percepción^ del ^entorno^ y adelantar sus reacciones.
La ciencia positiva es útil porque amplifica, refina y explica lo
percibido, aunque sólo lo percibido.


\Section La metaparadoja

El ^materialismo^ es consistente, aunque con reparos; veámoslo con más
detalle. El ^sujeto^ materialista es incapaz de alcanzar todos los datos
generados por todos los objetos del universo, de modo que desconoce
necesariamente la mayor parte de lo que sucede. La ^ignorancia^ del
sujeto materialista es esencial, en este caso por el necesario
^desconocimiento^ del sujeto, que tiene una capacidad limitada. Tampoco
niega el materialismo la aleatoriedad, a la que recurre cuando necesita
explicar las probabilidades esenciales de la física cuántica.

Las fuentes de la ^probabilidad^ sirven para distinguir la posición
epistemológica de cada cual. Descontando el ^desconocimiento^, que todos
tenemos que aceptar humildemente, las diferencias las hace el ^azar^ y
la ^libertad^. Ustedes los ^dualistas^ aceptan ambas: azar y libertad.
Ellos los ^materialistas^ aceptan el azar pero rechazan la libertad.
Justo lo contrario que nosotros los ^subjetivistas^, que rechazamos el
azar y aceptamos la libertad. Hay, por último, un grupo de materialistas
radicales, denominados ^deterministas^, que únicamente aceptan el
desconocimiento como fuente de indeterminación.
$$\vbox{\halign{\strut
  \hfil#\quad&\vrule\quad\hfil#\hfil&\quad\hfil#\hfil&
              \quad\hfil#\hfil\quad\vrule\crcr
  Probabilidad& Azar& Ignorancia& Libertad\cr
  \noalign{\hrule}
  Dualismo&\sc     sí&\sc sí&\sc sí\cr
  Materialismo&\sc sí&\sc sí&\sc no\cr
  Subjetivismo&\sc no&\sc sí&\sc sí\cr
  Determinismo&\sc no&\sc sí&\sc no\cr
  \noalign{\hrule}}}$$

Los contenidos de la teoría materialista son consistentes, ya que no
existe contradicción entre ellos. Pero el ^materialismo^ es defectuoso
porque, para asegurarse su propia consistencia, se limita en exceso. Le
sonará ya repetido que el materialismo excluye todo lo espiritual y
teórico. Se lo repito porque, en este punto, acontece una meta-paradoja
que se refiere a la propia ^teoría^ materialista. Ocurre que el
materialismo, al rechazar las teorías, compromete la situación de la
propia teoría materialista, que queda en precario.


\Section La verdad

Y un fallo de la teoría objetivista atañe a la verdad de sus enunciados.
Recuerde que, según la doctrina objetivista, la ^verdad^ es la igualdad
del ^objeto^ material exterior con su reproducción interna. Es decir, si
el objeto interior es igual al objeto exterior, entonces el conocimiento
representado por el objeto interior es verdadero. La cuestión es cómo
podría verificarse que la igualdad se satisface. Para comprobar la
igualdad habría que acceder a ambos términos, al objeto interno y al
objeto externo, y entonces compararlos. Pero esto es imposible, porque
el objeto externo es el \latin{^Noumenon^} ignoto e inaccesible de
\[Kant]. Resumo: la verdad objetivista es inverificable, porque es una
comparación y uno de los términos es inalcanzable. ¿De qué vale una
verdad inverificable?

La ^verdad^ de \[Tarski] es una operación de ^desentrecomillado^
$$\hbox{`La nieve es blanca' es verdad \latin{si y sólo si}
         la nieve es blanca}$$
que recuerda a los lógicos que la verdad de una oración depende de si
ésta se corresponde o no con la ^realidad^. En general, se supone que la
realidad es la realidad material exterior, y entonces es meramente una
definición técnica de la verdad materialista.

Pero, si se interpreta subjetivamente la realidad, entonces me vale. Lo
que quiero decir es que, si se entiende que la ^realidad^ es el modelo
del ^exterior^ que nos presenta nuestro aparato perceptivo, o sea, según
nuestra terminología, el mundo exclusivamente semántico, entonces la
verdad de una expresión sintáctica consiste en que describa fielmente la
representación semántica. La ^verdad^ es la adecuación fiel de la
^sintaxis^ a la ^semántica^.

Quiero que vea ahora algunas características de mi versión de la verdad,
por si decidiera usted que le interesa adherirse a ella. La verdad no
es absoluta, porque depende del ^sujeto^. Pero, sin embargo, no depende
de la ^voluntad^, porque la ^percepción^ es un mecanismo que viene 
especificado en los ^genes^ y sobre el que no tenemos control voluntario
alguno. Vemos blanca la nieve, sin elección. Así que `la nieve es
blanca' es una verdad firme, y que puede verificarse; basta mirar. Un
punto para mi.

Mi verdad puede generalizarse para admitir como verdadera la oración
`los unicornios no existen'. Para conseguirlo hemos de sustituir la
representación exclusivamente semántica, esto es, puramente perceptiva,
por el conjunto completo de mis creencias, que es mi mundo entero, y que
incluye, tanto la ^realidad^ que percibo, como las ^teorías^ que creo
que son verdaderas. Esta verdad es más general pero también más débil
que la anterior, ya que algunos afirmarían que la oración
`somos libres' es verdadera, y otros que no lo es. Lo que pasa es que,
de hecho, es exactamente así como sucede, de manera que mi definición
de verdad se ajusta a su uso. Otro punto para mi ---gano dos a cero---
y yo mismo pito el final del partido.


\Section El subjetivismo

Ahora que ya me sé ganador, le contaré un secreto. Para algunos mi
^subjetivismo^ es sólo medio subjetivismo porque reconozco que hay algo
extra-subjetivo en la ^piedra^ que veo. La piedra está en parte fuera de
mi y en parte dentro de mi. La piedra no está fuera, aunque haya algo
fuera y otro algo dentro que hace que usted y yo veamos la misma piedra.
No habría piedra sin ^sujeto^, ya que lo extra-subjetivo, aunque es
necesario, no es suficiente. Y, además, de las dos partes de la piedra,
la única que conozco es la subjetiva. Por esto, aparte de reconocer que
la piedra tiene un ingrediente extra-subjetivo, poco más puede decirse
de tal componente. Porque, repito, todo lo que el sujeto sabe de la
piedra es subjetivo.

Un error frecuente entre quienes intentan acercarse al subjetivismo es
creer que existe una relación de uno a uno entre los objetos y los
\latin{^Noumena^}. No, esto sería objetivizar el exterior, parcelándolo
y segmentándolo, y la partición es un artefacto perceptivo que comprime
y altera lo que percibimos. Quizás nuestra mejor aproximación consista
en suponer que el ^exterior^ es un todo único interrelacionado, pero,
aun así, es un apaño que solamente nos sirve como sucedáneo.

No es verdad, entonces, que los subjetivistas sostengamos que no
existiría nada exterior si no hubiera sujetos. Quien tal cree iguala
^exterior^ a ^realidad^, de modo que cuando oye a un subjetivista
afirmar que `la realidad es subjetiva' interpreta que está diciendo que
`el exterior es subjetivo'. Pero, el exterior no es subjetivo, sino que,
muy al contrario, lo exterior es lo que no es sujeto.

Yo, que me tengo por un ^subjetivista^, y no por medio, sí que creo que
hay un ^exterior^ que es independiente de los sujetos, pero, como mucho,
puedo conceder que es como un todo. Lo que intento hacerle comprender es
que conviene ser muy precavido al hablar sobre como es el exterior.
Porque yo estoy convencido de que nuestro modelo del exterior, que es lo
que habitualmente se conoce como ^realidad^, introduce muchos
artefactos, aunque nos sea muy útil para esquivar las ^piedras^ que nos
lanzan. La distorsión más notable, porque es la primera y más
desvirtuadora, es la objetivación de lo exterior; las líneas son
artefactos de la percepción. Por eso, la parcelación del exterior en
objetos es una _simplificación<comprimir>, seguramente necesaria para
reducir su complejidad, pero que es seguro que lo altera.


\Section Tres citas

Que el ^exterior^ sea independiente de los sujetos no implica, pues, que
``el ^sol^ y los planetas y las ^montañas^ de la tierra'' sean
independientes de los sujetos. El sol es parte de la ^realidad^ y, como
la ^piedra^, tiene un componente exterior, necesario, pero insuficiente.
Así de fácil se explica que ``el sol y los planetas y las montañas de la
tierra'' dependan del ^sujeto^ que los percibe. La cita es de \[Stroud],
concretamente de un pasaje en el que confiesa no entender a \[Kant]; yo
se lo estoy explicando.
% \(The Quest for Reality), by \[Stroud], page 196.
% ``It is not easy to accept, or even to understand,
% this philosophical theory [\[Kant]'s \(Critique of Pure Reason)].
% Accepting it presumably means believing that
% the sun and the planets and the mountains on earth
% and everything else that has been here so much longer than we have
% are nonetheless in some way or other dependent on
% the possibility of human thought and experience.''

Otra cita que viene al caso se debe a \[Einstein]. Debe de ser
maravilloso componer una ^teoría^ matemática abstrusa, aunque basada en
sólidos razonamientos físicos, como la ^teoría de la relatividad^, y
comprobar que sus predicciones se cumplen puntualmente. Por eso se
entiende que \[Einstein] se inspirase en \[Kant] para escribir que ``el
hecho de que el mundo de nuestras experiencias sensibles sea
comprensible es un milagro''. Pero no es un milagro, sino la explicación
de cómo ^percibimos^ el mundo. La realidad es la sensación ordenada y
comprimida, o sea, comprendida. ^Comprimir^ y ^comprender^ son más
sinónimos de lo que creen algunos.
% ``The fact that it [the world of our sense experiences]
%   is comprehensible is a miracle'',
% \[Einstein], \(Phisics and Reality), in \(Ideas and Opinions), page 292.

La tercera cita a propósito de \[Kant] y el ^subjetivismo^ aborda la
misma idea, pero por su flanco sintáctico. Es del mejor \[Wittgenstein],
y dice que
 $$\hbox{``nada ilógico {\em puede} ser pensado''.}$$
 % \(Tractatus Logico-Philosophicus) \S 5.4731
La explicación subjetiva de esta sentencia ya debería serle fácil. La
^lógica^ no es algo extra-subjetivo, sino una manera ordenada y
comprimida de expresar el funcionamiento de la ^razón^, que es un ^motor
sintáctico^. Como todos los pensamientos del ^sujeto^, sin excepciones,
son productos de su razón, resulta que cualquier cosa que pueda pensar
un sujeto se ajusta necesariamente a la estructura impuesta por el motor
sintáctico que la generó. Para mi, entonces, esto no es más que una
^tautología^ que, a modo de parodia, queda:
 $$\hbox{`nada inimaginable {\em puede} ser ^imaginado^'.}$$


\Section La sima

Empezamos este ^libro^ buscando ^grietas^ en el templo del ^saber^ y
hemos descubierto que el edificio está partido por la mitad, así que
hay, a día de hoy, dos cuerpos de conocimiento completamente separados:
^ciencias^ (números) y ^artes^ (letras). Buscábamos fisuras y
encontramos una sima. La situación es precaria y urge construir un nuevo
edificio para reunir en él, de nuevo, la ^materia^ con el ^espíritu^, la
^realidad^ con la ^teoría^, y lo que se ^ve^ con lo que se ^dice^.

Buscando las grietas nos hemos percatado de su origen. Al congelarse, el
agua aumenta de volumen, de manera que si primero se infiltra líquida y
después se hiela, actúa como una cuña y puede causar el agrietamiento de
las rocas más duras. No se me asuste; el agua no es la causa del
resquebrajamiento del saber, y simplemente nos servirá como inspiración.
Al investigar los vicios del ^objetivismo^, observamos que su pecado
consiste en cristalizar como sustancia los ^objetos^, que no son más, ni
menos, que fluida ^información^.

El ^materialismo^ encuentra las dos fracciones del edificio y toma
partido: una parte le vale y la otra no. Postula que sólo vale la parte
mensurable; el materialismo niega la ^inmensidad^. El materialismo
afirma que el ^espíritu^ no puede influir en la ^materia^, porque supone
que son de tan distinta naturaleza, uno ficticio y la otra real, que es
insensato intentar cualquier aproximación entre ellos.

El razonamiento materialista se desmorona en cuanto se demuestra que
ambos, ^teoría^ y ^realidad^, son ^información^. Nuestra propuesta es
que el origen de toda la ^información^ es el ^problema aparente de la
supervivencia^. La ^vida^ explica entonces la materia y el espíritu, la
realidad y la teoría, y lo que se ve y lo que se dice.


\Section Todo son datos

Después de haberse leído todas mis alocadas ideas sobre los asuntos
más inverosímiles, se merece usted un resumen. ¡Qué menos!
Lamentablemente, que usted se lo merezca, no me dota a mi de recursos
extraordinarios, de modo que tendré que valerme de mis capacidades
habituales, que, por desgracia, no aseguran el éxito de la empresa. Eso
sí, le prometo intentarlo.

Si quiere usted quedarse con un único concepto, uno solo, éste es el de
`^subjetivismo^'. No lo hubiera creído, si no fuera porque acabo de
hacerlo: he resumido todo el ^libro^ en una única palabra. A lo que iba,
el subjetivismo consiste en distinguir la ^realidad^ del ^exterior^. Y,
aunque parece fácil, no debe de serlo, porque casi nadie me entiende.
Hay algo fuera, pero no es la realidad. Lo que hay dentro y fuera es
^información^, o, más precisamente, datos: todo son datos.

La información contenida en un ^dato^ depende de cuanto contradiga dicho
dato el ^conocimiento^ vigente del ^sujeto^ que lo recibe. Si el dato se
ajustara exactamente a lo que el sujeto preveía, y además lo tuviera por
absolutamente ^cierto^ y completamente indudable, entonces la cantidad
de información que contendría el dato sería nula, cero, porque el sujeto
no tendría que añadir ningún conocimiento adicional al que ya tenía. Lo
que pasa es que esto es imposible, porque lo único que es absolutamente
^indubitable^ para un sujeto es su propio pensar. Por esto, según la
^ley de la información incesante^, un dato siempre aporta información al
sujeto que lo recibe.

La ley de la información incesante le asegura al sujeto una lluvia
pertinaz de información que lo inundará si no pone algún remedio. Si no
^comprime^ la información, ésta le desbordará. El ^conocimiento^ es la
información acumulada, ordenada y comprimida. La ^percepción^ es el
primero de los procesos de compresión que toma información y produce
conocimiento. El conocimiento perceptible es la realidad. Los sujetos
disponemos, además, de un ^motor sintáctico^, así que también producimos
conocimiento teórico. Los sujetos disponemos de un ^doble compresor^.

La distinción entre la ^información perceptiva^ y la ^información
teórica^ es seguramente contingente, ya que se debe a una razón de
diseño evolutivo, y la ^evolución^ darviniana es oportunista. Entonces,
aunque nosotros no podamos, por diseño y construcción, percibir los
conceptos teóricos, esto no implica que la ^realidad^ sea mejor que la
^teoría^. Y mucho menos, como proponen los materialistas, que haya de
rechazarse todo lo espiritual. Porque, si yo estoy en lo cierto, para
entender lo que es un ^sujeto^, la ^realidad^ es insuficiente ---no
basta percibir--- y es necesaria la ^teoría^ ---hay que pensar. Y
nosotros somos sujetos vivos; usted y yo incluidos.

Recuerde mi propuesta. La ^vida^ es {\em el} ^problema^, y cada ser vivo
es un resolutor del ^problema de la supervivencia^. Solamente los
^sujetos^, que disponemos de un ^motor sintáctico^, podemos
representarnos problemas y resoluciones, y por esto sólo los sujetos
somos conscientes de que el mundo es enigmático y nosotros resolutivos.
Nuestra naturaleza es resolutiva, y la del mundo es problemática. Así
que el mundo no es una máquina enorme, como creen los materialistas,
sino un ^enigma^ inmenso.


\Section Amén

Estamos terminando, y antiguamente ningún libro de ^filosofía^ podía
omitir una prueba, al menos, de la existencia de ^Dios^. \[Descartes],
por ejemplo, se aprovechó de una ^aporía^ para fundar la suya. Observó
atinadamente que es lógicamente imposible que algo ^finito^ comprenda
algo infinito, porque no cabe. Y, sin embargo, nosotros somos finitos y
comprendemos el infinito. Para \[Descartes] la resolución del enigma es
fácil: Dios es todopoderoso e infinito y quiere que nosotros lo
conozcamos. Pero a mi, desgraciadamente, no me vale su prueba, porque
prefiero una explicación más terrena: comprender el ^infinito^ es
entender que la serie 1,~2, 3, 4,~$\ldots$ no finaliza, ¡sin tener que
seguirla hasta el final! O es captar que `esta frase es falsa' es una
^paradoja^, en vez de quedarse para siempre atrapado en un círculo
vicioso. Así que, rechazada la del maestro, se impone que le presente mi
prueba irrefutable de la existencia de Dios.

Sólo hay dos posibilidades: que usted lo sepa todo, o que usted no lo
sepa todo. Estudiemos cada uno de los casos, empezando por el fácil. Si
usted lo sabe todo, entonces usted es Dios y, en este caso, queda
demostrada la existencia de Dios. Nos toca ahora examinar el otro caso.
Si usted no lo sabe todo, entonces eso que usted no sabe es Dios.
Sospecho que usted podría replicar que eso que usted no sabe podría ser
cualquier otra cosa. Pero no debe hacerlo, porque, y este es el meollo
de la demostración, lo que usted no sabe no puede servirle para
argumentar cosa alguna, porque su argumento estaría hueco, vacío y sería
absurdo. Y, como usted no puede refutarla, mi prueba es irrefutable.

Lo curioso de esta demostración es que le resulta más útil a un ^ateo^
que a un creyente. Porque un creyente necesita ^información^ positiva
sobre Dios, así que su Dios no puede ser inefable.

Si sólo lee lo que está escrito, le parecerá que soy un ^blasfemo^.
Aunque seguramente podría haberlo expresado de un modo menos crudo, o
sea, más cocinado, si relaciona la prueba con el resto de la ^teoría de
la información^, se percatará de que es una consecuencia lógica de ella.
Porque todo nuestro conocimiento tiene, en último término, un valor
utilitario; el ^conocimiento^ es una herramienta de supervivencia. De
esta observación sobre la naturaleza utilitaria de todo nuestro
conocimiento se sigue que, si Dios es algo más que una herramienta de
^autoayuda^, entonces no puede estar en lo que conocemos, y, por
consiguiente, sólo puede estar en lo que no conocemos.

Dése cuenta de que si yo niego que haya ^piedras^ fuera, si le digo que
no podemos saber qué hay fuera, mucho menos le voy a poder decir qué
quiere Dios, o cómo es Dios, o incluso si hay Dios o no. A cambio, me
voy a permitir darle un ^consejo^: no crea a quien le diga que sabe lo
que quiere ^Dios^. Amén.


\Section Quod erat demonstrandum

Al principio del ^libro^ le prometí que, al finalizarlo, sería usted
^filósofo^, y ya estamos en la última sección. Así que llega el momento
de comprobar si usted es, o no es, filósofo.

A lo largo del libro he ido escribiendo varias definiciones de filósofo.
Son mis interpretaciones personales, y no sería justo que las utilizase
para esta prueba definitiva, que debería ser imparcial. Así que, para
dar el veredicto final, usaré la definición etimológica.

La palabra `^filosofía^' viene, como no, del ^griego^. Conjunta dos
ideas: `filo', que es antónima de `fobia', y que significa `amor' o
`querencia'; y `sofía', que es `saber' o `^conocimiento^'. Así que un
filósofo es quien ama la sabiduría, es quien quiere saber, o sea, que un
filósofo es, ni más ni menos, que una persona curiosa.
$$\hbox{Filósofo} = \hbox{Persona curiosa}$$

Una `persona curiosa' es tanto una persona que tiene ^curiosidad^, como
una persona que despierta la curiosidad. También puede entenderse que es
una persona aseada, pero esta tercera acepción sí que no nos cuadra a
los filósofos.

De estas definiciones se sigue que hemos de distinguir al ^sabio^, que
es quien sabe, del ^filósofo^, que es quien quiere saber. Podemos
admitir que hay una relación entre saber y querer saber, pero es claro
que no son lo mismo. Es decir, un sabio ha de ser filósofo, pero un
filósofo no tiene que ser sabio.

De modo que, si ha tenido usted la curiosidad suficiente para haber
leído hasta aquí, es que es usted una persona curiosa. Y en conclusión,
y como queríamos demostrar, ¡usted es un filósofo!


\endinput
