% edc5.tex (RNMCG20040924)


\Section El aprendiz

Los tanteos tienen un defecto, a saber, que el error es una parte
esencial del ^tanteo^, y por esto los tanteos son también conocidos como
métodos de ^prueba^ y ^error^. Si hacemos un tanteo con papel y lápiz
para resolver un problema aritmético, la situación puede no ser grave,
pero tratándose del ^problema de la supervivencia^ la cuestión es,
literalmente, de ^vida^ o muerte. Un modo de evitar el error consiste en
prever el resultado de la acción antes de ejecutarla. Para prever lo que
ocurrirá, se necesita un ^modelo^ sobre el que se pueda simular la
acción antes de ejecutarla actualmente. Si la modelación es adecuada, la
simulación servirá de ^pronóstico^. La simulación con modelo es,
obviamente, una resolución por ^analogía^, o ^traslación^.

Si los diseños evolutivos han ido creciendo en complejidad, como nos
dice la ^ley de la información creciente^, el siguiente paso resolutivo
fue el ^aprendiz^, que implementa la traslación. El aprendiz tiene tres
partes: un ^cuerpo^, como el del adaptador, capaz de varios
comportamientos; un ^modelador^ que construye dentro del aprendiz un
modelo que imita el comportamiento del exterior; y un ^simulador^ que
emplea el modelo para hacer pronósticos antes de elegir uno de los
comportamientos propios.
$$\hbox{Aprendiz}
    \llave{Gobernador \llave{Modelador\cr Simulador}\cr
           Cuerpo\strut}$$

Entonces, para diseñar un aprendiz, han de determinarse con precisión
sus tres partes: el ^cuerpo^ con sus posibles comportamientos, la manera
concreta de modelar el comportamiento del exterior, y el método para
resolver el problema interiorizado, que podría ser por tanteo. Queda
indeterminado el ^modelo^ y, en doble grado, el comportamiento actual
del ^aprendiz^, que depende tanto de de las circunstancias exteriores,
como del modelo interior.

Los ingenieros suelen diseñar ^mecanismos^. De hecho, `ingenio' ha
venido a significar `máquina o artificio mecánico'. Sólo en algunos
casos diseñan ^adaptadores^, es decir, aparatos que resuelven por
^tanteo^, como los termostatos. Un ^termostato^ típico es capaz de dos
comportamientos, calefactar o no calefactar, y tiene unos circuitos de
control que eligen el comportamiento actual, ya no en la fase de
^diseño^, sino sobre la marcha, en la fase de funcionamiento. Y casi
nunca diseñan aprendices. ¿Por qué los ingenieros no diseñan aprendices?
Porque solamente es interesante diseñar un aprendiz cuando la modelación
del ^entorno^ en el que se desarrollará el artefacto puede hacerla mejor
el propio ingenio en tiempo real e \latin{in situ} que el ^ingeniero^
tranquilamente por adelantado y en su estudio. El casi hace referencia,
a día de hoy, a que estas condiciones se verifican únicamente en los
viajes no tripulados a otros planetas.


\Section La realidad

El nombre de ^aprendiz^ es cabal, porque aprende. El ^adaptador^, que
tantea, puede modificar su comportamiento para adaptarse a las
circunstancias en las que se encuentra, pero no aprende de sus errores.
El adaptador tiene que sufrir el inalterable proceso de ^prueba^ y
^error^ cada vez que su ^entorno^ varía, ya que no dispone de medios
para adelantarse a él. El aprendiz, en cambio, puede utilizar los
errores para modificar su ^modelo^ y así no repetir los errores. Es
decir, si el aprendiz ha ejecutado un comportamiento, es porque su
previsión era favorable, de modo que si el comportamiento ha fallado, es
porque el ^pronóstico^ ha fallado, y entonces el modelo ha de ser
mejorado. Si el ^modelo^ interior es modificado adecuadamente, el
aprendiz evita repetir los errores, y esto es aprender.

Para nosotros los subjetivistas, el modelo interior del aprendiz, que
imita el comportamiento del ^exterior^, es la ^realidad^. Y que la
^percepción^ construya una realidad de objetos, y no otra cosa, es,
desde este punto de vista más abstracto de la ^teoría del problema^, un
detalle de importancia menor.

No obstante, una cosa es cierta. Para poder internalizar el ^problema de
la supervivencia^, han de poderse representar internamente tanto la
realidad, que es el comportamiento del exterior, como los posibles
comportamientos del propio aprendiz. Es decir, lo que sí es requerido es
que el aprendiz sea capaz de representarse soluciones, en forma de
comportamientos. A los lenguajes que representan soluciones, pero no
resoluciones, los llamamos ^semánticos^.

Sea la realidad objetiva o de otra manera, el problema de la
supervivencia aparece trasladado dentro del simulador del aprendiz. El
subjetivismo permite distinguir nítidamente el problema de la
supervivencia de su interiorización por traslado, que denominaremos el
^problema del aprendiz^. El problema del aprendiz no es un ^problema
aparente^ porque el modelo es información o, en la parla subjetivista,
porque la ^realidad^ es ^información^.


\Section El conocedor

El adaptador y el aprendiz son capaces de varios comportamientos, esto
es, de varias soluciones, pero cada uno busca la solución de una manera
única que viene fijada genéticamente. De modo que el ^mecanismo^, el
^adaptador^ y el ^aprendiz^ son resolutores que resuelven cada uno de
una única manera. El siguiente paso de esta carrera en busca de la
máxima flexibilidad, a costa de una mayor complejidad, es el diseño de
resolutores capaces de varias resoluciones. Llamaremos conocedores a
estos resolutores múltiples. Un ^conocedor^ capaz de resolver según las
circunstancias como un mecanismo, o como un adaptador, o como un
aprendiz, puede superarlos a todos ellos; y ésta es la justificación
evolutiva del conocedor.

Lo dicho del adaptador vale para el conocedor, siempre que se traduzcan
adecuadamente las referencias a soluciones por referencias a
resoluciones. Así que, si el adaptador es capaz de varios
comportamientos, el conocedor es capaz de varias resoluciones. Además,
el cuerpo del adaptador se traduce en la ^mente^ del conocedor, que es
capaz de varias resoluciones. Y el gobernador del adaptador se traduce
en la ^inteligencia^ del conocedor, que es quien determina qué
resolución aplicar en cada momento.
$$\hbox{Conocedor}\llave{Inteligencia\cr Mente}$$

La propia ^evolución^ darviniana funciona como un conocedor cuya
inteligencia es combinatoria. Los ^conocedores^ de los que nosotros
descendemos, y que fueron diseñados por la evolución darviniana, tenían
una ^inteligencia emocional^. Con esto quiero decir que, para determinar
que resoluciones probar, no mutaban las resoluciones, ni las combinaban
por parejas, como hace la evolución darviniana, sino que usaban
emociones para seleccionar subproblemas y resoluciones. El hambre y la
perplejidad son dos de los ^sentimientos^ que influencian la resolución
a aplicar: el ^hambre^ porque selecciona un subproblema y desatiende los
otros, y la ^perplejidad^ porque se concentra en la mejora del ^modelo^.

Ya nos habíamos topado con los sentimientos. Entonces no profundizamos
en el asunto, y me limité a apuntar que los sentimientos más
fundamentales son, también, los significados más básicos. Ahora voy a
remediar la situación.


\Section La emoción

Para que tenga una visión más panorámica de nuestro linaje, intentaré
engarzar esta evolución abstracta deducida directamente de la ^teoría
del problema^ con nuestra propia línea evolutiva.

Los adaptadores de los que nosotros descendemos ya usaban ^objetos^. Por
esto es tan difícil erradicar el ^objetivismo^, arraigado como está
desde hace quinientos millones de años. El ^gobernador^ de estos
adaptadores seleccionaba el comportamiento actual en función de los
objetos que la percepción determinaba que estaban presentes. Para cada
combinación de objetos presentes, tenían predeterminado genéticamente un
^comportamiento^.

En cierto punto de nuestro linaje apareció un ^aprendiz^, esto es, un
organismo que era capaz de crear, destruir y modificar los objetos y las
relaciones entre los objetos. El propósito de estas variaciones era que
la red de objetos interna fuera capaz de pronosticar con acierto el
^comportamiento^ del ^entorno^ ^exterior^. Si todo iba bien, el
resultado era una ^realidad^ de objetos que se comportaba como el
exterior. ¿Tengo que repetir que esto es perfectamente posible sin que
fuera haya objetos?

El siguiente jalón tuvo lugar cuando apareció un ancestro capaz de
ejecutar varias resoluciones que seleccionaba en función de sus
^sentimientos^, o sea, un ^conocedor emocional^. Normalmente el
conocedor emotivo funciona como un aprendiz, pero cuando reconoce que el
^modelo^ interior no pronostica con acierto, esto es, cuando se siente
^perplejo^, puede tantear entre varios comportamientos predeterminados,
que es como funciona un adaptador. Otro sentimiento es el ^miedo^, que
ocurre cuando el conocedor emocional no encuentra en sus simulaciones un
comportamiento que solucione el problema. Pues bien, la ^teoría del
problema^ dice que, tampoco en estas circunstancias, debe el ^conocedor^
comportarse como un ^aprendiz^. Es decir, el conocedor debe evitar
aprender cuando siente miedo, porque generalizar en coyuntura tan
adversa podría hacerle caer en el ^atractor depresivo^ de \[Camus].


\Section El sujeto

El ^conocedor^ es un tanteador de resoluciones, y los tanteadores tienen
un defecto, a saber, que necesariamente yerran y no escarmientan, ya que
no aprenden de sus equivocaciones. Así que el ^adaptador^ no aprende a
comportarse, ni el conocedor aprende a resolver. Entonces, si el
conocedor era con las resoluciones lo que el adaptador con las
soluciones, ahora el ^sujeto^ será a las resoluciones lo que el
^aprendiz^ era a las soluciones. El aprendiz interioriza las soluciones
y el sujeto interioriza las resoluciones. El aprendiz aprende a
comportarse y el sujeto aprende a resolver.
$$\vbox{\halign{\strut
  \hfil#\quad&\vrule\quad\hfil#\hfil&\quad\hfil#\hfil\quad\vrule\crcr
  &           Tanteo&    Traslación\cr
  \noalign{\hrule}
  Solución&   Adaptador& Aprendiz\cr
  Resolución& Conocedor& Sujeto\cr
  \noalign{\hrule}}}$$

La ^teoría del problema^ no nos proporciona más ^categorías^, de manera
que, añadiendo el primer eslabón de la cadena, que es el mecanismo que
implementa la resolución rutinaria, completamos los cinco jalones de la
^evolución^ de resolutores que culmina en el sujeto.
$$\hbox{Mecanismo} \supset \hbox{Adaptador} \supset
  \hbox{Aprendiz} \supset \hbox{Conocedor} \supset \hbox{Sujeto}
$$

Continuando la comparación del sujeto con el aprendiz, el ^sujeto^ tiene
tres partes: una ^mente^, como la del conocedor, capaz de varias
resoluciones; un ^inquisidor^, que construye dentro del sujeto un
^modelo^ del problema externo; y una ^razón^ que prevé el resultado de
resolver de distintas maneras el problema encontrado por el inquisidor.
Como la razón del sujeto tiene que representar y ejecutar resoluciones,
ha de emplear un ^lenguaje simbólico^, con ^sintaxis recursiva^. Esto
exige que la razón del sujeto sea un ^motor sintáctico^~${\cal P}_{\frak
U}$. Ya era hora de empezar a cuadrar filosóficamente los resultados
obtenidos siguiendo a \[Gödel] y \[Turing].
$$\hbox{Sujeto}
    \llave{Inteligencia
            \llave{Inquisidor\cr
                   Razón = ${\cal P}_{\frak U}$}\cr
           Mente\strut}$$


\Section La consciencia

Lo más propio del ^sujeto^ es el simbolismo con sintaxis recursiva que
le permite representarse internamente problemas y resoluciones. Y basta
entonces percatarse de que el propio sujeto es un resolutor, para
concluir que el único resolutor que puede representarse a sí mismo es el
sujeto. El sujeto puede representarse la situación completa, esto es, el
problema al que se enfrenta y a sí mismo como resolutor. Queda así
demostrada matemáticamente la ^consciencia^ y la ^autoconsciencia^ de
los sujetos.

Piense en ello, porque siendo, como es, requisito indispensable que los
sujetos dispongan de una ^razón^ con ^lenguaje simbólico^, y siendo
nuestra especie \latin{^homo sapiens^} la única que emplea un lenguaje
simbólico, la conclusión es que los únicos sujetos vivos somos nosotros.
Somos la consciencia de la ^vida^.

El mismo argumento, a saber, que un simbolismo permite la representación
de resoluciones, sirve para fundamentar la ^empatía^ de los sujetos. El
único resolutor que puede representarse a otro resolutor es un sujeto,
porque sólo un sujeto puede representarse resoluciones. Resulta,
entonces, que nosotros los sujetos podemos ponernos empáticamente en el
lugar del ^patito^ que se ha perdido de su madre, pero un pato no puede
ponerse en el lugar de otro pato. ¿Le parece extraño? Puede parecerlo,
pero, si mi ^teoría del problema^ es correcta, así es como es, y no de
otra manera.


\Section Hipótesis

Hemos tenido que conectar tantos conceptos para alcanzar nuestra meta de
explicar la ^consciencia^ ---^vida^ con ^problema^, ^resolución^ con
^simbolismo^, ^motor sintáctico^ con ^razón^, y ^sujeto^ con
^evolución^--- que ahora mismo, de súbito, mi demostración de la
consciencia podría parecerle confusa, o aun inconcebible. Para intentar
aclarar sus dudas, o, por lo menos, para centrárselas, le resumiré los
hechos y las hipótesis que nos han traido hasta aquí. Aviso: calificaré
como ^hecho^ cualquier aserción que tenga un grado de veracidad, no
absoluto, ya que no creo en tanto por culpa de \[Lakatos], pero tampoco
menor al de una demostración matemática.

Es un hecho que, para poder representar las resoluciones de los
problemas, han de cumplirse ^cuatro condiciones^ concretas: oraciones en
forma de árbol, variables libres, asignación ilimitada y condiciones. Es
un hecho que el _cálculo<lambda>~$\lambda$ de \[Church] queda definido
necesaria y suficientemente por esas cuatro condiciones. Es un hecho que
todas las sintaxis recursivas son equivalentes; en particular el
cálculo~$\lambda$ es equivalente a los lenguajes tratados por las
^máquinas universales^ de \[Turing], que disponen de un ^motor
sintáctico^. Es un hecho que un resolutor que disponga de un motor
sintáctico será capaz de representarse resoluciones y, por lo tanto, a
sí mismo ---y he denominado ^sujeto^ a un resolutor con motor
sintáctico.

Es una ^hipótesis^, aunque incontestada desde su formulación en 1936, que
cualquier transformación sintáctica que podamos ^imaginar^ puede
realizarla una máquina universal de \[Turing], o, más específicamente,
su motor sintáctico, siempre que disponga de bastante tiempo y de una
cinta lo suficientemente larga ---esta es la ^tesis de
\[Church]-\[Turing]^. Y, que la tesis de \[Church]-\[Turing] sea
verdadera, significa que nosotros diponemos de un motor sintáctico. Es
una hipótesis que la ^vida^ es un ^problema^, {\em el} problema, y, si
lo es efectivamente, entonces nosotros somos sujetos vivos, es decir,
resolutores con motor sintáctico del ^problema aparente de la
supervivencia^.

Los párrafos anteriores son densos, lo sé. Lo que no sé es cómo explicar
por qué aprendemos antes a hablar que a dibujar sin tanto ni tan
sofisticado aparato teórico. Y es que, claro, la inocente preguntita me
ha obligado a explicar la consciencia, que no es fácil, ¿o sí? En
cualquier caso, ya iba siendo hora de reanudar la búsqueda de ^grietas^,
que, con tanta elucubración, teníamos algo abandonada. Además, me parece
que lo que viene es más fácil. ¡Sígame, por favor!


\Section El nacimiento

Para empezar, le plantearé una ^aporía^ relacionada con la
^consciencia^. No se recuerda el ^nacimiento^ propio ni a los cuatro
años de edad, cuando la retentiva es grande, sólo han pasado cuatro
años, y el nacimiento es un suceso altamente extraordinario. ¿Por qué no
recordamos nuestro nacimiento?

Según la teoría presentada, que liga la consciencia al ^lenguaje
simbólico^, la causa podría ser que al nacer todavía no hemos
desarrollado el ^habla^. O sea, que los recién nacidos no disponen de un
lenguaje simbólico, por esto no son conscientes, y sólo se recuerda
conscientemente lo experimentado conscientemente. Yo, desde luego, así
lo creo. Veamos.

Sólo tenemos acceso consciente a lo que es tratado por la ^razón^, y la
razón es un ^motor sintáctico^ que sólo trata con palabras. Las
^palabras^ simbólicas son objetos ---ya lo veremos---, pero no todos los
objetos son palabras. Entonces, el único enlace que tiene la consciencia
con los objetos no verbales, que son los objetos que no son palabras, es
a través del puente de las palabras semánticas, porque éstas se
refieren, precisamente, a objetos que no son palabras. Y, para poder
recuperar a la consciencia objetos no verbales, antes ha de establecerse
una conexión entre dichos objetos y las palabras.
$$\hbox{\sl Habla\/} =
  \hbox{\sl Razón\/} =
  \hbox{\it Ratio\/} =
  \hbox{\it Logos\/} =
  {\mit\Lambda}\acute o\gamma o\varsigma$$

La consciencia sólo alcanza directamente las palabras. Lo que no se
verbaliza, ni como ^pensamiento^ ni como ^habla^, queda inaccesible a la
^consciencia^. Al menos, directamente.


\Section El despertar

La ^muerte^ no puede experimentarse directamente ---yo, al menos, no se
lo aconsejo. Pero, hay un asunto relacionado que sí podemos investigar:
la pérdida de ^consciencia^. Cada día, cuando nos quedamos dormidos,
pasamos necesariamente de la consciencia a la falta de consciencia, así
que yo he intentado muchas veces recordar el instante preciso en el que
se produce el tránsito. Nunca lo he conseguido, y sospecho que es
imposible. ¿Por qué es imposible ser consciente del último instante de
consciencia antes de dormir?

Yo me he buscado una explicación, que examina primero el ^despertar^. Si
estoy dormido en un instante $t$, y soy despertado bruscamente en el
instante siguiente $t+1$, como todavía está en la ^memoria a corto
plazo^ lo sucedido en $t$, resulta que el instante $t$ pasa a ser
consciente. Por eso, en estos casos, se recuerda el ^sueño^ que se está
teniendo. Pero, si no me hubieran despertado en $t+1$, entonces el
instante $t$ seguiría siendo inconsciente. Esto demuestra que en el
instante $t$ puede no saberse si en el propio instante $t$ estoy
consciente, o no lo estoy. ¿No es inquietante?

Algo similar ocurre cuando nos quedamos dormidos. No es posible recordar
el último instante de consciencia, porque para recordarlo
conscientemente sería preciso ligarlo a algunas palabras, y para
realizar tal operación sería necesaria una acción consciente, ya que es
con palabras, en el instante siguiente, prolongándose entonces la
^vigilia^. De modo que es imposible recuperar el último instante de
consciencia que se tuvo justo antes de dormir. Esto es curioso, porque
significa que podemos ser conscientes en un instante que vamos a ser
incapaces de recuperar. Y esto es extraño, porque esto es como lo que
ocurre al soñar, y lo que, en cambio, no sucede cuando estamos
despiertos. Expresado paradójicamente, resulta que el último instante en
el que estamos despiertos, ya estamos durmiendo.

Somos conscientes mientras estamos despiertos, aunque la consciencia
puede recuperar de la memoria a corto plazo un retazo de sueño y, a
cambio, cuando nos quedamos dormidos, perdemos los últimos instantes
de consciencia.


\Section El sueño eterno

No somos sujetos al nacer, así que todos sabemos lo que significa no ser
sujetos. Es actuar sin que nuestros actos nos dejen recuerdos
conscientes de lo que hemos hecho. Es como vivir un ^sueño^ eterno. ¿No
sería una ^pesadilla^?

\eject

Pero tampoco tiene que retroceder a su niñez para hacerse una idea de la
diferencia que supone la ^consciencia^. Supongamos que usted es un
cocinero experimentado, que su plato favorito es la ^tortilla^, y que ha
hecho miles de ellas. Supongamos que hoy está muy preocupado y que
mientras preparaba la tortilla no ha dejado de pensar en el asunto que
le inquieta. Entonces le preguntan si hay ^mantequilla^ en la ^nevera^.
Usted no lo sabe. Aunque se da cuenta de que ha tenido que abrirla para
tomar los huevos, y que la mantequilla suele colocarse en la misma
repisa que los huevos, no es consciente de haber abierto la nevera
porque no lo recuerda. Mucho menos puede decir si hay mantequilla
dentro. Ha cocinado la tortilla sin ser consciente de ninguno de los
pormenores de su actividad.

Aunque usted no haya experimentado este suceso concreto, estoy seguro de
que puede contar muchos otros casos similares en los que ha realizado
una actividad sin ser consciente de ella. Si, por ejemplo, hace cada día
el mismo trayecto para ir desde su casa hasta la oficina en donde
trabaja, es fácil que sus pensamientos le distraigan la atención, y sea
inconsciente del camino recorrido. Si va caminando, entonces es seguro
que ha evitado a los otros transeuntes y esquivado las farolas, aunque
no se haya percatado de ello. Esto muestra que el comportamiento no
consciente puede ser muy complejo, y que la complejidad no marca la
diferencia que media entre un acto consciente y otro que no lo es.

Pero, si se encuentra de pronto algo inesperado en su ^rutina^ diaria,
entonces abandona sin dilación sus pensamientos y atiende inmediatamente
a lo que sucede a su alrededor. Usando los conceptos de la ^teoría del
problema^, la explicación es que cuando falla la resolución por rutina
es necesaria otra resolución, y por esta causa la ^atención^ se centra
en el problema a resolver. Si fuéramos conocedores emocionales, entonces
tantearíamos otra resolución guiándonos por nuestros sentimientos. Como
somos sujetos, razonamos sobre varias posibles resoluciones que
evaluamos mentalmente antes de decidirnos por una.

Merced al ^lenguaje simbólico^, los sujetos podemos representarnos
internamente las resoluciones posibles para ver, en su acepción
introspectiva, cuál soluciona mejor el ^problema^ antes de acometer
actualmente la resolución más prometedora. Esto es ^razonar^.


\Section La reflexión

A mi me parece que la ^consciencia^ va monitorizando lo que pasa por la
^memoria a corto plazo^. Lleva necesariamente un pequeño retraso con
respecto a lo que ocurre, pero proporciona ^reflexión^. Literalmente, la
consciencia refleja lo que vemos sobre lo que vemos. Por eso, se ha
confundido muchas veces la consciencia con un ^homúnculo^, esto es, con
un pequeño hombrecillo que ve lo que nosotros vemos y entonces toma las
decisiones conscientes. El truco, como digo, no lo hace un homúnculo,
sino un ^espejo^.

Al nivel más abstracto de la teoría del problema, la reflexión es
posible merced a la ^sintaxis recursiva^ del ^lenguaje simbólico^, que
permite ver resoluciones a un resolutor, el sujeto. Pero quizás sea más
interesante investigar la situación en nuestro caso concreto, ligado al
uso de objetos.

Decíamos que la ^percepción^ tomaba los datos de los sentidos, como por
ejemplo los puntos de color de la retina, y entregaba un dibujo a modo
de resumen de la situación, en el que aparecían trazados los límites
entre los ^objetos^ reconocidos. Si la imagen retinal es como la
^fotografía^ aérea de una región, los objetos son las parcelas
etiquetadas que superponemos a la foto para interpretarla. El propósito
de la percepción es _resumir<comprimir> ^información^, básicamente
señalando qué objetos están presentes. Lo repito una vez más, aun a
riesgo de ser pesado: no es que fuera haya objetos, es que nosotros
trabajamos sobre resúmenes sumarísimos en forma de objetos y, lo más que
podemos decir, es que esta simplificación funciona en la práctica lo
suficientemente bien.

En los lenguajes semánticos, la palabra es una propiedad adicional del
objeto, como su forma o su color. La primera diferencia entre los
lenguajes semánticos y los simbólicos es que la ^palabra^ en un
^lenguaje simbólico^ es, ella misma, un ^objeto^. Como los objetos ya
eran las entidades cognitivas autónomas para nuestros antepasados, la
manera que la ^evolución^ encontró de implementar las palabras meramente
sintácticas, que no se refieren a ningún objeto semántico, o sea, que
están libres de significado y, por lo tanto, son autónomas, fue hacer
que las palabras fueran ellas mismas objetos, y no propiedades de
objetos. No me entienda mal, seguramente ocurrió al revés, y fue el
hecho más o menos fortuito de hacer objetos con las palabras el que hizo
posible, como efecto colateral no buscado, la aparición del simbolismo
y, consiguientemente, de la reflexión.


\Section La escritura

Independientemente de como haya sido la secuencia histórica y las causas
del origen del simbolismo en los humanos, es importante que nosotros
distingamos las propiedades de la ^palabra^ de las propiedades de la
^cosa^ a la que se refiere la palabra simbólica. Por ejemplo, la palabra
`^piedra^' es bisílaba, pero la cosa piedra no tiene dos sílabas.

Las ^palabras semánticas^ son aquéllas que, como la palabra `piedra', se
refieren directamente a un ^objeto semántico^. Pero, en general, las
palabras de los lenguajes simbólicos, al ser objetos enteros y no meras
propiedades, no tienen que referirse necesariamente a objetos
semánticos. De esta manera, las palabras simbólicas pueden ser meros
artefactos sintácticos. Por ejemplo, la palabra `^qué^' es puramente
sintáctica y no hace referencia a ningún objeto semántico.

Que las palabras sean objetos tiene otra consecuencia, seguramente
tampoco buscada, pero también importante. Cada objeto tiene varias
propiedades distintivas, de manera que puede ser reconocido por
cualquiera de ellas ---el olor de la carne en la parrilla nos sirve tan
bien para reconocerla como si la viéramos. Pues cuando la ^palabra^ es
un ^objeto^, también puede ser reconocida de diferentes modos, y por eso
las palabras simbólicas pueden ser vistas, además de oídas. La
^escritura^ es posible porque las palabras simbólicas son objetos.


\Section Una vida de perros

Suponga que describe usted en un ^lenguaje simbólico^, por ejemplo en su
idioma materno, lo que está viendo. Las palabras de su descripción
tomarán el dibujo producido por su ^percepción^, y señalarán los objetos
semánticos que estén presentes en él. Compare ahora lo que hace al
describir, con lo que hace al ver. ^Ver^ es tomar la imagen fotográfica
de la ^retina^ y señalar qué objetos están presentes, así que el
lenguaje simbólico permite, no ver lo que hay fuera, sino ver lo que se
ve. Así es como la ^reflexión^ produce ^introspección^.

No se quede ahí. Intente imaginarse lo que es ver sin poder ver lo que
ve. ¿Puede? Es, otra vez, como un ^sueño^, o como lo que hace abstraido.
En el sueño ve lo que ocurre, le inspira sentimientos, pero ni actúa
voluntariamente ni reflexiona sobre ello, así que queda olvidado, como
si no lo hubiera vivido. Aunque esto no es del todo así, porque le
queda, aunque inaccesible para la ^consciencia^, del mismo modo que que
le quedan los acontecimientos de la más tierna infancia. Así es la vida
de los perros y de otros seres vivos que calificaríamos de conocedores.

La vida de los conocedores está siempre atada al ^presente^. Los
conocedores viven ligados, engranados y nunca distanciados de su propio
presente. ^Ven^ lo que ocurre, pero no ven lo que ven, y, entonces, no
pueden reinterpretarlo. Ven, pero no ven; porque ven, en su acepción
perceptiva, pero no ven, en su acepción introspectiva. Sólo pueden
evaluar lo que ocurre en el instante en el que está ocurriendo, es
decir, en tiempo real, que diría un ^ingeniero^ en su jerga.

Tanto a los conocedores como a los aprendices el ^pasado^ les influye
porque su actividad va modificando la ^realidad^, que es su ^modelo^ de
cómo se comporta lo ^externo^. Por ejemplo, un ^perro^ recuerda si su
amo ya llegó a casa. Cuando su amo está en casa, el perro está más
contento, emoción que inferimos de su comportamiento, que entonces es
más dinámico y alegre. Pero no recuerda del mismo modo que nosotros y,
por ejemplo, no puede rememorar el momento de la llegada. Es decir, si
su amo está o no en casa es un dato de su modelo interno, de su
realidad, que el perro actualiza adecuadamente y que efectivamente
condiciona sus emociones y su comportamiento, pero que no puede manejar.
El pasado para los conocedores es como la ^percepción^ para nosotros los
sujetos, es decir, es algo que viene dado y es inalterable, es un dato
impuesto e involuntario, y no como nuestros recuerdos, que podemos
rememorar trayéndolos voluntariamente al presente.


\Section El símbolo

Así que el ^pasado^ influye en los conocedores, aunque no puedan
rememorar. Es decir, los conocedores no pueden volver a ver. No pueden
hacer que los mecanismos perceptivos, capaces de analizar las
situaciones, retomen situaciones pasadas. Los conocedores no disponen de
^introspección^, no ven su propio interior. Y como tampoco ven el
interior de los demás, tampoco disponen de ^empatía^.

Para ver lo que se ve, la ^palabra^, que en un principio era una
propiedad más del ^objeto^, tuvo que ser ella misma un objeto. Así que,
en la ^revolución simbólica^, la palabra pasó, de ser un signo sonoro
que servía para reconocer un objeto, a ser todo un objeto. A la palabra
que es objeto la denominamos ^símbolo^. Como las palabras simbólicas
son, ellas mismas, objetos, pueden ser etiquetas de otras palabras
simbólicas, implementando así la recurrencia sobre los objetos. Hay
otros pasos que también son necesarios para alcanzar la ^recursividad^
---¿se acuerda de las cuatro condiciones?---, pero éste fue seguramente
el primero y decisivo.

En fin, es importante que entienda la diferencia que existe entre
nosotros los sujetos y los conocedores. Y aun más que la comprenda
dentro de usted mismo, es decir, que sea usted capaz, al menos, de
atisbar cómo se enfrenta al problema de la supervivencia un ^conocedor^,
como lo es, por ejemplo, su perro. A mi me produce escalofríos, pero,
además de aterrador, es esclarecedor.


\Section Freud

A pesar de lo que afirmaba \[Calderón], la vida de los sujetos no es
^sueño^. Aunque casi es cierto que seguimos soñando cuando estamos
despiertos y conscientes. Quiero decir que, como descubrió \[Freud], la
vida del ^sujeto^ no coincide con la vida consciente del sujeto. Hay, en
todo momento, una parte que es ^inconsciente^. En los sueños que
recuperamos a la consciencia cuando el ^despertar^ es brusco, podemos
observar el funcionamiento en solitario del inconsciente, pero esto no
implica que cuando estamos despiertos y conscientes operen únicamente
los procesos conscientes. Lo hacen, pero no somos conscientes de ellos.
La acción de la consciencia no apaga el inconsciente, sino que se
superpone a él. Otra ^tautología^: no somos conscientes de nuestros
procesos inconscientes.

La ^consciencia^ es sólo un espejo levísimo y sutilísimo que refleja la
superficie de un vasto, profundo y poderoso mar. Sutilísimo porque el
reflejo lo produce el ^habla^, así que son unos simples sonidos efímeros
e insustanciales ---las palabras--- los que nos dejan ver lo que vemos.

Es mucho más lo que desconocemos que lo que conocemos de nosotros
mismos. Algunos poetas lo adivinaron ya hace algún tiempo: nuestro yo
consciente es una suerte de novela que cada uno va escribiendo sobre uno
mismo; usted y yo también. Y da miedo pensar que las ^autobiografías^
son el género literario menos veraz, menos que la ^biografía^ y la
^historia^, y mucho menos que la ^sátira burlesca^ y el ^libelo
difamatorio^. Para que no pueda desdecirme en un futuro, prometo
solemnemente, aquí y ahora, que denunciaré, sin reparar en gastos y por
todos los medios que hubiere a mi alcance, al editor que se atreviere a
publicar mi autobiografía. Por su provecho, le animo a que se adhiera
usted a este juramento.

La ^consciencia^ del ^sujeto^ coincide con su ^razón^, que es aquella
parte en la que puede representarse las resoluciones, y así evaluarlas
mentalmente. Si usted divide racionalmente un ^problema^ en otros dos,
entonces conocerá las razones por las que lo hizo. Pero es mucho más lo
que usted hace y determina fuera del alcance de su consciencia,
muchísimo más, que lo que decide racionalmente. Por este motivo, usted
desconoce la causa de sus sentimientos, el origen de la ^realidad^, y el
sentido de la ^vida^. ¿Por qué vivimos? Yo tampoco lo sé.

Nuestra ^introspección^ es muy limitada y superficial. Vemos cada vez un
único ^problema^, y no vemos sus raíces, sino únicamente lo que aflora a
la ^consciencia^. Así, nuestras explicaciones de nuestros propios
propósitos son frecuentemente racionalizaciones urdidas por nuestra
parte consciente, que hacen inferencias a partir de los resultados, pero
sin llegar a los orígenes, y, por lo tanto, sin alcanzar los motivos
profundos ni las causas primeras de nuestros propios actos.


\Section Pensar es hablar sin decir

Estas afirmaciones se siguen fácilmente del hecho de que nosotros no
somos unos ^sujetos^ cualesquiera, sino que somos unos sujetos muy
concretos diseñados por la ^evolución^ darviniana a partir de unos
^conocedores^, desconocidos ahora, pero que en su día fueron también
unos conocedores muy concretos. Nuestros antepasados fueron unos
conocedores concretos que descendieron, posiblemente, de otros
conocedores concretos. Antes, en nuestra línea evolutiva hubo
^aprendices^, y aun antes ^adaptadores^, y al principio ^mecanismos^.

Nuestra línea de ^evolución resolutiva^ ha pasado, creciendo en
complejidad, por los siguientes jalones:
$$\vbox{\halign{#\hfil&\quad#\hfil\cr
 Mecanismo& Resolución por rutina\cr
 Adaptador& Resolución por tanteo de soluciones\cr
 Aprendiz&  Resolución por traslación de soluciones\cr
 Conocedor& Resolución por tanteo de resoluciones\cr
 Sujeto&    Resolución por traslación de resoluciones\cr
 }}$$

Somos parte de una línea evolutiva porque la ^evolución^ darviniana
diseña por ^tanteo^ combinatorio y acumulativo con mutaciones. Es decir,
que tenemos cuatro miembros y cada miembro cinco dedos, no porque ésta
sea una decisión de diseño especialmente adecuada al proyectar un
^sujeto^, sino porque, en algún momento de nuestra línea evolutiva, un
diseño con esas características tuvo la oportunidad y la suerte de
prosperar y dejar descendencia. Nosotros simplemente heredamos esas
peculiaridades.

Es así que somos un conocedor que habla. El ^pensamiento^ no es otra
cosa que un ^habla^ internalizada. Esto ya lo vio en 1934 \[Vygotsky]
estudiando el desarrollo psicológico de los ^niños^, y más recientemente
\[Bickerton] en sus investigaciones sobre las lenguas francas, que no
francesas. Tuvo que ser así, y además tengo una prueba empírica. Se
trata de una disculpa: `perdona que no te haya entendido, pero es que
estaba pensando en otra cosa'. La disculpa funciona porque es cierto que
no podemos entender lo que nos dicen cuando estamos pensando en otros
asuntos. Y la explicación más plausible del fenómeno es que también al
pensar utilizamos el mecanismo que descodifica el habla, quedando así
ocupado. Sí, lo está entendiendo bien, le estoy diciendo que al pensar
tenemos que escucharnos a nosotros mismos.

Pensar es hablar con uno mismo, y hablar es pensar entre varios.


\Section El motor sintáctico

Somos ^sujetos^ diseñados a partir de un conocedor. Al parecer, nuestro
diseño ha consistido, simplificándolo casi hasta la caricatura, en tomar
un ^conocedor^ de entre los primates para añadirle, de alguna manera, un
^lenguaje simbólico^. Como sabemos, además de los mecanismos necesarios
para producir y descodificar secuencias sonoras complejas, cuyo diseño
no nos interesa aquí especialmente, lo esencial e indispensable para
procesar un lenguaje simbólico es disponer de un ^motor sintáctico^.

Entiéndalo bien porque, aunque no es fácil, es fundamental. Una vez que
se dispone de un motor sintáctico, pero no antes, la decisión de cómo
implementar cada procedimiento, física o algorítmicamente, es una mera
cuestión de eficiencia, y ya no de eficacia. Se trata de un compromiso
ingenieril en el que se intercambia ^velocidad^ por ^complejidad^. El
procedimiento se ejecutaría más rápido físicamente, pero para ello
habría que complicar el motor sintáctico con operaciones que no son
estrictamente necesarias.

Para que se haga una idea, es como decidir qué teclas poner en una
^calculadora^. La de la multiplicación, aunque es redundante porque una
^multiplicación^ es una ^suma^ reiterada, aparece en todas porque se usa
con mucha frecuencia. En cambio, la tecla del ^logaritmo^ neperiano,
aunque ahorra todavía más operaciones, sólo la tienen las calculadoras
científicas, porque se emplea raramente. Pues bien, el descodificador de
secuencias sonoras complejas y el preprocesador que hace el ^análisis
sintáctico^ superficial de \[Chomsky] no son esenciales porque el ^motor
sintáctico^ podría suplirlos, aunque tendríamos que hablar muchísimo más
despacio.

La cantidad de ^información^ necesaria para describir un ^motor
sintáctico^ no es muy grande, del orden de los diez mil bitios. De
hecho, es mucho menor que la precisa para describir los mecanismos que
descodifican la secuencia sonora del habla. Y los componentes necesarios
para construir un motor sintáctico pueden ser simplicísimos; son
suficientes puertas lógicas elementales o, incluso, como nos enseñaron
\[McCulloch] y \[Pitts], ^neuronas^ simplificadas. Además, se aprovechó
toda la maquinaria de manipulación de objetos y propiedades que ya
utilizaban nuestros antecesores primates.


\Section El diseño por acreción

Otro punto del diseño de nuestra especie como sujetos consiste en que el
^motor sintáctico^ está acoplado al resto de la maquinaria cognitiva por
la ^memoria a corto plazo^. Sólo podemos conjeturar sobre las ventajas e
inconvenientes de este punto de ensamblaje frente a otras posibilidades.
Podemos suponer que no requirió modificaciones muy profundas del
cerebro, y que proporcionó unas ventajas suficientes para perdurar.

Seguramente la memoria a corto plazo ya era, para nuestros antecesores,
el punto en el que el mecanismo de la ^atención^ reunía los resúmenes
más importantes elaborados por los sentidos y los órganos perceptivos y
propioceptivos, y desde el que se generaban las secuencias de órdenes de
actuación hacia los órganos efectores y los músculos. De ser así, la
memoria a corto plazo ya era entonces el lugar del ^control^ máximo y
más global.

Así que una de las consecuencias del diseño incremental y acumulativo
que es típico de la evolución darviniana, es que nosotros no somos unos
sujetos diseñados integralmente, sino por ^acreción^. Pero no vamos a
quejarnos ahora, digo yo. Además, ¿a quién demandar?


\Section El homúnculo

En esta descripción, la situación de la ^consciencia^ racional con
respecto al conjunto del sujeto semeja a la del ^presidente^ de una
enorme ^empresa^ que no puede salir de su ^despacho^. Al presidente le
llegan informes globales sobre el estado de su propia corporación, sobre
la competencia, y sobre todos los asuntos relativos a su negocio. Estos
informes le pueden llegar periódicamente, o a petición propia, o incluso
por decisión de sus subordinados. De los informes, el presidente puede
deducir qué es lo que va mal, o qué puede llegar a convertirse en un
^problema^, y ordenar en consecuencia las acciones correctivas que
estime oportunas. También puede tomar la iniciativa para mejorar la
situación de la empresa, si deduce de los informes que el momento es
propicio. Por último, los problemas que no hayan podido ser solucionados
a otro nivel corporativo, le serán planteados al presidente para su
resolución.

Si la empresa es grande y compleja, creer que lo único que sucede es
aquello que el presidente observa y decide es, en palabras llanas, un
grave error de perspectiva. El mismo error, traspuesto al caso del
sujeto, es creer que solamente ocurre aquello de lo que el ^sujeto^ es
consciente. Por ejemplo, nos parece que ^ver^ es más fácil que ^dividir^
porque somos inconscientes de los procesos perceptivos preparatorios y
sólo somos conscientes de su resultado ya acabado ---¿se acuerda del
^restaurante^?---, igual que al presidente le parece que su trabajo es
el más difícil, y cree que por eso es el mejor pagado.
% Quién tiene el poder de fijar los sueldos.

En los sistemas políticos de ^gobierno^, suele ser muy acertado limitar
el tiempo de permanencia en el ^poder^. Esta disposición, además de
evitar el nocivo enquistamiento de las estructuras clientelares, evita
el enojoso distanciamiento entre las necesidades del ^pueblo^ y las de
sus gobernantes. Observe usted que, trasladándonos de nuevo al ^sujeto^,
esta ruptura entre las necesidades conscientes y racionales, por un
lado, y las necesidades corporales y afectivas, por el otro, se
corresponde con la ^demencia^.


\Section El robot

Antes de seguir quiero zanjar definitivamente una cuestión. La
diferencia fundamental entre una ^computadora^ y una persona es que la
persona humana es un ^sujeto^ del ^problema de la supervivencia^
diseñado por ^acreción^ a partir de un ^conocedor emotivo^, y la
computadora es un ^motor sintáctico^.

La computadora no es un sujeto completo, sino una de sus partes,
básicamente la que corresponde a la ^razón^. Puede razonar con total
precisión y a enorme velocidad, pero, como le falta un ^inquisidor^, no
ve el mundo como un ^enigma^, que es como lo vemos los sujetos. Que la
computadora no sea un sujeto, no quiere decir que no se pueda construir
una máquina que sea un sujeto de cierto problema. Es posible, y, además,
la razón de tal sujeto tendría que ser una computadora. Denominaremos
^robot^ a cualquier artefacto que sea un sujeto.

Se abren, en este punto, varias vías de análisis para el investigador
interesado en los robotes. La principal se refiere a la elección del
^problema del robot^: ¿debería ser el ^problema de la supervivencia^? De
serlo, el robot sería en cierto modo parte de la ^vida^, pero, para
serlo de pleno derecho debería ubicarse en la ^cadena trófica^, que es
la cadena alimentaria, así que sus adquisiciones de energía deberían
producirse por ^combustión^ de materia orgánica, y también su
constitución material debería componerse exclusivamente de materia
orgánica. El otro problema ingenieril a resolver para crear un robot
vivo concierne a su reproducción. Los peligros, para la propia vida, de
diseñar un robot vivo, o sea, una máquina que sea un sujeto del problema
de la supervivencia, son grandes y las dificultades técnicas mayores que
si se elige otro problema.

Si, animados por estas advertencias, los robotes se diseñaran para
ayudar a las personas, entonces su problema no sería el de la
^supervivencia^, y las intuiciones de \[Asimov] podrían ser realizadas.
En este caso, el diseño del robot como ^sujeto^ sería mucho más limpio y
nítido que el de nosotros los humanos. De todos modos, si uno de los
requisitos del diseño de los robotes es que se puedan entender y
comunicar con nosotros, entonces deberán imitar algunas de nuestras
peculiaridades.


\Section Humano, demasiado humano

Para mi, lo más interesante del diseño de los ^robotes^ atañe a su
^consciencia^. La nuestra, como consecuencia de haber sido diseñados por
^acreción^, está estrechamente ligada a la ^memoria a corto plazo^ y por
esto ejerce un ^control^ centralizado, pero la consciencia de los
robotes puede ser mucho menos angosta y mucho más distribuida. De ser
así, la consciencia y el yo del robot no podrán valerse de la metáfora
del ^homúnculo^ ^presidente^. La cuestión es si, a pesar de estas
diferencias, todavía podríamos entendernos.

La otra divergencia grande entre nosotros los humanos y los robotes
concierne a los sentimientos que, en nuestro caso y por el mismo motivo
de diseño, tienen un peso muy grande, incluso demasiado grande. Nosotros
somos, al fin, unos conocedores emotivos que hablamos, ya que la
inteligencia de los conocedores de los que descendemos era emocional.
Nada de esto ha de tenerse en consideración al proyectar un robot, a no
ser que se pretenda que entiendan nuestra ^música^.

De antiguo se sospecha que sería ventajoso un ^comportamiento^ más
racional y menos emocional que el humano, así que, si se confirma la
sospecha, los diseños robóticos prescindirán de los sentimientos y serán
completamente racionales. Lo curioso es que nosotros los humanos no
podremos evitar juzgar como muy fríos unos comportamientos más
racionales que los nuestros. Dicho de modo ^tautológico^, que es como a
mi más me gusta, los robotes serán inhumanos.

Esto, lo sé, va a asustar a muchos ---no sé si también a usted--- porque
la ^razón^ produce monstruos. Es fácil ver el razonamiento lógico e
implacable que ha llevado a una fría ^asesina^ a envenenar a su
adinerado ^marido^ para heredarlo tras enamorarse de otro hombre. En un
caso así, ya es peligroso el mero hecho de considerar racionalmente las
ventajas y los inconvenientes de resolver el problema por ^homicidio^.
También es inmoral meditar sobre el interés de una ^guerra^.

Solamente añadiré dos apuntes más sobre este asunto, porque aún es
pronto para alcanzar un juicio más ecuánime y definitivo. Uno es que la
vida salvaje es más cruel que la humana ---lo contrario, aunque nos lo
hayan dicho, es un mito falso. Otro, que, en los ejemplos anteriores, la
irracional carga emotiva es enorme. Quiero decir que, aunque en los dos
casos citados se usa la ^razón^, el origen mismo del conflicto es
claramente pre- e irracional. En el primer caso es el ^instinto sexual^,
y en el segundo es el ^instinto territorial^ o de ^poder^.


\Section El problema del sujeto

La ^consciencia^ es el modo de ^conocimiento^ característico de los
sujetos. En nuestro caso, la consciencia es una capa superficial
adherida a la ^memoria a corto plazo^, que utiliza palabras para obtener
el reflejo. Con mayor libertad de diseño, la consciencia será integral y
la ^reflexión^ comprehensiva. Sea como fuere, el ^yo^ es el ^modelo^
consciente que tiene el ^sujeto^ de sí mismo. El sujeto, lo repito
porque es la clave de mi ^teoría de la subjetividad^, es el único
^resolutor^ que puede representarse a sí mismo.

El sujeto es un resolutor y, en consecuencia, se representará a sí mismo
como una ^resolución^. Concretamente, el sujeto se representará a sí
mismo como la resolución que considera que es la mejor contra el
^problema de la supervivencia^, y que, por ese mismo motivo, es la que
está aplicando actualmente. Vale, cuando tiene usted razón tengo que
dársela, porque, efectivamente, no es exactamente contra el problema de
la supervivencia, sino contra la representación del problema de la
supervivencia que ha elaborado el inquisidor del sujeto. Y, para no
confundirnos nunca más, llamaremos ^problema del sujeto^ a la versión
del problema de la supervivencia interiorizada por su ^inquisidor^.

Antes de ponderar las diferencias que pueden existir entre el problema
de la supervivencia y el problema del sujeto, quiero que vea la
diferencia que existe entre el ^problema del aprendiz^ y el problema del
sujeto. Tal vez le convenga repasar ahora lo dicho antes sobre el
aprendiz; no hace falta que yo se lo repita porque esto es un libro, así
que sigue estando en donde estaba cuando lo leyó. No me sea usted vago
y, si no lo recuerda, reléalo. Que dónde está, que por qué no le hago
una referencia; no me haga enfadar, ¿para qué cree usted que me he
esforzado en elaborar un ^índice alfabético^? Bueno, venga, se lo
explico en la sección siguiente.


\Section El mundo es un enigma

La forma del ^problema del aprendiz^ viene fijada por diseño, es decir,
que los aprendices vivos la heredan genéticamente. Los aprendices
modelan el ^comportamiento^ externo pero no el ^problema^ externo,
porque su lenguaje de representación es ^semántico^ y no simbólico. En
cambio, los sujetos somos inquisitivos, porque modelamos el problema
externo. Para nosotros los sujetos el mundo es un ^enigma^, y por esto
nos preguntamos continuamente por todo, o por casi todo.

Así que los ^aprendices^ son capaces de generar comportamientos
^novedosos^, porque evalúan las soluciones posibles, pero no pueden
^diseñar^ herramientas nuevas, porque no evalúan las posibles
resoluciones, y una ^herramienta^ es un medio de ^resolución^. Sólo los
^sujetos^ podemos diseñar de dentro a fuera, esto es, solamente
nosotros, por ser simbólicos, podemos convertir nuestras ideas en
herramientas. Los ^conocedores^ se encuentran en una posición
intermedia, ya que pueden diseñar por tanteo, pero son incapaces de
teorizar, porque no pueden ver introspectivamente ni el problema que
afrontan ni sus posibles resoluciones.

Las diferencias entre los ^aprendices^ y los ^sujetos^ son nítidas. Los
aprendices no alcanzan el ^problema^ en ninguna de sus variedades: ni
como ^pregunta^, ni como ^cuestión^, ni como ^duda^, ni como ^enigma^.
Tampoco alcanzan los aprendices ningún tipo de ^resolución^: ni los
^planes^, ni los ^diseños^, ni los ^algoritmos^, ni los ^razonamientos^.
Y menos aun las ^teorías^, que son entramados de problemas y
resoluciones, con hipótesis, conjeturas y supuestos.

Hay menos diferencias entre los ^conocedores^ y los ^sujetos^,
aunque son rotundas.
Los conocedores disponen de maquinaria de resolución, como los sujetos,
pero su imaginación es semántica, como la de los aprendices,
así que pueden representarse comportamientos y objetos,
pero no problemas ni resoluciones.
Por esta razón, ante un problema nuevo, el conocedor es incapaz 
de prever cuál será el resultado de cada manera de resolverlo,
y ha de elegir por tanteo el modo de resolución.
Cuando consigue solucionarlo así, incorpora heurísticamente la
^resolución^ utilizada al conjunto de sus conocimientos,
y de aquí su nombre de ^conocedor^.

En el ^conocedor^, la resolución de problemas no es racional,
sino emocional, ya que está gobernada por sus ^emociones^. 
Cuando se enfrenta a un problema se siente ^perplejo^,
y si lo soluciona ^feliz^,
pero el tránsito de uno a otro estado no es razonado,
porque no puede interiorizarlo.
Sólo los ^sujetos^, por disponer de un simbolismo,
podemos resolver los problemas ^teóricamente^,
esto es, sin enfrentarnos a ellos.


\Section La conjuntivitis

Como le decía antes de la excursión comparativa, el ^problema de la
supervivencia^ es externo, y nosotros los sujetos únicamente tenemos
acceso a nuestro modelo de él, que es lo que hemos llamado el ^problema
del sujeto^. De modo que, como el \latin{^Noumenon^} de \[Kant], el
problema de la supervivencia queda fuera de nuestro alcance. Nunca
podremos entender qué es la ^vida^, porque es justo al revés: entendemos
para vivir.

Aun así, hay una obviedad que puede darnos una pista sobre una
diferencia que podría existir entre el problema de la supervivencia y el
problema del sujeto: nuestro ^conocimiento^ es aditivo; y ahora no digo
que sea adictivo. Por ejemplo, cuando conocemos una nueva ciudad no
olvidamos otra, sino que simplemente añadimos la nueva ^información^ al
conjunto de nuestros conocimientos previos. A veces olvidamos, claro,
pero muchas veces no es que hayamos olvidado para siempre, sino que no
somos capaces de recordar. Y, si nos preocupa tanto que alguien se
olvide por completo de lo que hizo anteayer, es porque lo sano es no
^olvidar^. La _cuestión<aporía> es, ¿por qué no olvidamos?

Cada nuevo dato consciente que obtenemos lo añadimos a lo que ya
sabíamos, y por esto solemos referirnos al conjunto de nuestros
conocimientos. Esto es literal y técnicamente exacto porque, para
incorporar cada nuevo dato, hacemos precisamente la ^conjunción^ lógica
de los conocimientos que ya teníamos con la nueva información. No
entraré en disquisiciones sobre las propiedades de la conjunción lógica,
cuyo origen vulgar es la ^conjunción copulativa^ gramatical `y', pero
que sea conmutativa y asociativa facilita el tratamiento independiente
de cada elemento. El tratamiento separado de cada dato simplifica las
tareas cognitivas del sujeto.

Esto último es especialmente revelador. Otra vez encontramos una razón
de ^economía^ cognitiva detrás de una obviedad; antes con los objetos y
ahora con la adición de conocimiento. El efecto es, seguramente, el
mismo en ambos casos, a saber, que para simplificar distorsionamos y
troceamos. Así que hemos de ser suspicaces con las operaciones lógicas
---^negación^, ^disyunción^ y ^conjunción^. Padecemos de ^conjuntivitis^
cognitiva.

Una curiosidad gramatical antes de seguir, para que la medite: el ^punto
ortográfico^ debería ser considerado como una conjunción copulativa más.
Los ^niños^, durante esa fase en la que comienzan cada oración con un
`y', lo saben. Y después, por vagancia, nos ahorramos decir la `y';
usted y yo, también.


\MTbeginchar(1.25cm,4cm,0pt);
 \MT: save u, v, ow; u = h/5; v = h - u;
 \MT: color col; col = 0.5white;
 \MT: z1a = (0,0); z2a = (w,0); z3a = (w,v/3);
 \MT: z1b = (0, v/6); z2b = (w,v/2); z3b = (w,5v/6); z4b = (0,v/2);
 \MT: z1c = (0,2v/3); z2c = (w,v); z3c = (0,v);
 \MT: fill z1a -- z2a -- z3a -- cycle withcolor col;
 \MT: fill z1b -- z2b -- z3b -- z4b -- cycle withcolor col;
 \MT: fill z1c -- z2c -- z3c -- cycle withcolor col;
 \MT: z0 = (w/2,h-(u/2)); % ow = 15;
 \MT: z91 = (-10.5cm+w,0);
 \MT: z92 = (-10.5cm+w,2y0);
 \MT: pickup pencircle scaled 3pt;
 \MT: draw z91 -- z0 -- z92 withcolor 0.75white;
 \MT: unfill fullcircle scaled (w/2) shifted z0;
 \MT: y3d = y4d = y0; x3d - x4d = 5w/8; x3d + x4d = w;
 \MT: x1d = x4d; x2d = x3d; y1d = y2d; y1d = v;
 \MT: fill z1d -- z2d -- z3d -- z4d -- cycle withcolor col;
 \MT: pickup pencircle scaled 2pt;
 \MT: draw fullcircle scaled (w/2-1pt) shifted z0 withcolor col;
 %\MT: draw (0,0)--(w,0)--(w,h)--(0,h)--cycle;
\MTendchar;

\Section La torre de Hércules

Cada información es tratada, pues, como una ^condición^ adicional del
 \vadjust{\hbox to 0pt{%
  \kern9.25cm % 10.5cm - wd
  \vbox to 0pt{\kern9.5pc\box\MTbox\vss}%
  \hss}}%
^problema del sujeto^ que, según las circunstancias, habrá de
considerarse, o no, al buscar una ^resolución^. Suponga, por ejemplo,
que viene usted por primera vez a ^La Coruña^, y descubre que aquí
construyeron un ^faro^ los romanos que aún sigue en pie. Este
^conocimiento^ se añade conjuntivamente a su conocimiento del mundo, de
manera que ahora, para usted, las cosas son tal como eran {\em y},
además, los romanos edificaron en La Coruña un faro que aún se conserva.
En principio, debería ser invalidada cualquier ^solución^ que usted
encontrara y que contradijera este dato.

La agregación de condiciones por conjunción tiene el efecto de ir
disminuyendo el número de posibilidades de satisfacer el conjunto. Me
explico. Suponga que este año quiere veranear en una ciudad con ^playa^
y gallega. Hay varias soluciones a su problema. Suponga que, por algún
motivo, decide más tarde añadir una condición adicional, de manera que
lo que desea es veranear en una ciudad con playa y gallega {\em y} con
un faro histórico. La nueva condición elimina todas las ciudades sin
faro histórico, por lo que sus opciones quedan muy reducidas. No le va a
quedar otro remedio que venirse a La Coruña a contemplar la ^torre de
Hércules^. Por cierto, etimológicamente `La Coruña' viene de `la
columna', de Hércules, claro.{\advance\hsize by -1.8cm \par}

Disminuir los grados de libertad de un ^problema^ puede facilitar su
^resolución^. Por ejemplo, si sólo queda una posibilidad, entonces hemos
dado con la ^solución^. Pero también puede suceder que no quede
posibilidad alguna de satisfacer la condición conjunta, y entonces nos
encontramos en una situación paradójica. Si, volviendo a su veraneo,
usted insistiese en elegir una ciudad con playa y gallega y con un
templo griego, entonces sería incapaz de satisfacer su deseo.

El ejemplo es inocuo, pero la mengua de ^libertad^ por la adición
conjuntiva de condiciones al ^problema del sujeto^ es peligrosa. Piense
que, cada nueva información que recibe, cada ^dato^ que obtiene, se
convierte en una ^condición^ adicional que ha de ser satisfecha en todas
y cada una de sus soluciones. Si quiere usted mantener su coherencia y
su consistencia lógica, ha de componerse un modelo del mundo que
satisfaga todos y cada uno de los datos que ha ido obteniendo a lo largo
de su vida. Esto es imposible, y por ello todos soportamos nuestras
contradicciones; hasta cierto punto.


\Section El análisis

Llamaremos analitismo, de ^análisis^, a esta manera de construir el
^problema del sujeto^ que consiste en ir añadiendo condiciones por
^conjunción^. El analitismo se corresponde en la ^sintaxis^ con el
^objetivismo^ en la ^semántica^, ya que ambos implementan la estrategia
del `divide y vencerás' para economizar recursos computacionales. Ambos
son suficientemente efectivos para mantenernos vivos, y aun más, pero
trocean y distorsionan para alcanzar sus objetivos. No son esenciales en
el diseño de los sujetos, pero son parte del diseño concreto de nosotros
los sujetos humanos, que los heredamos genéticamente. A nosotros nos es
imposible ^ver^ de otro modo, es decir, nuestra visión perceptiva es
objetiva y nuestra visión introspectiva es analítica.
$$\vbox{\halign{\strut\hss#\enspace$\cdot$\enspace&
                \hss#\hss& \enspace$\cdot$\enspace #\hss\crcr
 Habla& Análisis& Sintaxis\cr
 Percepción& Objeto& Semántica\cr
}}$$

Yo le veo otra virtud a esta estrategia analítica, que consiste en
ampliar conjuntivamente el conocimiento. Porque, la ^resolución^ que
soluciona el último ^problema^, con todas las condiciones conjuntadas,
también soluciona los anteriores problemas. ¿No lo ve? Volvamos al
supuesto de que este año quiere usted veranear en una ciudad con ^playa^
y en ^Galicia^ {\em y} con un ^faro^ histórico. Como la nueva condición
histórica se añadió por conjunción, resulta que las soluciones del nuevo
problema, y en concreto ^La Coruña^, también solucionan necesariamente
el problema anterior. Vaya tontería, ¿no? Sí, pero no, es decir, la
jerga obscurece lo obvio.

Ahora suponga, por un momento, que alguna de las condiciones resultara
irrelevante y veamos qué sucede. Con la estrategia acumulativa, aunque
despreciemos una ^condición^, la ^solución^ sigue siendo válida. Por
ejemplo, aunque finalmente le resulte indiferente que haya o no faros
históricos, su viaje a La Coruña sigue cumpliendo sus otras condiciones,
por lo que no ha perdido su tiempo ni ha de reconsiderar su decisión. El
ejemplo es, otra vez, inocuo, pero si el problema es el ^problema de la
supervivencia^ y nuestros conocimientos son necesariamente
inconsistentes, como lo son, entonces encontrar soluciones robustas, que
sean válidas incluso aunque no se sostengan todas las condiciones, es
una ventaja decisiva.

Otra posibilidad es que el ^inquisidor^, en lugar de añadir
conjuntivamente las nuevas informaciones como condiciones adicionales
del ^problema del sujeto^, lo rehiciera completamente cada vez. Esta
posibilidad es peor por, al menos, dos razones. Una es que la cantidad
de proceso computacional que es necesario para replantear completamente
un problema es muchísimo mayor que la que se requiere para añadir
simplemente una nueva condición. La otra razón es que, redefiniendo cada
vez el problema, podría ocurrir que la última resolución, que soluciona
el último problema, no solucionara los anteriores. Es normal que no lo
vea, y no me pida un ejemplo, porque no se lo puedo proporcionar: es tan
difícil evitar analizar lo que se dice o piensa, como evitar ver los
objetos. Son dos imposibilidades para los sujetos humanos; para usted y
para mi, también.


\Section La historia

Vamos añadiendo conjuntivamente condiciones que han de ser recordadas,
claro, pero esto no termina de responder completamente a la pregunta que
nos hacíamos: ¿por qué no olvidamos? La respuesta definitiva hay que
buscarla en la ley de la información incesante. Para ello vamos a
proceder por reducción al absurdo.

Si existiera una ^teoría deductiva^ perfecta capaz de predecir sin errar
todo cuanto ocurre, entonces sería ocioso retener todos los sucesos que
van acaeciendo. Es fácil percatarse de que, en esas felices
circunstancias, no valdría la pena retener la infinidad de pormenores
concretos que van sucediéndose, porque éstos podrían ser deducidos de la
teoría en cualquier momento. Pero nosotros sí que memorizamos lo que nos
va ocurriendo, llegando a almacenar enormes cantidades de datos, y esto
puede ocurrir por dos razones. Una es que la premisa no se sostenga, lo
que significaría que no existe tal teoría perfecta. La otra posibilidad
es que, simplemente, nuestro diseño no sea el óptimo, de manera que,
aunque exista tal teoría, no somos capaces de emplearla. Aun aceptando
que nuestro diseño no es el óptimo, a mi me parece que la evolución
nunca desaprovecharía una ocasión tan clara de economizar tantos
recursos cognitivos.

Por otro lado, si es cierto que somos libres, entonces es imposible que
exista una teoría capaz de predecir perfectamente todo cuanto sucederá
en el futuro. En ese mismo supuesto, tampoco lo ocurrido en el pasado es
reducible a una ley no estadística, ya que en el origen de algunos de
los sucesos estuvo el libre albedrío de los sujetos. De manera que, en
estas circunstancias, la historia es una narración de sucesos que no
puede encorsetarse en una teoría deductiva; la ^historia^ no es una
teoría deductiva, es una ^teoría narrativa^. Luego, si esto es así como
lo digo, y somos efectivamente libres, no queda otro remedio que
memorizar los acontecimientos que se van sucediendo.


\Section Las matemáticas

Yo aun diría más; y lo digo. Las teorías deductivas son enormemente
económicas porque son capaces de producir mucha ^información^ a partir
de muy poca. Buscamos teorías deductivas porque ^comprimen^ la
información, y así nos ayudan a luchar contra la ^ley de la información
incesante^ que amenaza con anegarnos. De modo que la ^economía^
cognitiva también explica por qué buscamos con ahínco teorías
deductivas.

El caso arquetípico de teoría deductiva es el ^sistema axiomático^
ligado al ideal formalista de \[Hilbert]. \[Hilbert] propuso en 1920 un
programa cuyo objetivo era reducir todas las ^matemáticas^ a un sistema
axiomático capaz de producir por deducción todas las verdades
matemáticas. De haberse culminado con éxito el programa, el sistema
axiomático resultante habría representado una desmedida condensación de
información, y habría sido el compendio sumarísimo de todas las
matemáticas. Como ya le he contado, \[Gödel] en 1931 y en 1936 \[Church]
y \[Turing] demostraron matemáticamente que es imposible alcanzar el
ideal formalista, porque no existe ningún sistema axiomático consistente
y capaz de sintetizar todas las matemáticas.

Que la ^historia^ no quepa en una teoría deductiva puede no extrañar a
muchos, pero que ni siquiera quepan las ^matemáticas^, tendría que haber
escandalizado a todos. Mi interpretación de estos hechos es que tampoco
las matemáticas se libran de la inexorable ^ley de la información
incesante^. Los conocimientos matemáticos son, como todos los
conocimientos, subjetivos. Lo que distingue al ^conocimiento matemático^
es su naturaleza algorítmica, siendo un ^algoritmo^ un método de
^resolución^. Es decir, que el propósito de las matemáticas es producir
resoluciones, y éstas no pueden ser refutadas directamente por una
^medida^, como sí que lo pueden ser las soluciones físicas. Las
matemáticas son puramente sintácticas, o sea, teóricas.

Por ejemplo, el ^algoritmo^ de la ^suma^ nos permite enunciar la
siguiente ^verdad matemática^: uno más uno son dos; resumida en la
expresión $1+1=2$. Pero, cuidado, porque $1+1=2$ es una verdad
algorítmica que no siempre es aplicable a la ^realidad^. Mi ejemplo
predilecto es éste: si usted suma una gota de agua a otra gota de agua,
no obtiene dos gotas de agua, sino una; resumido $1+1=1$. Por lo tanto,
$1+1=2$ es una verdad teórica que nos resuelve algunos problemas, pero
no todos, y es tarea nuestra determinar cuando se puede aplicar y cuando
no. La conclusión es que el propio algoritmo de la suma es útil con
mucha frecuencia, pero no siempre.

No es posible rechazar una teoría matemática confrontándola contra
la realidad; la confrontación nos puede mostrar, eso sí, su grado de
utilidad. Pero esto no suele interesar demasiado a los matemáticos, que,
por otro lado, están de suerte, porque la demostración de \[Gödel] les
asegura que nunca van a quedarse sin trabajo.


\Section La física

Para que no le queden dudas sobre el alcance de los teoremas de nuestros
héroes \[Gödel], \[Church] y \[Turing], debe saber que incluyen a
cualquier teoría que utilice las operaciones aritméticas ---^suma^,
^resta^, ^multiplicación^ y ^división^. Es decir, que no afecta
solamente a las ^matemáticas^, sino que la más leve de las
cuantificaciones hace que una teoría sea presa de la godelización.

Ninguna teoría que incluya la ^aritmética^ puede ser completa y
consistente, y en cualquier teoría que incluya la aritmética existen
paradojas indetectables con los medios de la propia teoría. Repito que
estoy simplemente enunciando los resultados de teoremas matemáticos
probados antes del año 1937.

La ^física^ queda plenamente afectada por estos ^teoremas de
indecidibilidad^. Sin embargo, todavía hoy en el año 2005 hay muchos
físicos, como \[Hawking], que no aceptan estas conclusiones y 
buscan una teoría unificada completa y consistente que abarque todos
los fenómenos físicos. 
Tal vez se pregunte usted por qué, aunque ya debería saberlo.
Son materialistas que piensan que las ^leyes^ físicas son propiedades
de la materia y no construcciones teóricas. 
Para ellos la física no es una ^teoría^, sino la mera ^realidad^.
Según esta concepción, las leyes físicas serían una
realidad elusiva que ni se percibe ni podría percibirse, y aun así
objetiva, absoluta y alcanzable. Pero a mi me parece una extensión
excesiva del concepto de realidad que el ^materialismo^ se ve forzado
a admitir para salvar la física, que es su más preciado tesoro.


\endinput
