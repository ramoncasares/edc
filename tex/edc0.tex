% edc0.tex (RMCG200050707)

% Front matter

\pdfcode
 \begingroup%\corkfont
  \pdfinfo{
   /Author (© Ramón Casares 2010)
   /Title (El doble compresor)
   /Subject (La teoría de la información)
   /Creator (Licencia http://creativecommons.org/licenses/by-sa/3.0/)
   /Keywords (filosofía epistemología problema
              sujeto objeto información)}
 \endgroup
\pdfendcode

\shipout\vbox{\background
\pdfWhite
\hbox{}
\vskip2in
\centerline{\ptitlefont El doble}
\vskip1pc
\centerline{\ptitlefont compresor}
\vskip10pc
\centerline{\psubtitlefont La teoría de la}
\vskip6pt
\centerline{\psubtitlefont información}
\vskip2pc
\centerline{\pauthorfont Ramón Casares}
\pdfBlack
}

% [1] Autor y título

\centerline{\fontone Ramón Casares}
\vskip1pc
\hrule height 1pt
\vskip1.5pc
\centerline{\fontzero El doble compresor}

\vfill\break % [2] Colección (en blanco)
\advance\pageno1

% [3] Autor, Título, Subtítulo

\centerline{\fontone Ramón Casares}
\vskip1pc
\hrule height 1pt
\vskip1.5pc
\centerline{\fontzero El doble compresor}
\vskip1pc
\centerline{\fontone La teoría de la información}

\vfill\break % [4] Créditos

\null

\def\URI#1#2{\leavevmode\pdfcode \pdfstartlink
  attr {/Border [0 0 0]}
  user {/Subtype /Link /A << /Type /Action /S /URI /URI (#2) >>}%
 \pdfendcode{\tt#1}\pdfcode\pdfendlink\pdfendcode}

\vfill
 {\sl El doble compresor}\par
 {\sl La teoría de la información}\par
 1ª edición (\todayiso)\par
 \null\par
 Publicado por \URI{www.ramoncasares.com}{http://www.ramoncasares.com}\par
 \copyright\ Ramón Casares, 2010\par
 Este libro queda liberado conforme a los términos de la licencia\par
 \ccbysa\ {\sf Creative Commons Attribution-ShareAlike 3.0},\par
 \URI{http://creativecommons.org/licenses/by-sa/3.0/}%
     {http://creativecommons.org/licenses/by-sa/3.0/}\par
 \null\par
 ISBN-13: 978-1-4536-0915-6\par
 ISBN-10: 1-4536-0915-6\par

\break % [5] Dedicatoria

\null
\vskip5pc
\begingroup\it
 % Dedicatoria
 \rightline{A mis hijos Ramón e Inés}
\vskip3pc
 % Agradecimientos
 \rightline{Valentín Fernández Vidal}
 \rightline{sabe de este libro}
 \rightline{más de lo que él mismo cree.}
 \rightline{Muchas gracias.}
 \rightline{R.C.}
\endgroup
\vfil
\break

\null\vfill\break % [6] Blanco

\endinput
