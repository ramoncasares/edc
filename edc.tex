% edc.tex (RMCG20040427)

%\mag1200 % OPTIONAL MAGNIFICATION

\input explain
\input fonts
\input index
\input metatex

%\input crops % OPTIONAL CROPS (Beware the next line!)
\newdimen\hpage \newdimen\vpage % already defined in crops.tex

\def\[{\string\[} % to allow ^�lgebra de \[Boole]^
\files

% Fonts

\cmfonts
\font\frakx=eufm10
\font\frakvii=eufm7
\newfam\frakfam \textfont\frakfam=\frakx \scriptfont\frakfam=\frakvii
\def\frak{\fam\frakfam\frakx}

\font\logo=logo10

\pdfcode
 \pdfmapline{+SkakNew-DiagramT SkakNew-DiagramT <SkakNew-DiagramT.pfb}
\pdfendcode

\catcode`\�=13 \def�{\"o} % G�del, Schr�dinger
\toks0=\expandafter{\doaccents\do^^f6}\edef\doaccents{\the\toks0}

\catcode`\@=11

% Layout

\let\docinfo\relax \let\infodoc\relax

\headline={\hfil}
\footline={\tenrm\ifodd\pageno \docinfo\hfil\folio
            \else \folio\hfil\infodoc \fi\strut}

\hpage=6in \hsize=10.5cm
 \hoffset\hpage \advance\hoffset-\hsize \divide\hoffset2
 \advance\hoffset-1in
\vpage=9in \vsize=39\baselineskip \advance\vsize\topskip % 39 + 1 lines
 \voffset\vpage \advance\voffset-\vsize \advance\voffset-2\baselineskip
 \divide\voffset2 \advance\voffset-1in

\pdfcode \pdfpageheight=\vpage \pdfpagewidth=\hpage \pdfendcode
%\setcrops % OPTIONAL (Beware the top of this file!)

\parskip=0pt plus 0.0001pt minus 0.0001pt
\parindent=20pt 
\raggedbottom

\def\breakif#1{\vskip#1\baselineskip \penalty-250 \vskip-#1\baselineskip}

%\colors      % OPTIONAL COLORS
\def\RED{\ifx\Yellow\relax\else
 \hbox to0pt{\Yellow\vrule width20pt height8pt depth2pt\Black\hss}\fi}

% Grid

%\tracingpages=1 % to check that the lines are on the grid

\abovedisplayskip=12pt
\belowdisplayskip=12pt
\abovedisplayshortskip=0pt
\belowdisplayshortskip=12pt

\newbox\displaybox
\newdimen\displayht
\newdimen\displaydp

% if \baselineskip is 12pt,
% the strut height is 8.5pt     17/24 = 0.7083..3
% and its depth is 3.5pt.        7/24 = 0.2916..6

% It doesn't work with alignment displays (The {\TeX}book, pg 190)

\def\griddisplay#1$${%
 \setbox\displaybox=\hbox{$\displaystyle{#1}$}%
 \displayht=\ht\displaybox \displaydp=\dp\displaybox
 \advance\displayht by \lineskiplimit
 \advance\displayht by 0.75\baselineskip
 \divide\displayht by \baselineskip
 \multiply\displayht by \baselineskip
 \advance\displaydp by \lineskiplimit
 \advance\displaydp by 0.25\baselineskip
 \divide\displaydp by \baselineskip
 \multiply\displaydp by \baselineskip
 \ht\displaybox=0pt \dp\displaybox=0pt
 \baselineskip=\displayht
 \box\displaybox
 \ifdim\displaydp=0pt\else \vadjust{\kern\displaydp}\fi$$}
\everydisplay{\griddisplay}

\def\nodepth#1{\setbox0=\hbox{#1}\dp0=0pt\box0}

% Maths

\def\eqalign#1{\vbox{\openup1\jot \m@th
 \ialign{\strut\hfil$\displaystyle{##}$&$\displaystyle{{}##}$\hfil
 \crcr#1\crcr}}}

\def\inmmode$#1${\ifmmode #1\else $#1$\fi}
\def\llave#1{\inmmode$\left\lbrace\vcenter{\normalbaselines \m@th
 \halign{&\hbox{\rm##}\hfil\crcr#1\crcr}}\right.$} %$

\def\frac(#1/#2){\leavevmode % The TeXbook, exercise 11.6, page 311
 \raise.5ex\hbox{\the\scriptfont0 #1}\kern-.1em
 /\kern-.15em\lower.25ex\hbox{\the\scriptfont0 #2}}

\let\implies=\Longrightarrow
\let\evalsto=\longrightarrow

% Emphasis with automatic italic correction (\/).
% Use: {\em italic, but {\em roman}, text}.
\def\em{\ifdim \fontdimen1\font>0pt \rm
 \else \it \expandafter\aftergroup \fi \itcor}
\def\itcor{\ifhmode \expandafter\itpuncl@ok \fi}
\def\itpuncl@ok{\begingroup\futurelet\ITCt@mpa\itcort@st}
\def\itcort@st{\def\ITCt@mpb{\ITCt@mpa}%
 \ifcat\noexpand\ITCt@mpa,\setbox0=\hbox{\ITCt@mpb}%
  \ifdim\ht0<0.3ex \let\itc@rdo=\endgroup \fi\fi \itc@rdo}
\def\itc@rdo{\skip0=\lastskip \ifdim\skip0=0pt \/\else
 \unskip \/\hskip\skip0 \fi \endgroup}

% Sectioning

\def\Title<#1>{\vfill\break\pdflabel\toc0{#1}\vglue36pt % 10+12+12 no
 \begincenter \fontzero\baselineskip24pt #1\endcenter
 \nobreak\vskip0pt\relax}
\def\Subtitle<#1>{\par
 \begincenter \fonttwo\baselineskip24pt #1\endcenter
 \nobreak\vskip0pt\relax}
\def\Author<#1>{\par
 \begincenter \fontone\baselineskip24pt #1\endcenter
 \nobreak\vskip12pt\relax}

\newcount\secno \secno=0
\newcount\parno \parno=0

\def\section{\number\secno}

\font\fonttwosym=cmbsy10 scaled\magstep1
\def\Stitle{{\fonttwosym \char"78}} %"

\def\Section#1 \par{\everypar{}\parno=0  \advance\secno1 
 \vskip0pt plus 4\baselineskip\penalty-43
 \vskip0pt plus-4\baselineskip \vskip\baselineskip
 \titleline{\noindent\Stitle\fonttwo\section\space}%
   {\fonttwo#1\pdflabel\toc1{#1}\lbl{#1}{\section}}%
 \everypar{\numberedpars}\nobreak}

\def\titleline#1#2{\setbox0\hbox{#1}\dimen0=\hsize \advance\dimen0 -\wd0
 \line{\box0\hss\vtop{\hsize=\dimen0 \raggedright\let\\=\ \noindent #2}}}

\newdimen\oldparindent \oldparindent=\parindent \parindent=0pt

\def\numberedpars{\global\advance\parno1 \pdflabel
 \noindent\hbox to\oldparindent{{\sevensy\char"7B %" little par sign
  \teni\number\parno}\hfil$\cdot$\hfil}\ignorespaces}

% Index

\def\latin#1{{\it #1}\itcor}
\def\point{\subitem*$\bullet$*}

\def\[#1]{\leavevmode\ndx{#1}2\penalty\@M\hskip\z@skip{\sc #1}} % \[Descartes]
\def\(#1){{\sl#1}\itcor} % \(On Freedom)

%\def\<#1>{} % \<label>
%\def\>#1>{} % en la \S6, p�gina 34

\def\specialhat{\ifmmode\def\next{^}\else\let\next=\beginvref\fi\next}
\def\specialund{\ifmmode\def\next{_}\else\let\next=\beginhref\fi\next}
\catcode`\^=13 \let^\specialhat \catcode`\_=13 \let_\specialund

\def\index#1#2{\leavevmode\ndx{#1}{#2}\RED\penalty\@M\hskip\z@skip
 \ignorespaces}

\def\beginvref#1^{{\def\[{\string\[}\index{#1}{1}}#1} % ^word^, ^�lgebra de \[Boole]^
\def\beginhref#1<#2>{{\def\[{\string\[}\index{#1=#2}{1}}#1} % _word<sub>

\def\hname{[}\def\hbook{(}\def\hlabel{<}\def\href{>}
\def\|#1#2|{\def\1{#1}% \|+informaci�n| #1 is a code
 \ifx\1\hname [#2]\else
 \ifx\1\hbook (#2)\else
 \ifx\1\hlabel $\langle$#2$\rangle$\else 
 \ifx\1\href #2$>$\else %$
 \errmesage{Unknown code #1}\fi\fi\fi\fi}

% Pdf Watermark
\newtoks\oldheadline \oldheadline=\headline
\pdfcode
 \def\watermark{% 
  \pdfliteral{q 1 G 1 g 169 -32 m}%% �
  \pdfliteral{82 -97 109 -111 110 -32 c}% Ramon
  \pdfliteral{67 -97 115 -97 114 -101 c}% Casare
  \pdfliteral{115 -32 50 -48 49 -48 c}% % s 2010
  \pdfliteral{b Q}%
 }
 %\headline={\watermark\the\oldheadline } %%%%%%%%%%%%%%%%%%%%%%%
\pdfendcode

\catcode`\@=12

%%%%%%%%%%%%%%%%%%%%%%%%%%%%%%%%%%%%%%%%%%%%%%%%%%%%%%%%%%%%%%%%%%%%%%%

\input edc0.tex

\Title<El doble compresor>
\Subtitle<La teor�a de la informaci�n>
\Author<Ram�n Casares>

\input edc1.tex
\input edc2.tex
\input edc3.tex
\input edc4.tex
\input edc5.tex
\input edc6.tex

\vskip1pc
\centerline{\bf Fin}


\vfill\break %%%%%%%%%%%%%%%%%%%%%%%%%%%%%%%%%%%%%%%%%%%%%%%%%%%%%%%%%%

\dimen0=\topskip \advance\dimen0 by2\baselineskip
\vbox to\dimen0{\kern\baselineskip
 \centerline{\fontone �ndice alfab�tico\pdflabel
                \toc3{�ndice alfab�tico}}\vss}
\vskip 2\baselineskip

%Texto explicatorio, si es preciso.

\catcode`\@=11 \catcode`\^=7
\everypar={}\normalbottom
\def\[#1]{{\sc #1}} % \[Descartes]
\def\vea{{\it vea\/}}
\def\yvea{{\it vea tambi�n\/}}

% Output for INDEX adapted from [417]

\newdimen\fullhsize \fullhsize=\hsize
\newdimen\fullvsize \fullvsize=\vsize
\def\fullline{\hbox to\fullhsize}

\newdimen\gutter \gutter=1pc
\newbox\partialpage
\def\begindoublecolumns{\begingroup
 \output={\global\setbox\partialpage=\vbox{\unvbox255\bigskip}}\eject
 \output={\doublecolumnout}%
 \hsize=\fullhsize \advance\hsize by-\gutter \divide\hsize by2
 \vsize=\fullvsize \multiply\vsize by2 \advance\vsize by2pc}
\def\enddoublecolumns{\output={\balancecolumns}\eject
 \endgroup \pagegoal=\vsize}

\def\doublecolumnout{\splittopskip=\topskip \splitmaxdepth=\maxdepth
 \dimen0=\fullvsize \advance\dimen0 by-\ht\partialpage
 \setbox0=\vsplit255 to\dimen0 \setbox2=\vsplit255 to\dimen0
 \shipout\vbox{\vbox to 0pt{\vskip-22.5pt
  \fullline{\vbox to8.5pt{}\the\headline}\vss}\nointerlineskip
  \vbox to\fullvsize{\boxmaxdepth=\maxdepth \pagesofar}
  \baselineskip=24pt \fullline{\the\footline}}\advancepageno
 \unvbox255 \penalty\outputpenalty}
\def\balancecolumns{\setbox0=\vbox{\unvbox255}\dimen0=\ht0
 \advance\dimen0 by\topskip \advance\dimen0 by-\baselineskip
 \divide\dimen0 by2 \splittopskip=\topskip
 {\vbadness=10000 \loop \global\setbox3=\copy0
  \global\setbox1=\vsplit3 to\dimen0
  \ifdim\ht3>\dimen0 \global\advance\dimen0 by1pt \repeat}
 \setbox0=\vbox to\dimen0{\unvbox1}%
 \setbox2=\vbox to\dimen0{\dimen2=\dp3 \unvbox3 \kern-\dimen2 \vfil}%
 \pagesofar}
\def\pagesofar{\unvbox\partialpage
 \wd0=\hsize \wd2=\hsize \fullline{\box0\hfil\box2}}

%%%

\newcount\lastpgno \lastpgno=0
\newcount\thispgno \thispgno=0
\newcount\level \level=0
\newif\ifrange\rangefalse

\def\bibbreak{\par\vskip-2pt plus 1fil \penalty-200 \vskip2pt plus -1fil\relax}
\def\newentry{\par\vskip-2pt plus 1fil \penalty-200 \vskip2pt plus -1fil\hang\noindent}
\def\newsubentry{\par\hang\noindent\kern10pt}

%\ndxline{abogada}{1}{0007}{1}{7}

\def\ndxline#1#2#3#4#5{\let\oldkey=\newkey \def\newkey{#1}\level=#2 %
 \ifnum\level<0 % is not an index enter
  \ifrange --\number\lastpgno \fi \rangefalse
  #1%
 \else
  \ifx\newkey\oldkey \thispgno=#5 %
   \ifnum\lastpgno=\thispgno \else \advance\lastpgno1
    \ifnum\lastpgno=\thispgno \rangetrue \else
     \ifrange \advance\lastpgno-1 --\number\lastpgno\fi
     \lastpgno=#5\rangefalse
     , \pdfgoto{\ifnum\level=3 {\bf#5}\else #5\fi}{#3}\fi\fi
  \else % \newkey<>\oldkey
   \ifrange --\number\lastpgno\fi \rangefalse
   \ifnum\level<10 \newentry \else \newsubentry \fi
   \lastpgno=#5 %
   \ifnum\level>9 \advance\level-10 \fi
   \ifcase\level {\tt\newkey}\or{\rm\newkey}\or{\sc\newkey}\or
                 {\bf\newkey}\or{\it\newkey}\or{\sl\newkey}\else
                 {\Red \newkey\Black}\fi ,\space\space
   \pdfgoto{\ifnum\level=3 {\bf#5}\else #5\fi}{#3}\fi\fi\ignorespaces}

\begindoublecolumns
\parindent=20pt \rightskip=0pt plus 4pc \hyphenpenalty=250
\input auxiliar.abc
 \ifrange --\number\lastpgno \fi \rangefalse
\enddoublecolumns


\vfill\break %%%%%%%%%%%%%%%%%%%%%%%%%%%%%%%%%%%%%%%%%%%%%%%%%%%%%%%%%%

\dimen0=\topskip \advance\dimen0 by2\baselineskip
\vbox to\dimen0{\kern\baselineskip
 \centerline{\fontone �ndice\pdflabel\toc3{�ndice}}\vss}
\vskip 2\baselineskip

\def\tocline#1{\ifcase #1\let\next=\toclinezero \or
 \let\next=\toclineone \else \let\next=\toclineindex \fi \next}

\def\toclinezero#1#2#3#4{\bigbreak \pdftocline{#1}{#2}%
 \line{\bf \pdfgoto{#1}{#2}\hfil}\ignorespaces}
\def\toclineone#1#2#3#4{\par \pdftocline{\noexpand�#3 #1}{#2}%
 \contentsline{#1}{#2}{$\S#3$\quad}{#4}\ignorespaces}
\def\toclineindex#1#2#3#4{\par \pdftocline{#1}{#2}%
 \contentsline{{\bf #1}}{#2}{}{#4}\ignorespaces}

\newdimen\tridig \setbox0=\hbox{$\S123$\quad}\tridig=\wd0

\def\contentsline#1#2#3#4{\setbox2\hbox{#1}%
 \setbox0\hbox to\tridig{\hfil#3}%
 \dimen0=\hsize \advance\dimen0 by -\wd0
 \multiply\dimen0 by 8 \divide\dimen0 by 10
 \ifdim\dimen0>\wd2 \line{\box0 #1\tocleaders \pdfgoto{#4}{#2}}\else
  \line{\box0 \vtop{\hsize=\dimen0 \raggedright \normalbaselines
   \let\\=\ \noindent #1\strut}\tocleaders \pdfgoto{#4}{#2}}\fi}
\def\tocleaders{\leaders\hbox to\baselineskip{\hss\bf.\hss}\hfil}

\input auxiliar.toc

 \vfill % Colof�n

\begincenter \obeylines % \normalbottom \baselineskip=12pt
 Este libro ha sido tipografiado por el autor
 usando el sistema del Profesor D.~E.~Knuth (Stanford University).
 He utilizado su programa {\TeX} para componer mi texto
 con sus tipos de la familia Computer Modern,
 y para colocar mis figuras,
 que hice con su programa {\logo METAFONT}.
\endcenter
\bigskip
\centerline{\hbox{\pdfximage width 4cm {amazon/cc-by-sa-2.pdf}%
                  \pdfrefximage\pdflastximage}}

\let\vfill=\relax
\bye
