% edc3.tex (RMCG200450703)


\Section La mentira

Todos vemos que las manzanas se caen cuando maduran, pero podemos decir
que no se caen. Esta simple observación explica la eficacia de su
^caricatura^ y, a la vez, muestra la {\em potencia} del ^lenguaje^. En
lo dicho cabe lo que es y lo que puede ser, el ^acto^ y la ^potencia^,
pero, solamente lo que es, es verdaderamente. Por eso, en lo que se
dice, hay que discernir lo verdadero de lo falso. El habla excede la
realidad, y por eso hay verdades y falsedades. Sin lenguaje no hay
^mentiras^.

Creo que es fácil caer, otra vez, en la trampa objetivista, así que
intentaré evitarle a usted pasar por tal trance. El engaño se vale ahora
de la verdad.

Retomaremos un experimento anterior. Tome una ^piedra^, colóquela
delante de usted, y dígame qué ve. Puede usted decirme la ^verdad^, o
sea, que ve una piedra, o puede usted mentirme. Como la verdad es que ve
usted una piedra, entonces resulta que hay una piedra ahí fuera
---concluirá rápidamente cualquier objetivista. Pero no es
necesariamente así. Recuerde lo que ya le dije. Aunque no haya piedras
ahí fuera, lo cierto es que, en esa situación, usted, yo, o cualquier
otra persona (vidente), no puede evitar ver la piedra. Por esta precisa
razón, es verdad que usted ve una piedra, aunque fuera no haya piedras.

Basta ya. Todo esto es ridículo. Quiere confundirme con su enredosa
palabrería, pero no lo conseguirá. Si fuera no hay piedras, ¿cómo quiere
que tome una piedra? Si me dice que tome una piedra, es que hay piedras
fuera, y usted lo sabe igual que yo.

No sé quién le habrá dado a usted permiso para escribir el anterior
párrafo en {\em mi} ^libro^, pero está claro que no lo he convencido,
todavía. La cuestión es que, como los dos vemos forzosamente la piedra
del mismo modo, porque no podemos alterar el funcionamiento de nuestro
aparato ^perceptivo^, entonces podemos referirnos sin ambigüedad a la
región del espacio que ambos etiquetamos como `piedra'. Además, es una
manera útil de interactuar entre nosotros, y con el entorno, y por eso
la utilizamos.


\Section Ver es difícil

Por otra parte, ¿qué utilidad adicional obtiene usted suponiendo que la
^piedra^ existe fuera? Es repetirme, pero tal suposición ignora que la
piedra que vemos es el resultado de un proceso enormenente complejo. Le
concedo, de nuevo, que vemos necesaria y forzosamente el ^exterior^ de
ese modo, pero yo creía que ya le había dado razones suficientes para
dudar de lo ^obvio^. Le daré otro motivo de ^duda^, tal vez el último.

Hacer una ^división^ de dos números largos no es excesivamente difícil,
aunque también es fácil equivocarse, ya que para completarla
correctamente hay que realizar sin yerros toda una larga secuencia de
operaciones intermedias. Pero nos es mucho más fácil ver una piedra, y
distinguirla, por ejemplo, de una concha marina, que hacer una división.
Ver lo que está en frente es el arquetipo de lo ^fácil^ y el ideal de lo
obvio.

Pues ésta es otra ^ilusión^. Lo cierto es que ver requiere una mayor
cantidad de procesamiento de información que dividir. Y mayor es poco.
La capacidad de proceso que necesita un artefacto capaz de distinguir
visualmente una piedra de una concha es enormemente mayor que la que
necesita una máquina ^calculadora^ capaz de ^sumar^, ^restar^,
^multiplicar^ y ^dividir^. Como siempre, le invito a que no me crea e
investigue por su cuenta, o, si no se siente capaz, le pregunte a un
^ingeniero^. En último caso, el precio o la fecha de aparición de cada
una de estas dos máquinas debería serle suficiente para hacerse un
juicio.

Lo que nos parece fácil es lo más difícil. Nos parece fácil ver, pero es
difícil. La piedra que vemos es el resultado de un proceso enormemente
complejo que nos resulta fácil. Yo aun diría más, la piedra es el
resultado de un proceso enormemente complejo que, por sernos tan fácil,
ustedes los ^objetivistas^ ignoran ---¿se acuerda del ^restaurante^? O
sea, que yo comprendo su error y, cuando usted también lo comprenda,
entenderá el ^subjetivismo^. De momento, me contentaré con que mis
argumentos le obliguen a desconfiar de lo obvio.


\Section Mi filosofía es obscura

El ^materialismo^ desprecia todo aquello que no puede medirse. Medir es
más general que percibir, porque no se limita a lo que nuestro cuerpo
biológico es capaz de captar, sino que acepta también lo que captan los
aparatos mecánicos. Quiero decir que, aunque sólo vemos la ^luz^, que es
una pequeña parte del espectro radioeléctrico, el materialismo trata por
igual a todas las ondas electromagnéticas del espectro. Podemos decir
sin faltar a la verdad que el materialismo desprecia lo ^imperceptible^,
observando que, si un aparato mecánico puede obtener ciertos datos,
entonces no es imposible pensar que podríamos disponer de un sentido que
también los obtuviese. Los rayos {\sc x} son perceptibles porque, aunque
no podamos percibirlos, tampoco es absurdo pensar que podríamos
percibirlos.

El ^objetivismo^ es una hipótesis sobre el ^exterior^. Postula que el
exterior está constituido por objetos. Un ^objeto^ es un ente que ocupa
un ^espacio^, que perdura en el ^tiempo^, y que tiene ciertas
propiedades. Las propiedades del objeto pueden ir cambiando. Para el
objetivismo, en resumen, el exterior es tal como lo percibimos,
ignorando que ver es ^difícil^ y que la percepción efectúa un
tratamiento muy complejo de los datos captados.

Materialismo y objetivismo toman en serio lo percibido, como si la
^percepción^ fuera un cristal perfectamente transparente que nada oculta
ni deforma. El materialismo postula que solamente existe lo que podemos
ver, y el objetivismo afirma que, además, es tal como lo vemos. La
doctrina que acepta conjuntamente el materialismo y el objetivismo es el
^realismo^ estricto. Y el ^positivismo^ es la transposición directa del
realismo filosófico más estricto a la ^ciencia^. Pues bien, la ciencia
niega la libertad y encuentra absurdas las paradojas cuánticas, porque
la doctrina preponderante en ella es el positivismo.

El realismo es la filosofía obvia. Según el realismo, sólo existe lo
^obvio^, y no se puede dudar de lo obvio. Yo no acepto ninguna de estas
dos propuestas. Opino que hay cosas que no son obvias, y que aun de las
obvias hay que dudar. Por esto mi filosofía es obscura.


\Section Semántica y sintaxis

Para poder meditar con precisión sobre la obscura relación que liga la
^percepción^ con el ^habla^, vamos a distinguir dos capas en el
^lenguaje^: la ^semántica^ y la ^sintaxis^. Olvídese, de momento, de lo
que estas dos palabras le sugieren porque, seguramente, aquí las vamos a
emplear de otro modo. Lo que pretendo es partir el lenguaje por la veta
que separe más nítidamente lo perceptible de lo no perceptible.

Calificaremos como semántico todo aquello que es perceptible, o
asequible a la percepción. Un ^caballo^, ya que podemos verlo, y
palparlo, sería un objeto semántico. El color gris, que también vemos,
es un objeto semántico. La acción de correr es semántica, porque podemos
ver correr a un caballo. Y como podemos percibir si el caballo corre
poco, o mucho, también son objetos semánticos los adverbios `poco' y
`mucho'. Como primera aproximación, son semánticos los sustantivos, los
^adjetivos^, los verbos y los adverbios. Con la semántica podemos
expresar lo que percibimos: `caballo gris correr mucho' ---que diría el
^indio^ de una vieja película del ^oeste^.

La semántica puede representar lo percibido, pero no lo que es
inasequible a la percepción. La ^realidad^ se representa en la
semántica, pero en la semántica no caben las teorías. Podemos describir
la realidad e imaginar teorías porque nuestra habla excede la realidad,
o sea, porque nuestro lenguaje, además de semántico, es sintáctico.

Aunque le parezca que estoy, otra vez, especulando más allá de lo
razonable, si me presta atención, advertirá que sólo estoy vistiendo con
definiciones una de las últimas obviedades: podemos hablar de todo
cuanto vemos, y de más cosas.


\Section El lenguaje semántico

Voy a proponerle ahora otro par de definiciones ---pero no se me asuste,
porque son sencillas. Vamos a llamar ^lenguaje semántico^ a aquél que
solamente dispone de semántica, sin sintaxis, y ^lenguaje simbólico^ al
que está completo, con ^semántica^ y ^sintaxis^. Las lenguas humanas
son, todas ellas, lenguajes simbólicos ---incluido este ^castellano^ que
está usted leyendo.

Todas las palabras de un lenguaje semántico se refieren a objetos
producidos por la ^percepción^. En un lenguaje semántico cada ^palabra^
es el nombre de un objeto producido por la percepción. El nombre del
objeto es un sonido asociado al objeto, o sea, que el nombre es otra de
las propiedades del objeto, como lo es su forma o su sabor. Las
propiedades del objeto sirven para reconocerlo, y por esto, el nombre es
un signo del objeto.

Se puede hacer rabiar fácilmente a un ^niño^ pequeño cambiando el nombre
de algo que el niño conozca bien; por ejemplo, refiérase siempre a su
pelota como `pilita'. Si usted insiste, el niño se enfadará y le dirá
que no es `pilita', sino `pelota'. Es una broma cruel, porque el niño
pequeño habla un lenguaje semántico, y para él las palabras son
propiedades de los objetos, igual que su color, y en modo alguno
convenciones. Pero, a pesar de ser cruel, sigue siendo una broma que nos
hace gracia, porque nosotros no podemos entender la razón por la cual el
niño defiende con tanta convicción el nombre de una cosa que nosotros,
que hablamos un lenguaje simbólico, sabemos que es meramente
convencional. Que el nombre es convencional nos lo demuestra que los
nombres de las cosas son distintos en ^castellano^, en ^inglés^ y en
^japonés^.

Para que una especie con un sistema perceptivo complejo disponga de un
lenguaje semántico, sólo es necesario que pueda añadir al objeto una
propiedad arbitraria, su nombre. Por ejemplo, \[Pavlov] nos mostró que
se puede enseñar a un ^perro^ que el toque de una ^campana^ es una señal
de que va a comer en breve. Basta acostumbrarlo tocando la campana
siempre que se le va a dar de comer. Una vez así instruido, en cuanto el
perro oiga la campanada, segregará saliba. Tras el aprendizaje, el
sonido de la campana pasa a funcionar del mismo modo que el olor de la
comida. Esto nos indica que un perro puede añadir una propiedad
arbitraria a un ^objeto^ semántico.


\Section El lenguaje de los simios

Hay varios experimentos en los que se han enseñado varias decenas de
palabras a ^chimpancés^. A pesar de la propaganda inicial, lo cierto es
que el lenguaje de estos ^monos^ enseñados es completamente agramatical,
es decir, que no son capaces de hacer construcciones sintácticas. Usando
nuestras definiciones, que para eso las hemos hecho, los chimpancés
pueden aprender un ^lenguaje semántico^, pero no un ^lenguaje
simbólico^.

Es interesante observar que, la gran mayoría de las veces, sus
expresiones son peticiones. Y es interesante, porque los datos
anteriores nos describen una situación curiosa que esta observación
desvela. Fíjese que los ^perros^, los chimpancés, y seguramente los
miembros de muchas otras otras especies animales, son capaces de asociar
una propiedad cualquiera a un objeto producido por la ^percepción^.
Pueden así añadir un nombre oído, que es un sonido, al conjunto de las
propiedades que determinan un objeto concreto, y, de esta manera, pueden
después recuperar ese objeto al oir el sonido que le sirve de nombre.
Esto significa que muchas especies tienen una cierta facilidad para
desarrollar un lenguaje semántico. Lo curioso es que ninguna lo
llega a desarrollar completamente.

Sospecho que, para usted, es muy poco útil que le describa la
^actualidad^, que es lo que justo ahora estoy yo percibiendo. Porque, si
usted está a mi lado escuchándome, entonces usted está percibiendo lo
mismo que yo, salvando la diferencia causada por la distinta
perspectiva. Por esto no es muy útil un lenguaje semántico que sólo
permite expresar lo percibido, lo actual, la ^realidad^.

Yo sólo le veo dos usos provechosos, aunque acepto sugerencias. El
lenguaje semántico puede servir para centrar la ^atención^. Si grito
``¡^león^!'', puedo advertir a toda la tribu de la cercanía de un
peligroso felino. La otra utilidad es, precisamente, la explotada por
los ^chimpancés^ enseñados, que usan el ^lenguaje semántico^ casi
exclusivamente para pedir. El lenguaje semántico permite expresar los
^deseos^. Esto es muy conveniente porque sólo yo puedo percibir mis
propios apetitos, que son invisibles para usted.


\Section El sentimiento

No se le escapa a usted que ahora estamos ampliando el concepto de
^percepción^ al incluir las sensaciones del ^cuerpo^ propio. Estamos
equiparando la sensación de ^hambre^, por ejemplo, con la visión de una
^piedra^. Si le molesta este abuso del lenguaje, puede usar el término
técnico de `^propiocepción^' para referirse a la percepción
intracorporal. En cualquier caso, lo que es irrebatible es que `hambre',
aunque no se pueda propiamente ver, es una ^palabra^ con un significado
tremendamente fuerte y definido, por lo que debe figurar, sin duda,
entre las palabras de un ^lenguaje semántico^.

Las palabras que designan sentimientos, como por ejemplo `^dolor^' o
`^felicidad^', son las palabras más semánticas. Son las que tienen unos
significados más profundos y arraigados. Son las más difíciles de
definir, porque no hay manera de perfilarlas mejor usando otras palabras
que todavía tengan más contenido semántico. Son las más fáciles de
^traducir^, porque todos los idiomas tienen palabras para los
significados más profundos, que son los sentimientos humanos.

La propiocepción nos enseña que la ^evolución^ descubrió que la manera
más sencilla de controlar el cuerpo, cuando es complejo, consiste en
aplicar a los datos procedentes del cuerpo los mismos procesos
perceptivos que a los datos proporcionados por los sentidos. Porque una
vez que ya se dispone de un mecanismo para interpretar datos, me refiero
a la ^percepción^ que procesa los datos del ^exterior^, basta reutilizar
ese mismo mecanismo para analizar los datos corporales.

El razonamiento previo se apoya en dos columnas: que la evolución
darviniana, aunque es ciega, es capaz de ^diseñar^, y que la percepción
es evolutivamente anterior a la ^emoción^. Y sostiene una viga: que los
^sentimientos^ son ^objetos^ que resumen información corporal.


\Section La teoría

Se me está abriendo demasiado el frente de ataque, así que dejaré, de
momento, estas últimas conclusiones como puesto avanzado sobre el
territorio enemigo aún sin conquistar, y me dedicaré con tenacidad y
aínco a asentar lo ganado antes.

Parece sensato reclamar que las teorías proporcionen alguna ventaja
visible y no sean meramente posibilistas, o, por lo menos, en caso de
duda y a falta de otro criterio, es razonable preferir la teoría que
produce más beneficios perceptibles, esto es, ^reales^. Las teorías
deben proporcionar algún provecho actual, pero pertenecen al ámbito de
lo ^posible^. Porque las teorías parten de cuestiones que interesa
dilucidar, y van razonando la pertinencia de cada una de las posibles
respuestas. Y como el lenguaje semántico, que es capaz de describir la
^realidad^, no alcanza las preguntas ni las posibilidades, tampoco puede
usarse para expresar razonamientos ni enunciar teorías.

Aunque ha habido intentos de expresar la ciencia materialista en un
^lenguaje fenoménico^, o sea, en lo que aquí estamos llamando ^lenguaje
semántico^, tal empresa está condenada desde el principio al más
ridículo de los fracasos. Para calibrar el desatino de semejante
proyecto, basta percatarse de que un lenguaje semántico no permite la
expresión de preguntas ni de condiciones, porque éstos no son objetos
perceptibles. En un lenguaje semántico no pueden plantearse preguntas,
que utilizan ^pronombres^ interrogativos, ni pueden expresarse las
condiciones, que emplean conjunciones. En fin, imagínese al mítico
^indio^ de las antiguas películas del ^oeste^, aquél del `^caballo^ gris
correr mucho', pero ahora ciñéndose a sustantivos, ^adjetivos^, verbos y
adverbios, y no a cualesquiera de ellos, sino evitando escrupulosamente
toda suerte de sustantivos abstractos y demás palabras no perceptibles;
imagíneselo ---si puede--- escribiendo un tratado de ^física^.

Ocurre que todo el ^futuro^ es ^teoría^; el futuro es mera ^potencia^.
Es más, solo el presente es actual. La realidad es el presente, porque
el pasado ya no es, y el futuro aún no es. Bonitas obviedades que no voy
a contestar, pero que hacen patente la exagerada limitación que se
autoimpone el materialismo.

Para que no le queden dudas en un asunto tan importante, le fijaré el
concepto. Una ^teoría^ es un conjunto de reglas y métodos que sirven
para componer las respuestas a un determinado tipo de preguntas. Si, por
ejemplo, las preguntas se refieren al calor y son cuantitativas,
entonces la termodinámica vale como teoría. Para ^explicar^ una
respuesta, hay que mostrar que la respuesta se confeccionó
^razonadamente^, es decir, que se elaboró de acuerdo a las reglas y
usando los métodos de la teoría. La propia teoría adquiere legitimidad
si sus respuestas son útiles o, en el caso de las respuestas
perceptibles, si se ajustan a la ^realidad^. Una teoría sin estas
pretensiones de legitimidad es meramente imaginaria y, aunque puede
tener un valor artístico, es inútil. Por esto, desde nuestra
privilegiada atalaya epistemológica, sobran las ^imaginaciones^.
Quédese, eso sí, con la definición escueta, que alcanzará todo su
significado algo más adelante: una ^teoría^ es un entramado de
^problemas^ y ^resoluciones^.


\Section El origen del lenguaje

Puede parecer que algunas cuestiones de la máxima importancia, como por
ejemplo el origen evolutivo del ^lenguaje^, nunca podrán ser
completamente esclarecidas, porque no podemos ser testigos de los
acontecimientos cruciales del proceso, ya que acaecieron una vez en
tiempos remotos, y ni se repiten ni pueden replicarse. Por la misma
razón, puede parecer que, si pudiéramos observar tales acontecimientos,
la respuesta a la cuestión sería evidente. A mi, en cambio, no me lo
parece, y me justificaré.

Sustituyamos la cuestión filogenética planteada por su correspondiente
cuestión ontogenética. En palabras llanas, fijémonos en cómo adquieren
el lenguaje los ^niños^, en vez de preguntarnos cómo apareció
evolutivamente el lenguaje. El proceso a explicar ahora es muy similar
al anterior, pero con la ventaja de que tenemos millones de ejemplos del
proceso, e incluso uno vivido por cada uno en primera persona. Pues
bien, a pesar de que podamos ser testigos del proceso, e investigarlo
experimentalmente, seguimos sin entenderlo cabalmente.

Mi conclusión es que en ambos casos nos falta una ^teoría^ que explique
lo que sucede. En el caso del ^origen del lenguaje^, nos faltan también
los datos empíricos, pero, aunque los tuviéramos, tampoco los
comprenderíamos sin una teoría.


\Section La potencia del lenguaje

Aunque nos parezca que la ^realidad^ lo es todo, resulta que los
lenguajes semánticos, que pueden representar absolutamente todo lo real,
tienen una capacidad limitada de representación. Voy ahora a presentarle
un ^lenguaje semántico^ sencillo que utilizaremos para fijar con mayor
precisión la capacidad y las limitaciones de la ^semántica^.

\font\chess=SkakNew-DiagramT scaled1200
\setbox\MTbox=\vbox{\chess
 \offinterlineskip\parskip0pt
 \parindent=40pt\everypar{}\let\\=\par
 \setbox0=\hbox{\chess r}
 \def\={\hbox to\wd0{\hrulefill}}
 \def\-{\hbox to0.4pt{\hrulefill}}
 \let\|=\vrule
 \def\.#1{\leavevmode\raise 2pt\vbox{\hbox to\wd0{\hfil\tt #1\hfil}}}
      \leavevmode\kern\wd0\-\=\=\=\=\=\=\=\=\-\\
      \.8\|rmblkZns\|\\
      \.7\|opo0ZQop\|\\
      \.6\|0Z0o0Z0Z\|\kern20pt \hbox{\tt 1.\ e2-e4\ \ \ e7-e5}\\
      \.5\|Z0a0o0Z0\|\kern20pt \hbox{\tt 2.\ f1-c4\ \ \ f8-c5}\\
      \.4\|0ZBZPZ0Z\|\kern20pt \hbox{\tt 3.\ d1-h5\ \ \ d7-d6}\\
      \.3\|Z0Z0Z0Z0\|\kern20pt \hbox{\tt 4.\ h5-f7++}\\
      \.2\|POPO0OPO\|\\
      \.1\|SNA0J0MR\|\\
      \leavevmode\kern\wd0\-\=\=\=\=\=\=\=\=\-\\ \kern3pt
      \leavevmode\kern\wd0\kern0.4pt\.a\.b\.c\.d\.e\.f\.g\.h\\}

Hay varias maneras de anotar las partidas de ^ajedrez^, pero la más
fácil de explicar es la ^notación algebraica^ completa. Para anotar un
movimiento en esta notación, simplemente se apuntan dos casillas:
aquélla en donde estaba la pieza que se mueve, y aquélla a donde se
mueve la pieza. Y cada una de las sesenta y cuatro casillas del tablero
se identifica por una letra y un número. La letra indica la columna, de
la {\tt a}, que es la que está más a la izquierda del jugador que lleva
las piezas blancas, a la {\tt h}, que es la que está más a la derecha.
El número indica la fila, de la {\tt 1}, que es la fila más cercana al
jugador blanco, a la {\tt 8}, que es la más cercana al jugador negro.
Por ejemplo, el rey blanco al comenzar la partida se encuentra en la
posición {\tt e1}. Un ^mate pastor^ en esta notación queda:
$$\box\MTbox$$

La ^notación algebraica^ completa es semántica porque se limita a
describir lo perceptible, que en el caso del ^ajedrez^ son las jugadas
que se efectúan. No puede, en cambio, anotar problemas ajedrecísticos,
ni expresar teorías sobre la fortaleza de las piezas y de las
posiciones, ni describir los distintos razonamientos que el jugador
sopesa mentalmente para decidir su jugada, y por esto los comentarios de
la partida se escriben en un lenguaje simbólico, como el ^castellano^.

Para describir lo que sucede en una partida de ajedrez no es preciso
anotarlo todo. Por ejemplo, como la posición inicial está completamente
determinada por las reglas del juego, no es necesario anotarla. Así, un
lenguaje semántico mínimo se limitaría a anotar, de cada posibilidad que
conceden las reglas, lo que actualmente ha sucedido. En el caso del
ajedrez, basta anotar las jugadas elegidas sucesivamente por los
jugadores de entre aquéllas que son legales en cada posición. La
notación presentada no es mínima porque permite anotar, además de los
movimientos legales, otros que no lo son. Por ejemplo, el movimiento
{\tt b1-f7} es ilegal en cualquiera de las circunstancias, porque no hay
ninguna pieza que pueda realizarlo, y sin embargo puede expresarse en la
^notación algebraica^ completa.
% Otros movimientos, como el {\tt e2-e4}, son legales, o no, dependiendo
% de la pieza que se encuentre en la posición {\tt e2}; por supuesto, es
% ilegal si no hay ninguna pieza en {\tt e2}.

Como se pueden anotar todos los movimientos posibles de una partida de
^ajedrez^, resulta que esta notación puede representar, no sólo lo que
ocurre actualmente en una partida, sino también lo que podría ocurrir. Y
es que no puede ser de otro modo, cualquier lenguaje que pueda expresar
lo que es, también tiene que ser capaz de expresar lo que podría ser.
Por ejemplo, como podría ser el caso que `^caballo^ gris correr poco',
en vez del verdadero `caballo gris correr mucho', el lenguaje semántico
ha de ser capaz de expresar ambos.

Así que, aunque el ^lenguaje semántico^ sólo pretende representar la
^realidad^, también ha de representar la realidad posible. Esto nos
proporciona dos importantes conclusiones. Una, que el lenguaje es
inseparable de la posibilidad, y de la ^mentira^. Y dos, que la
demarcación entre el ^acto^ y la ^potencia^ no es la que establece la
diferencia de expresividad que existe entre un lenguaje semántico y uno
^simbólico^. Lo siento por el viejo \[Aristóteles], pero hemos de usar
otros medios más precisos para el deslinde.


\Section El vestido

Una _curiosidad<aporía>, ¿por qué nuestra especie es la única que se
viste? ¿Existe alguna relación entre vestirse y hablar? Desde luego,
nuestra especie es la única que habla y también es la única que se
viste. Aunque hay cangrejos ermitaños que usan un envase de hojalata a
modo de ^vestido^, y también podemos enseñar a vestirse a un ^mono^.
Incluso podemos suponer ---ignoro si es cierto o no--- que el mono,
imitando los movimientos de un ^sastre^, es capaz de aprender a cortar
telas y coserlas adecuadamente para componer un traje. Es decir, los
comportamientos del mono enseñado y del sastre podrían ser
indistinguibles. La diferencia es que el sastre puede ^razonar^ sobre el
diseño del vestido y, por ejemplo, prescindir de las mangas y
confeccionar un chaleco porque es más conveniente para afrontar los
calores del futuro verano. El mono puede aprender comportamientos por
imitación, pero no ^diseña^ porque sólo mira hacia fuera.

La afirmación anterior es gratuita si no podemos distinguir si el mono
imita o diseña. La novedad del diseño marca la diferencia. Para diseñar
algo nuevo hay que mirar hacia dentro, porque todo cuanto hay fuera está
ya realizado. Mirando lo que hay fuera podemos ^copiar^ e ^imitar^, pero
no diseñar algo novedoso. Otra obviedad, pero ésta con una conclusión
paradójica: si se mira hacia fuera todo lo que se ve es perceptible, de
manera que para ver lo que no es perceptible hay que mirar hacia dentro.

\breakif8


\Section El diseño

Para ^diseñar^ una máquina que sepa jugar al ^ajedrez^ bastan dos
módulos. Uno, fácil de diseñar, capaz de generar un movimiento, que
puede ser tan simple como elegir a suertes, o de cualquier otro modo,
dos casillas distintas de las sesenta y cuatro del tablero. Otro, menos
fácil, capaz de validar si un movimiento es legal, o no. Ya se imagina
usted como podríamos hacer funcionar esta máquina: hacemos que el
generador proporcione movimientos, hasta que el validador dé con uno que
considere correcto, y ése es el movimiento que juega la máquina. Por
supuesto, esta máquina jugará muy malamente al ajedrez; sabe mover las
piezas, pero nada más.

Si queremos mejorarla, podemos añadirle un evaluador de posiciones, que
es difícil de diseñar. Con este tercer módulo, que proporciona el valor
de cada posición, nuestro ^algoritmo^ se complica un poco, pero no
mucho. El generador y validador han de generar varios movimientos
legales, pongamos diez, entonces el evaluador puntuará cada una de las
diez posiciones resultantes, y finalmente la máquina seleccionará el
movimiento que conduce a la posición que ha obtenido la mejor puntuación
de las diez.

La máquina que acabamos de diseñar puede realizar movimientos
ajedrecísticos inéditos. Ocurre que todos y cada uno de los movimientos
son parte de la ^realidad posible^ que, recordemos, puede ser expresada
en un ^lenguaje semántico^. Un movimiento de ^ajedrez^ nunca jugado y
espectacular, de ^campeón del mundo^, se anota igual que cualquier otro,
como {\tt h5-f7}, y por eso es una posibilidad al alcance de nuestra
máquina. Así que tampoco es la ^novedad^ la marca que señala lo que es
inexpresable en un lenguaje semántico.

Es crucial distinguir dos niveles de ^posibilidad^. Uno es el que
utiliza la máquina al sopesar los distintos movimientos posibles, diez
en el ejemplo. Y otro el que estamos usando nosotros al ^razonar^ sobre
los posibles ^diseños^ de una máquina que juegue al ajedrez. Un
^lenguaje semántico^ permite el primero, porque es realidad posible,
pero no el segundo, porque es ^teoría^.

\begingroup
 \vskip2\baselineskip
 \leftskip=20pt \rightskip=20pt \everypar={}\noindent
 ``Yo, a los cinco años, dibujaba como \[Leonardo da Vinci],
 y me han hecho falta ochenta años para dibujar como un niño.''
 \hfill \[Picasso] (1961)\par
\endgroup

\MTbeginchar(84pt,84pt,0pt);
 \MT: save u; u = w/20;
 %\MT: pickup pencircle scaled 0.4pt;
 %\MT: draw (0,0) -- (w,0) -- (w,h) -- (0,h) -- cycle;
 \MT: pickup pencircle scaled 1pt;
 \MT:  z1 = (8u,20u);
 \MT:  z2 = (3u,12u);
 \MT:  z3 = (10u,7u);
 \MT:  z4 = (15u,12u);
 \MT:  z9 = (9u,19u);
 \MT: draw z1 .. z2 .. z3 .. z4 .. z9; % cara
 \MT:  z11 = (4u,10u); z12 = (0,11u);
 \MT: draw z11 .. z12; % brazo izquierdo
 \MT:  z21 = (15u,12u); z22 = (20u,14u);
 \MT: draw z21 .. z22; % brazo derecho
 \MT:  z31 = (9u,7u); z32 = (8u,0);
 \MT: draw z31 .. z32{dir 260}; % pierna izquierda
 \MT:  z41 = (10u,7u); z42 = (11u,0);
 \MT: draw z41 .. z42{dir 290}; % pierna derecha
 \MT:  z51 = (8u,10u); z52 = (11u,10u);
 \MT: draw z51{dir -30} .. z52{dir 15}; % boca
 \MT: fill fullcircle scaled (2pt) shifted (9u,12u);
 \MT: fill fullcircle scaled (2pt) shifted (11u,12u);
\MTendchar;

\Section Hablar para dibujar

Los primeros ^dibujos^ los hacemos entre los tres y los cuatro años. Me
refiero a dibujos que intentan representar alguna situación. Si se da un
lápiz a un ^niño^ más pequeño también hace rayas, pero que no intentan
representar nada. Lo mismo ocurre si se da el lápiz a un ^mono^ adulto.
Los primeros dibujos suelen ser ^monigotes^ que representan personas
---ya sabe, como cara sirve un círculo con dos puntos dentro que hacen
de ojos y una raya de boca, el cuerpo se omite, y las extremidades son
cuatro rayas que salen del círculo. La _cuestión<aporía> es cómo
explicar estos hechos, ¿por qué aprendemos antes a hablar que a dibujar?
\hangindent=-\wd\MTbox
\hangafter=4
\vadjust{\vbox to 0pt{\vss\hbox to \hsize{\hss\box\MTbox}\kern3pt}}
\par

Quien no vea el interés que puede tener saberlo, quien opine que los
hechos son así, y que no hay que darle más vueltas, ése no es
^filósofo^. Quien no tiene una explicación que él mismo juzga plausible
para un hecho, tiene un agujero en su imagen mental. Pero no se preocupe
usted por él, porque no corre peligro. Como ocurre en el caso del
agujero retinal causado por el ^punto ciego^, somos capaces de tolerar
enormes dosis de incoherencia interna sin apenas percatarnos, y por esta
razón somos tan pocos los filósofos.

Antes de darle la respuesta a la pregunta ---tendrá una contestación
preliminar al final de la sección siguiente---, le explicaré qué interés
puede tener. Supongamos, para ello, que se conoce la verdadera razón por
la que se aprende antes a hablar que a dibujar. Pues esa misma razón nos
permitiría asegurar que, quienes pintaron las ^cuevas de Altamira^ hace
catorce mil años, hablaban. Tal vez esto era esperable, tal vez. Desde
luego sería más interesante la situación si la respuesta confirmara que
hablar y dibujar son dos manifestaciones de un mismo fenómeno, porque
entonces, y dado que sólo se han encontrado pinturas en excavaciones
arqueológicas ocupadas por el \latin{^homo sapiens^}, se demostraría que
somos la única especie que ha hablado, y, por consiguiente, la primera.

Acabo de utilizar ^diacrónicamente^ el concepto de ^especie^, y sólo
tengo una excusa: catorce mil años, o setecientas generaciones, son
evolutivamente un momento. Lo cierto, ^sincrónicamente^, es que somos la
única especie viva que habla y también la única especie viva que dibuja.
Y sólo comenzamos a dibujar cuando aprendemos a hablar. No digo nada
más, pero a mi me parece, por lo menos, sospechoso.


\Section Los primeros dibujos

Los primeros dibujos son listas de objetos. Si le pedimos a un ^niño^
pequeño que nos diga lo que ha dibujado, nos lo contará con total
precisión. ``[Esta es mi] mamá, [este mi] papá, [y esto el] coche'', nos
contestará señalando los tres únicos objetos presentes en el papel. Los
primeros dibujos no tienen perspectiva ni proporción ni ningún otro
rastro de realismo. La colocación obedece a relaciones lógicas o, más
frecuentemente, el dibujo es una simple yuxtaposición de objetos.

Quien se contenta con la explicación fácil, recurre a la torpeza del
niño. Pero esto no explica, por ejemplo, la colocación de los objetos
del dibujo. Porque no es más difícil situarlos de acuerdo, más o menos,
a su ubicación espacial, que de otra manera. Y, sin embargo, el niño
puede preferir colocar a mamá con papá, juntos, aunque ahora no lo
estén. Mi conclusión es que los primeros dibujos reproducen la imagen
mental del niño; ignoran por completo la imagen retinal.

Hay dos métodos fáciles de hacer representaciones gráficas: uno es el de
la ^cámara fotográfica^ y el otro el del niño pequeño. La cámara
reproduce la imagen retinal, el niño la mental. El método difícil es el
de los adultos que dibujamos mal, a causa de la ^interferencia^ que se
produce entre ambas imágenes, pero esto ya lo habíamos dicho.

Me queda cumplir mi promesa y responderle a la pregunta, ¿por qué
aprendemos antes a hablar que a dibujar? Resumamos primero los hechos.
Somos la única ^especie^ que usa un ^lenguaje simbólico^, y la única que
dibuja. Además, comenzamos antes a hablar que a dibujar, y, cuando lo
hacemos, representamos la imagen mental. Yo diría que, lo que sucede, es
que para dibujar hemos de ver los objetos producidos por nuestra propia
percepción, y que sólo al aprender a hablar empezamos a {\em verlos},
y lo digo. El lenguaje simbólico nos da acceso a nuestra propia imagen
mental. ¿No es sorprendente? Pero ---se pregunta usted---, ¿puede ser
cierto?


\Section La introspección

Para no extraviar nuestro rumbo, le haré un mapa de la situación. No
vemos lo que no existe, pero podemos hablar de lo que no existe.
Hablamos de lo que existe y de lo que no existe. He llamado ^semántica^
a la parte del ^habla^ que se refiere a lo ^perceptible^. Lo perceptible
es algo más que lo meramente actual, ya que incluye también la ^realidad
posible^. La realidad posible podría llegar a percibirse si se hiciera
actual. Pero hay otras posibilidades que no son perceptibles, y son
esas, precisamente esas, las que nos diferencian.

Después nos percatamos de que mirando hacia fuera podemos ^imitar^ y
^copiar^, pero no innovar, porque lo percibido es necesariamante algo ya
realizado. Y, sin embargo, tampoco es exactamente la innovación lo que
nos distingue. De esto nos dimos cuenta al reconocer que nuestra máquina
semántica de jugar al ^ajedrez^ puede realizar movimientos inéditos.

La diferencia que buscamos es sutil, pero importantísima, porque es la
que separa a nuestra ^especie^ de las demás. El caso es que ninguna de
las demarcaciones estudiadas hasta aquí nos vale: entre lo actual y lo
potencial la frontera está entre dos partes de lo potencial; y entre lo
existente y lo nuevo la raya separa dos partes de lo ^nuevo^.

Para explicar por qué el habla es anterior al ^dibujo^, dijimos que al
hablar accedemos a nuestra ^imagen mental^. Si esto es verdad, aunque
todavía no se lo he probado, entonces el habla proporciona
^introspección^, o sea, información sobre los procesos cognitivos
propios. Obviamente, ya que la ^percepción^ es el proceso que simplifica
la información recibida por los sentidos desde el ^exterior^, y no desde
el interior, solamente podemos percibir la ^información^ que viene del
exterior. Por lo tanto, lo que alcancemos introspectivamente no será
perceptible.

El ^lenguaje simbólico^ y la ^introspección^ parecen ser la causa y un
efecto de nuestra diferencia, porque, de alguna manera, nos permiten
superar la ^percepción^, e incluso la percepción posible. Por supuesto,
no debería creerse usted mis descabelladas afirmaciones si no le
presento una teoría que le explique de qué manera produce introspección
el mero hecho de hablar un lenguaje simbólico, y por qué un ^lenguaje
semántico^ no es suficiente. Se la presentaré, pero tiene usted que
tener paciencia, porque, como comprenderá, el argumento no es corto ni
fácil.

Va a ser un camino largo y sinuoso, así que no se me despiste. Para no
perderse, tenga siempre presente que lo que buscamos es explicar la
^introspección^; nuestra meta es la ^consciencia^.


\Section Sin deseos no hay problemas

Hay dos acepciones principales del verbo `^ver^'. La canónica es la que
se emplea en la oración `veo una ^piedra^ gris', y la otra en la frase
`ya veo cual es tu ^problema^'. La primera acepción es perceptiva y la
segunda introspectiva. La oración `ya veo cual es tu problema' es
introspectiva porque quien habla ha de ponerse ^empáticamente^ en el
lugar de su interlocutor, y entonces imaginarse a qué dificultad se
enfrenta el otro. Ya dijimos que los problemas no se ven con los ojos;
los problemas no son perceptibles.

Ningún ^lenguaje semántico^ puede expresar los problemas. La ^realidad^
es como es, y en ella no caben los problemas. Si un ^patito^ se ha
extraviado, simplemente se percibe que está lejos de su madre. Sólo
cuando suponemos empáticamente que el patito tiene ciertos deseos y
necesidades interiores, nos imaginamos que el patito tiene un problema.
El problema surge cuando las intenciones chocan contra la realidad. Si
la realidad satisface los deseos, entonces no hay problema.

Las filosofías ^estoicas^ se fundan en la ausencia de apetitos. Dicen
que eliminando el ^deseo^ se eliminan los problemas ---esto es cierto---
y que eliminando los problemas se alcanza la ^felicidad^ ---esto es
falso. La felicidad se consigue al solucionar cada problema. Sin
problemas la ^vida^ es, simplemente, aburrida. Pero no voy a mofarme de
los estoicos, que no hacen mal a nadie, sino a sí mismos.

Otra cosa son los ^materialistas^. La filosofía materialista ignora que
haya problemas. Ni siquiera consideran que la ^muerte^ sea un problema.
Se limitan a ^medir^, y se quedan tan contentos al verificar que,
también al morir, se conserva la ^energía^. ¡Vaya consuelo!


\Section El problema de la supervivencia

\[Locke] pensaba que nuestra mente, al ^nacer^, es como una hoja de
papel en blanco. No es así. Un bebé recién nacido dispone de una enorme
cantidad de información sobre lo que favorece la vida y retarda la
muerte. Recibe toda esta ^información^ como herencia genética.

La vida es así ahora, pero no pudo comenzar así. Considerando la ^ley
de la información creciente^, hemos de inferir que hubo un momento,
justo al principio, en el que toda la vida al completo disponía de tan
poca información que podía expresarse toda ella con un único ^bitio^. Un
bitio de información distingue entre dos posibilidades, que en este caso
son, obviamente, vivir y no vivir.

De acuerdo a estas ideas, la ^vida^ tuvo el ^problema^ de elegir, de entre
estas dos posibilidades, la única válida: vivir. Denominaremos ^problema
de la supervivencia^ a este problema que define epistemológicamente la
vida. La vida dedica todos sus recursos y esfuerzos a la resolución del
problema de la supervivencia.

Las palabras más famosas del más famoso de los poetas no lo son por
casualidad. ``To be, or not to be ---that is the question'' (en
castellano: ``Ser, o no ser ---ése es el problema'') es la manera
radical y genial que \[Shakespeare] empleó para formular el problema de
la supervivencia.


\Section El empirismo vital

\[Darwin] es el culpable de la diferencia que hay entre este ^empirismo^
que estamos proponiendo y el empirismo clásico de \[Locke]. El
mecanismo de ^evolución^ de las especies es capaz de sintetizar
^conocimiento^ y por esto el ser vivo nace con conocimientos innatos.
Luego es la ^vida^, entendida como un ente único al gusto de
\[Lovelock], la que ha obtenido todo su conocimiento de las medidas.

Nosotros, los individuos vivos, obtenemos parte del conocimiento de
nuestros sentidos, esto es, de ^mediciones^ hechas por nosotros mismos,
y otra parte la heredamos de nuestros padres. La parte heredada también
proviene de mediciones, no exactamente hechas por nuestros antepasados,
pero sí por la evolución jugando con sus vidas; para la evolución cada
individuo es una hipótesis cuyo éxito depende de la viabilidad de su
descendencia.

Además, las personas podemos ^comunicar^ a otros nuestras percepciones
y, por esta vía, podemos acceder a las medidas realizadas por personas
con las que no estamos emparentadas. Por esta tercera vía aumenta
enormemente nuestro caudal de conocimientos, pero esa es otra historia.


\Section ¿Qué es un problema?

Primero averiguamos, con \[Aristóteles], que la ^vida^ es ^información^,
después atribuimos la información incesante a la ^libertad^, y ahora
acabamos de descubrir que existe una relación estrecha entre la ^vida^ y
un ^problema^, el ^problema de la supervivencia^. Pero aún no sabemos
qué es un problema, ni qué relación liga al problema con la libertad y
con la información. A eso vamos.

El filósofo y pedagogo estadounidense del siglo {\sc xx}, \[Dewey], se
preguntó por qué los filósofos, que siempre se han esforzado en la
resolución de aporías, nunca se han preocupado por esclarecer el
concepto de `problema'. Lo que ocurre es, simplemente, lo mismo que
ocurre en todas las disciplinas científicas, y que ya denunciamos a
propósito de la ^biología^. Recordemos que la biología no define qué es
la vida, siendo su objeto, ni la ^física^ define qué es la energía,
siendo uno de sus conceptos principales. De modo que, siendo el problema
uno de los asuntos primordiales de la filosofía, la ^filosofía^ no
responde a la pregunta: ¿qué es un problema?

No tenga pena, pero aquí y ahora vamos a terminar con una ^tradición^
que ha perdurado casi tres milenios, desde el comienzo de la historia de
la filosofía, en la antigua Grecia, hasta hoy. Vamos a desvelar qué es
un problema.


\Section El problema

Un ^problema^ es una ^condición^ puesta a cierta ^libertad^. Así que los
problemas se componen de dos partes: libertad y una condición.
$$\hbox{Problema}\llave{Libertad\cr Condición}$$

Si no hay libertad, es decir, si no hay varias alternativas posibles,
entonces no hay problema. Sin posibilidades no hay problemas. Otra
definición: problema es ^potencia^ condicionada.

El ^fatalismo^ sólo es trágico si se consideran las posibilidades que el
destino niega. En principio, `fatal' es sinónimo de `inevitable', y
solamente desde la libertad cabe considerar funesta la fatalidad. Cuando
falta esa mirada posibilística, el fatalismo no es problemático en
absoluto, sino al contrario, ya que sólo afirma sosegadamente que todo
sigue su inalterable curso.

La ^condición^ sólo puede tener dos resultados, y uno de ellos se
denomina ^cumplimiento^ o ^satisfacción^. Así, cada posible opción
cumplirá o no la condición del problema. Toda aquella alternativa que
satisface la condición del problema es una ^solución^ del problema. Un
problema puede no tener ninguna solución, si ninguna alternativa cumple
la condición, o puede tener una única solución, o más de una, incluso
infinitas.

Una condición ($p$) puede ser ^negada^ ($\lnot p$), y dos condiciones
($p$, $q$) pueden ser combinadas por ^conjunción^ ($p
\land q$) o por ^disyunción^ ($p \lor q$). Estas tres operaciones
permiten componer del modo que se quiera una única condición a partir de
otras condiciones. Toda la ^lógica clásica^, expuesta por
\[Aristóteles], queda sumarísimamente compendiada, con la ayuda de las
^tablas de verdad^ de \[Wittgenstein], en el ^álgebra de \[Boole]^.
$$\vbox{\def\~{\hskip 3pt minus 0.5pt \relax}
     \setbox0=\hbox{$p \land q$ es verdad \~ si, y sólo si, \~
                       $p$ es verdad { \bf y } $q$ es verdad}
     \dimen0=\wd0 \box0 \kern3pt
     \hbox to \dimen0{$p \lor q$ es verdad \~ si, y sólo si, \~
                      $p$ es verdad { \bf o } $q$ es verdad}
     \hbox to \dimen0{\hfil({\bf o} lo son ambas)}}$$


\Section La solución

Una ^solución^ del ^problema^ es un uso de la ^libertad^ que satisface
la ^condición^. Una ^resolución^ es un proceso que comienza con el
problema, con su condición y su libertad no ejercitada, y termina
felizmente cuando se halla una solución, esto es, cuando se encuentra
una manera de ejercer la libertad que cumple la condición. Un problema
puede tener una, más de una, o ninguna solución. Si un problema no tiene
solución, entonces no es posible concluir felizmente su resolución.
$$ \hbox{Problema} \longrightarrow
 \hbox{Resolución} \longrightarrow \hbox{Solución}$$

Tenga presente la distinción que aquí hacemos entre la resolución y la
solución del problema, porque normalmente se consideran sinónimos. La
resolución es un proceso, mientras que tanto el problema como la
solución son dos estados; el problema es un estado de ^indeterminación^,
y la solución es un estado de ^satisfacción^. O sea, que resolver es a
buscar como solucionar es a encontrar, y nótese que se puede buscar lo
que no existe. En cambio, para encontrar algo, hemos de verlo, de modo
que lo encontrado, que equivale a la solución, ha de ser algo
perceptible, ^semántico^.
$$\vbox{\halign{\strut\hss#& \hss# $\cdot$ \hss& #\hss\crcr
 Resolver& &Buscar\cr
 Solucionar& &Encontrar\cr
}}$$

Se puede explicar con otra analogía. El problema queda definido por la
tensión que existe entre dos opuestos: la libertad, que está exenta de
límites, y la condición, que es puro ^límite^. Esta tensión es la causa
del proceso de resolución. Pero una vez cumplida la condición y
consumida la libertad, la solución aniquila el problema. La resolución
es, pues, un proceso de ^aniquilación^ que elimina tanto la libertad
como la condición del problema para producir la solución.
$$\underbrace{\vcenter{
  \halign{\strut\hfil\rm#\crcr Libertad\cr Condición\cr}}
    }_{\hbox{\rm\strut Problema}} \, \Bigr\rbrace
 \mathop{\hbox to 80pt{\rightarrowfill}}\limits^{\hbox{\rm Resolución}}
 \hbox{\rm Solución}$$

Permítame un último ejemplo. En un problema de cálculo ^aritmético^, la
solución es un número y la resolución un ^algoritmo^, como, por ejemplo,
el algoritmo de la división.

Perdone mi insistencia, pero es imprescindible que distinga la
^solución^ de la ^resolución^, para que entienda la ^teoría del
problema^. Así que relea esta sección si no es capaz de dar
sentido a la frase: `he resuelto no volver a plantearme el problema,
porque sospecho que no tiene ninguna solución a mi gusto'. Sigamos.


\Section La resolución

Hay tres maneras de resolver un ^problema^: por ^rutina^, por ^tanteo^,
o por ^traslación^.
$$\hbox{Resolución}\llave{Por rutina\cr Por tanteo\cr Por traslación}$$

Para resolver un problema por ^rutina^, o sea, por mera costumbre y sin
necesidad de razonar, es preciso conocer la ^solución^, es necesario
saber que soluciona ese problema, y es menester ejercitar la solución.

Si no se conoce la solución del problema, pero se sospecha de un
conjunto de ^posibles^ soluciones, entonces se puede emplear el
^tanteo^, que es un procedimiento de ^prueba y error^. Tantear consiste
en probar cada posible solución hasta encontrar una que no resulte
errónea. Podemos diferenciar dos tareas al tantear: probar si una
posible solución cumple la ^condición^, y gobernar el proceso
determinando el orden de las pruebas. Como el _gobierno<gobernador>
puede ser realizado de diferentes modos, o sea, tiene ciertos grados de
libertad, si además le imponemos alguna condición, por ejemplo un plazo,
entonces resulta que el propio gobierno es también un problema. Y hay
tres maneras de resolver un problema (\latin{da capo}).

A la ^resolución^ por ^traslación^ también se le llama resolución por
^analogía^. La traslación de un problema consiste en convertirlo en otro
problema, que llamaremos ^cuestión^, que suele componerse de varios
subproblemas. Hay tres maneras de resolver la cuestión y cada uno de sus
subproblemas: por rutina, por tanteo, o por traslación ({\it da
capo\/}). Si hallamos una solución a la cuestión, que llamaremos
^respuesta^, y somos capaces de efectuar la traslación inversa, habremos
encontrado una solución al problema original.
$$\vbox{\halign{\strut\hss#\hss& \hss#\hss& \hss#\hss\crcr
 Problema&&Solución\cr
 $\downarrow$&&$\uparrow$\cr
 Cuestión&\space$\longrightarrow$ & Respuesta\cr
 }}$$


\Section El problema aparente

Siempre que nos enfrentamos a un ^problema^ disponemos de alguna
^información^ que nos guía en su ^resolución^. Parece obvio considerar
que la información es el tercer elemento que define el problema, con la
^libertad^ y la ^condición^, sobre todo porque sin ninguna información
solamente podríamos resolverlo a ciegas. Sin embargo, para alcanzar el
^origen de la vida^, hemos de prescindir de toda la información y
quedarnos únicamente con el problema tal como lo hemos definido. Porque,
si conforme a la ^ley de la información creciente^, la ^vida^ va
acumulándola, entonces, extrapolando hacia atrás en el tiempo,
hemos de suponer que empezó con la mínima cantidad de información, que
es cero, o sea, que la vida comenzó sin información.

Para indicar positiva e inequívocamente que nos referimos al problema
puro, llamaremos ^problema aparente^ al problema sin información, y que
es exclusivamente libertad y una condición. El problema aparente es el
que \[Klir] denomina problema de la caja negra puro. El problema
aparente es el problema mínimo, es decir, es el problema del que nada se
sabe. Aplicando esta terminología a la vida, resulta que el ^problema de
la supervivencia^ es un problema aparente. Y, por esto, el problema
aparente es literalmente el origen de todo.

Lo único que sabemos sobre un problema aparente es que es un problema, y
no otra cosa. Sabemos, entonces, que consta de libertad y de una
condición, pero ni siquiera conocemos la ^condición^. Si la conociéramos
y pudiéramos expresarla, entonces podríamos argumentar si una posible
solución es o no una solución antes de intentarla, y seríamos capaces de
razonar sobre las resoluciones más adecuadas. Si se fija bien, conocer
la condición es conocer el problema, porque la libertad es única y
siempre igual a sí misma. Pero eso es mucho más de lo que se conoce de
un problema aparente, y, por consiguiente, repito, la condición de un
problema aparente es desconocida.

Al no conocerse la condición, no puede calcularse por adelantado si una
posible ^solución^ del ^problema aparente^ lo es actualmente. Es decir,
un problema aparente no se puede resolver sin enfrentarse a él. Pero
como sabemos que el problema aparente es un ^problema^, y no otra cosa,
sabemos que podemos ejercitar la ^libertad^, es decir, sabemos que
podemos acometer su ^resolución^. Cuando la intentemos, pero no antes,
sabremos si esa resolución alcanza efectivamente una solución o no, y
esa información de {\sc sí} o {\sc no} se puede anotar con un ^bitio^.
Si posteriormente volviéramos a enfrentarnos al mismo problema, entonces
ya dispondríamos de un bitio de ^información^ y si, por ejemplo, la
primera vez hubiera sucedido que {\sc sí} lo habíamos solucionado,
entonces sería más razonable intentar la misma resolución que probar
una de las otras posibilidades.


\Section La resolución del problema aparente

Otra manera de entender el ^problema aparente^ es percatarse de que, ya
que nada sabemos de él, entonces no tenemos argumentos para descartar
ningún ^problema^ y podría ser cualquiera. El problema aparente puede
ser cualquier problema que se pueda imaginar. También podría ser uno
inimaginable, pero nos contentaremos con considerar todos los
imaginables, más que nada, porque nos es imposible examinar los
inimaginables según una ^tautología^ cercana a \[Wittgenstein]: lo
inimaginable no se puede ^imaginar^.

Ahora supongamos que se enfrenta usted repetidamente a un problema
aparente, que su primera resolución ha dado con una solución, y entonces
la ha repetido otras nueve veces, y también lo ha solucionado. O sea,
lleva diez de diez aciertos, \frac(10/10). Puede estar usted contento,
pero no seguro. Recuerde que lo único que sabe seguro, ahora, es que el
problema al que se enfrenta es cualquiera que pueda usted imaginarse,
con la condición de que sea solucionado las diez primeras veces por su
^resolución^. Por ejemplo, su problema podría aceptar cualquier
resolución, siempre y cuando no se repitiera más de diez veces seguidas.
Si fuera así, y repitiera por undécima vez, fallaría, pero, si luego en
la duodécima intentase cualquier otra cosa, acertaría.

Lo que importa es que, enfrentado repetidamente a un problema aparente,
la única ^información^ segura que usted tiene es la secuencia de
aciertos y fallos que han producido sus intentos de solución hasta ese
instante. Y yo soy capaz de imaginar, sea como sea la secuencia, dos
problemas distintos, uno ante el cual su próxima resolución acertará a
solucionarlo y otro ante el que fallará. Seguro que usted también puede
imaginárselos. Lo que esto significa es que, al enfrentarse a un
problema aparente, nunca es seguro que una resolución alcanzará
felizmente la solución.

No es posible resolver definitivamente un problema aparente.
Aunque se haya solucionado esta vez, no hay manera de certificar
que la exitosa resolución aplicada ahora proporcionará,
siempre y en cualquier circunstancia,
medios suficientes para satisfacer la condición del problema aparente, 
porque ésta es desconocida.
Así que un ^problema aparente^ puede ser solucionado, pero no resuelto,
y es imposible ^aniquilarlo^ definitivamente.

Siempre que nos enfrentamos a un ^problema aparente^, y aunque no sea la
primera vez, ocurre que antes de acometer su resolución hay
^incertidumbre^ acerca de su solución. Así que cada resolución de un
problema aparente aporta ^información^, conclusión que generaliza la
^ley de la información incesante^ a la ^resolución^ de todos los
problemas aparentes, incluido el de la supervivencia. El problema
aparente de la supervivencia proporciona el primer ^bitio^  de
información, y también los siguientes.


\Section Somos resoluciones

Un problema aparente podría ser cualquier problema. Por esto,
aunque un problema aparente no tiene una resolución definitiva,
sí que tiene resoluciones mejores y resoluciones peores:
son mejores las capaces de solucionar más posibles problemas. 
Cabe entonces plantear una meta-resolución que busque 
resolutores cada vez mejores de un problema aparente.
Y, con estas ideas, ya podemos encajar más precisamente la evolución
en la ^teoría del problema^.

La ^evolución^ darviniana es un proceso de ^resolución^ repetida del
^problema aparente^ de la supervivencia que acumula ^información^. En
ese proceso cada individuo es una de las repeticiones, lo que significa
que los individuos vivos somos resoluciones y no soluciones del problema
de la supervivencia. Dicho más simplemente, somos procesos, y no
estados; usted y yo también.

Ya le advertí de que la explicación de cómo un ^lenguaje simbólico^
proporciona ^introspección^ iba a ser larga. Aún falta, pero ya tenemos
una conclusión importante: somos resolutores. Es importante porque
permite reformular la cuestión de una manera más precisa. Si probamos
que un lenguaje simbólico puede representar resoluciones, y uno
semántico no, entonces habremos alcanzado una explicación satisfactoria
de la introspección reflexiva.

Tengo dos contestaciones para la cuestión así reformulada. Una es
sencilla, aunque parcial, ya que sólo demuestra que un ^lenguaje
semántico^ es insuficiente. La otra es completa, complicada y feraz.

La sencilla cabe entera en este párrafo. Aquí la tiene. Para representar
el proceso completo de ^resolución^ es menester representar el punto de
partida, el punto de destino, y el itinerario a seguir para alcanzar el
destino desde el origen. La contestación fácil se fija sólo en el punto
de partida, que es el ^problema^, porque ya sabemos que un ^lenguaje
semántico^ no puede representar problemas, y uno simbólico sí. En
definitiva, como somos resolutores, como para representar resoluciones
hay que representar problemas, y como el lenguaje semántico no puede
representar problemas, la conclusión es que para poder observarnos a
nosotros mismos, o sea, reflexivamente, no es suficiente disponer de un
lenguaje semántico. Para probar que un lenguaje simbólico es suficiente
necesitamos la contestación complicada y feraz.

La contestación completa, complicada y feraz no es para todos los
gustos. Yo, sin embargo, le aconsejo que la siga punto por punto y sin
saltarse nada, aunque sin detenerse en los asuntos que no entienda.
Porque, aunque no alcance algunos argumentos, que requieren
conocimientos matemáticos, al leerlos se percatará de cuales son los
fundamentos sobre los que descansan las conclusiones. Y las conclusiones
son muchas y muy importantes, se lo aseguro.

Para facilitarle las cosas, titularé ``¡Alto!'' una sección en la que le
resumiré los resultados de la excursión matemática. A partir de esa
sección debe usted volver a leer atentamente e intentando cuadrar todo
cuanto escribo. Pero, hasta allí, simplemente lea con atención.


\Section La palabra

La contestación complicada examina detalladamente cómo representar todos
y cada uno de los elementos de la ^resolución^: el origen, que es un
^problema^, el itinerario, que es un proceso de transformación, y el
destino, que es una ^solución^. En concreto, hemos de representar los dos
componentes del problema, que son la libertad y la condición, las tres
maneras de resolver, a saber, por rutina, por tanteo y por analogía, y
la solución. De este modo investigaremos cómo debemos extender un ^lenguaje
semántico^, que no puede representar problemas, para que pueda
representar resoluciones. Es decir, intentaremos componer un ^lenguaje
simbólico^ aumentando uno semántico. Ésta será nuestra tarea a partir de
aquí.

El lenguaje semántico tiene palabras para los objetos de la
^percepción^. Así `^caballo^' es una palabra que se refiere a un
^objeto^ que podemos ver. No hacen falta palabras para explicar el
^significado^ de la palabra `caballo', porque la palabra `caballo' no es
más que un sonido arbitrario que los hablantes del ^castellano^
aceptamos convencionalmente como una propiedad más del objeto caballo
producido por otros procesos cognitivos. Esto disuelve la ^paradoja del
diccionario^.

Un ^diccionario^ explica cada una de las palabras de un idioma
utilizando otras palabras del mismo idioma. Esas otras palabras también
aparecen en el dicionario, y son explicadas del mismo modo, o sea, con
más palabras. Así que, mirado en conjunto y por sí mismo, el diccionario
es un enorme ^círculo vicioso^ que no explica absolutamente nada. Puede
usted verificar esto muy fácilmente: tome un diccionario de una lengua
que desconozca completamente, y comprobará que no le sirve para aprender
ni una sola ^palabra^ de dicha lengua.

Para fijar el concepto, quédese con que la palabra semántica es una
etiqueta ---un nombre--- que sirve para referirse a un objeto. Y
recuerde que yo siempre hablo desde el subjetivismo. Me incomoda tener
que recordárselo, pero es que ustedes, los objetivistas, son pertinaces
en su contumacia.

El paso decisivo para ir del lenguaje semántico al simbólico es
sencillo: liberar las etiquetas. Mientras que en un lenguaje semántico
las palabras son etiquetas necesariamente ligadas a un objeto, en uno
simbólico pueden ser libres. Así, una palabra simbólica es una etiqueta,
sin más requisitos. En concreto, una palabra simbólica puede referirse a
un objeto, o puede referirse a otras palabras e, incluso, puede no tener
referente. Esto hace tres ^tipos de palabras^.

La manera natural de presentar las palabras que se refieren a otras
palabras es el diccionario. Pero, cuando se añaden al diccionario las
palabras sin referente y las que se refieren a los objetos de la
percepción, se crea la falsa ^ilusión^ de que el diccionario explica
todas las palabras de la lengua.


\Section La incógnita

Para expresar un ^problema^, hay que referirse a la ^condición^ y a la
^libertad^. Y, para anotar la libertad hay que usar una palabra sin
referente. Por ejemplo, si el problema es que no sabemos qué hacer, su
expresión más directa en castellano es: `¿qué hacer?'. En esta oración,
el ^pronombre^ interrogativo `^qué^' no se refiere a nada en concreto,
sino que es una palabra libre de significado y sin ataduras. Fíjese que
la palabra `qué' no es semántica, de modo que ha de ser puramente
sintáctica; `qué' es una mera etiqueta.

Por supuesto, la expresión de un problema puede ser menos directa. Por
ejemplo, cuando ^Hamlet^ plantea su ``ser, o no ser ---ése es el
problema'', no necesita utilizar un pronombre interrogativo. En puridad,
`ser o no ser' es una ^tautología^, y no un problema. Pero, cuando
después añade \[Shakespeare], por boca de Hamlet, que es {\em el}
problema, convierte súbita y necesariamente a cada alternativa en una
^posibilidad^, a la vez que nos urge a elegir libremente una de ellas.
La fuerza de la frase es formidable, porque nos obliga a reconocer que
morir es efectivamente una posibilidad. Y no una posibilidad cualquiera
de cualquier problema, porque es obvio que, para la ^vida^, vivir o
morir es {\em el} problema; para usted y para mi, también.

En ^matemáticas^ la manera típica de expresar la libertad de un problema
es usar la ^incógnita^ $x$, que funciona igual que el pronombre
interrogativo `qué'. Por ejemplo, si queremos averiguar qué número es
igual doblado que cuadrado, podemos expresarlo así:
$$x?\quad 2x = x^2 .$$

La ^condición^, en este caso, es la igualdad $2x = x^2$. Como la
igualdad es una relación que puede satisfacerse o no  ---{\sc sí} o {\sc
no}--- vale como condición. Más interesante es que en la expresión de la
condición debe aparecer necesariamente la incógnita, ya que de otro modo
la expresión no serviría para determinar si una posibilidad es solución
o no. Así que, en la condición, la incógnita $x$ actúa como ^variable
libre^, ^adjetivo^ que es muy adecuado. Una expresión con variables
libres se califica como abierta. Una ^expresión abierta^ se denomina
función. Una ^función^ no puede ser semántica porque incluye variables
libres, que son términos no semánticos.

Esta observación matemática puede generalizarse. En todos los lenguajes
simbólicos pueden utilizarse expresiones abiertas con variables libres.
Por contra, ningún lenguaje semántico admite expresiones abiertas.


\Section Tres problemas resueltos

La ^resolución^ del ^problema^ anterior
$$x?\quad 2x = x^2$$
podría seguir los siguientes pasos: 
\vadjust{\break\hrule width0pt\kern-10pt\relax}
$$\eqalign{
  2x      &= x^2 \cr
  2x - 2x &= x^2 - 2x \cr
  0       &= xx - 2x \cr
  0       &= (x - 2)x \cr
  [x - 2 = 0]           &\lor [x = 0] \cr
  [x + (-2) = 0]        &\lor [x = 0] \cr
  [(x + (- 2)) + 2 = 0  \phantom]&\hbox{+}\phantom[ 2] \lor [x = 0] \cr
  [x + (-2 + 2) = 2]    &\lor [x = 0] \cr
  [x + 0 = 2]           &\lor [x = 0] \cr
  [x = 2]               &\lor [x = 0] .}$$
De modo que tiene dos soluciones, dos y cero, porque un problema como
$x?\; x=2$ es lo que los matemáticos llaman trivial, aquí con razón.

En este caso la resolución ha consistido en ir trasladando el problema
por ^analogía^ hasta dar con dos problemas de los que se conoce la
solución y que, por lo tanto, pueden resolverse de modo ^rutinario^. No
es ésta la única manera de resolverlo. Recordemos que el ^tanteo^,
aunque sea matemáticamente menos elegante, es otra posibilidad.

Con una condición más estricta, el problema tendría una o ninguna
^solución^. Por ejemplo, si quisiéramos averiguar qué número positivo es
igual doblado que cuadrado, el problema se expresaría
$$x?\quad (x > 0) \land (2x = x^2) ,$$
y tendría una única solución, a saber, dos. Nótese que aquí hemos
compuesto la ^condición^ del problema por conjunción de otras dos
condiciones.

Y el ^problema^
$$x?\quad (x > 2) \land (2x = x^2)$$
no tiene solución, aunque puede ser resuelto. Para resolverlo basta con
establecer que efectivamente no tiene solución. Éste es otro ejemplo de
lo útil que resulta distinguir `^solución^' de `^resolución^'.


\endinput
