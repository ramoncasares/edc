% edc-triptic.tex (RMCG20100912)

\font\titlefont=texnansi-lmssbx10 at 90pt
\font\subtitlefont=texnansi-lmss10 at 40pt
\font\authorfont=texnansi-lmss10 at 50pt

\font\spinetitlefont=texnansi-lmssbx10 at 24pt
\font\spineauthorfont=texnansi-lmss10 at 24pt

\font\citefont=texnansi-lmss10 at 21pt
\font\textfont=texnansi-lmss10 at 14pt

\ansifont

\def\pdfRed{\pdfliteral{0 1 1 0 k}}
\def\pdfBlack{\pdfliteral{0 0 0 1 k}}
\def\pdfWhite{\pdfliteral{0 0 0 0 k}}

\pdfpageheight=9in \voffset=-1in 
\pdfpagewidth=12.5in \hoffset=-1in

\newbox\background
\setbox\background=\hbox{\pdfRed\vrule width 6in height 0pt depth 9in}
\wd\background=0pt 
\ht\background=0pt
\dp\background=0pt

%%%%%%%%%%%%% Portada

\newbox\portada
\setbox\portada=\vbox to 9in{\kern0.5in\copy\background
 \hbox to 6in{
 \hss\vbox{\hsize=6in
  \pdfWhite
  \hbox{}
  \vskip3in
  \centerline{\titlefont El doble}
  \vskip1pc
  \centerline{\titlefont compresor}
  \vskip10pc
  \centerline{\subtitlefont La teoría de la}
  \vskip6pt
  \centerline{\subtitlefont información} 
  \vskip2pc
  \centerline{\authorfont Ramón Casares}
 }\hss}\vss}


%%%%%%%%%%%%% Lomo

\def\rotatebox#1{%
 \pdfliteral{q}% save current graphic state in the stack
 \pdfliteral{0 -1 1 0 36 0 cm}% rotate -90º
 %\pdfliteral{1 0 0 1 0 0 cm}% identity transform
 \wd#1=0pt
 \box#1
 \pdfliteral{b Q}% close, stroke, fill, and restore graphic state
}

\newbox\lomotext
\setbox\lomotext=\hbox to 9in{\pdfRed
 \vrule width 9in height 0.35in depth 0.15in
 \kern -7in \pdfWhite
  \spinetitlefont El doble compresor\hss
  \spineauthorfont Ramón Casares\kern24pt}

\newbox\lomo
\setbox\lomo=\vbox{\rotatebox\lomotext}
\wd\lomo=0.5in
\dp\lomo=0pt
\ht\lomo=9in

%\shipout\box\lomo


%%%%%%%%%%%%% Contraportada

\newbox\contra
\setbox\contra=\vbox to 9in{\copy\background
 \vss\hbox to 6in{
 \hss\vbox{\hsize=5.5in
  \pdfWhite
  \hbox{}
  \begingroup\citefont\parindent=0pt\baselineskip=24pt\obeylines
``El mundo no es una máquina enorme,
\leavevmode\kern12pt como creen los materialistas,
\leavevmode\kern12pt sino un enigma inmenso'' {\textfont(§160 ¶6)}
\endgroup
\vskip 1pc
\begingroup\textfont\parindent=0pt\baselineskip=16pt\obeylines

En el libro le muestro las limitaciones del materialismo,
y le presento una teoría para superarlo.
Como vía de ataque utilizo el concepto de información,
que investigo con usted inspeccionando juntos las grietas de la
percepción. Mi evaluación subjetivista es concluyente:
la realidad no es lo que está fuera,
sino el producto de la percepción.
\null 
Pero, resulta que el poder expresivo de los procesos perceptivos
es limitado. La prueba es que podemos hablar de cosas que
no podemos observar. Por esto, la realidad no lo es todo, sino la
representación de lo exterior que construye la percepción
con sus exiguos recursos expresivos. Y el materialismo, al ceñirse
a lo observable, a lo medible, es innecesariamente pobre.
Urge superar esta situación.
\null
Visto que el habla es más expresiva que la percepción,
me fijo en que no podemos percibir los problemas.
Entonces demuestro que, para poder representar todos los
conceptos de la teoría del problema, es necesaria una
sintaxis recursiva. Esto es muy técnico, pero,
basta saber que la sintaxis del habla es una sintaxis recursiva,
para entender por qué sólo los sujetos parlantes podemos
examinar mentalmente las posibles resoluciones de un problema.
También le explico la consciencia, pero, para alcanzar los detalles,
ha de leer el interior de este libro.

\endgroup
\vskip2pc
\pdfBlack
\input ean13
\line{
\pdfximage{cc-by-sa-www.pdf}\pdfrefximage\pdflastximage
\hfil
\pdfliteral{q}% save current graphic state in the stack
  \pdfliteral{0 0 0 0 k 0 0 0 1 K}% set black stroke on white fill
  \pdfliteral{-5 -5 m}% moves to the origin
  \pdfliteral{110 -5 l}%
  \pdfliteral{110 92 l}%
  \pdfliteral{-5 92 l}%
  \pdfliteral{-5 -5 l}% 
  \pdfliteral{b Q}% close, stroke, fill, and restore graphic state
\ISBN 978-1-4536-0915-6
\barheight=2.5cm % this must be after \ISBN call
\EAN 978-1-4536-0915-6
\hskip9pt
}}\hss}\vss}


%\shipout\hbox{\box\contra}
%\shipout\hbox{\box\portada}

\shipout\hbox to \pdfpagewidth{\box\contra\hss\box\lomo\hss\box\portada}

\end

