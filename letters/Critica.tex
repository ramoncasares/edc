% Critica.tex (20060302)

\input doc
\def\quien{Ram�n Casares}
\def\eaddress{r.casares@ieee.org}
\def\date{6 de marzo de 2006}
\def\address{Editorial Cr�tica\\
             Diagonal 662--664, 7�\\
             08034 {\sc Barcelona}}
\letterlayout             


Muy se�or m�o:

El motivo de esta carta es ofrecerle el libro que la acompa�a, y que he
titulado {\sl El doble compresor}. El libro est� completamente terminado,
aunque acepto comentarios de cualquier tipo, incluso tipogr�ficos.

A pesar de lo que pueda sugerirle el t�tulo, se trata de un ensayo
filos�fico que, en �ltimo t�rmino, solamente pretende contestar tres
preguntas:
{\parskip=0pt
\point �por qu� no recordamos nuestro nacimiento?,
\point �por qu� aprendemos antes a hablar que a dibujar?, y
\point �por qu� nos resulta dif�cil dibujar?
\par\noindent}\ignorespaces
Son preguntas muy f�ciles pero con respuestas muy dif�ciles, porque no
pueden responderse adecuadamente sin explicar cabalmente la percepci�n,
el habla y la consciencia, y �stas ya son palabras mayores.
En fin, que �se es el entramado que he dispuesto para presentar
mi teor�a de la informaci�n.

Espero que el libro sea de su agrado, y pueda ser publicado por
Cr�tica.

\Firma{\sc Ram�n Casares}

\PD P.D. Le agradecer�a que, aunque demorara la respuesta definitiva,
acusara recibo de esta carta a mi direcci�n de correo electr�nico. Gracias.

\end
