% alianza.tex (20091009)

\input doc
\def\quien{Ram�n Casares}
\def\eaddress{r.casares@ieee.org}
\def\date{14 de octubre de 2009}
\def\address{Alianza Editorial\\
             C/ Juan Ignacio Luca de Tena, 15\\
            E--28027 {\sc Madrid}}
\letterlayout             


Muy se�or m�o:

El motivo de esta carta es ofrecerle un libro que he escrito,
y que le adjunto (la copia est� aumentada al 120\%).
El libro est� completamente terminado,
aunque acepto comentarios de cualquier tipo, incluso tipogr�ficos.

A pesar de lo que pueda sugerirle el t�tulo, {\sl El doble compresor}
es un ensayo epistemol�gico que, en una primera aproximaci�n,
plantea tres preguntas sencillas:
{\parskip=0pt
\point �por qu� nos resulta tan dif�cil dibujar bien?,
\point �por qu� aprendemos antes a hablar que a dibujar?, y
\point �por qu� no recordamos nuestro nacimiento?
\par\noindent}\ignorespaces
Son preguntas muy f�ciles de entender pero muy dif�ciles de contestar,
porque no pueden responderse adecuadamente sin explicar cabalmente
la percepci�n, el habla y la consciencia,
y estas ya son palabras mayores.
En fin, que este es el entramado que he dispuesto para presentar
mi teor�a de la informaci�n.

No voy a darle m�s explicaciones ya que el libro est� completo y
debe hablar por s� mismo. Sea, eso s�, benevolente, porque la
epistemolog�a es una materia exigente con el lector y
que no siempre puede ser simplificada.
Tampoco le adjunto mi {\em curriculum vitae}, 
porque soy un doctor ingeniero de telecomunicaci�n
a punto de cumplir los cincuenta,
y sospecho que esa informaci�n no nos ayudar� 
a vender m�s copias de un libro de filosof�a,
as� que p�ngase en lo peor.

Aun con esas, espero que el libro sea de su agrado e inter�s,
y pueda ser publicado por Alianza. 
Un saludo,

\Firma{\sc Ram�n Casares}

\PD P.D. Le agradecer�a que, aunque demorara la respuesta definitiva,
acusara recibo de esta carta a mi direcci�n de correo electr�nico. Gracias.

\end

