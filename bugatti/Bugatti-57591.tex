% Bugatti-57591.tex (RMCG20091014)

\input doc
\rmcglayout
\def\pounds{{\it\$}} %$

\title0 Bugatti Atlantic 57SC

\centerline{\hbox{%
 \pdfximage width 8cm 
   {1938-Bugatti-Type-57SC-Atlantic_Front.jpg}%
 \pdfrefximage\pdflastximage}}

\title1 Concorso d'Eleganza Villa d'Este 2009:

Bugatti Atlantic 57SC, 1938, owner Ralph Lauren (03/2009) - © BMW AG

Chassis \#57591 - The last Atlantic is the most published of the three
and can probably claim to be the most original.
It is instantly recognizable from its external headlights which many
people feel make it the most desirable of the three.
Ralph Lauren has owned this car since 1988.
With a strong inclination towards important and authentic cars,
 Ralph includes \#57591 as a highlight in his collection which has
 Type 57SC Gangloff Cabriolet \#57563.

\#57591 was initially purchased by R.B.\ Pope of London in Dark
Sapphire Blue. The car still bears its EXK6 plate from its original
registration in the UK. It was supercharged in 1939, before being
sold to notable author Barry Price in the sixties. Eventually
New York designer Ralph Lauren picked up the car and commissioned
Paul Russel to comprehensively restore \#57591 using as many original
parts as possible. During the lengthy two-year procedure, Paul and
his team discovered details such as original tan goatskin upholstery
and seats filled with horsehair bags wrapped in muslin.
After the project was complete it was shown at the 1990 Pebble Beach
Concours D'Elegance where it won best of show.
Like many of Ralph's Bugattis the car is finished in black.

\vskip 1pc
\centerline{\hbox{%
 \pdfximage width 10cm 
   {1938-Bugatti-Type-57SC-Atlantic_Rear2b.jpg}%
 \pdfrefximage\pdflastximage}}


 
\title1 Bugatti Type 57SC Atlantic: Coup de Grace\\by Ken Gross

People always ask if I have a favorite car — and I do. It's a Bugatti,
the Type 57SC Atlantic.

Although it's over 70 years old, the Atlantic's dramatic, art moderne
shape, clouded history and race-inspired, supercharged straight-8 engine
make it a timeless classic.

\title2 The Romance of the Streamline

It's truly stunning, with a hand-formed, all-aluminum body designed
under the close supervision of Jean Bugatti, the son of company founder
Ettore Bugatti. Recent research confirms that just four Atlantics were
made, and now only two survive. Each has won the coveted trophy for
``Best of Show'' at the Pebble Beach Concours d'Elegance.

The fate of the other two Atlantics is shrouded in mystery. One was
virtually destroyed by an SNCF train when it stalled at an unguarded
level crossing near Gien, France. The owner had purchased the car in the
name of his mistress so his wife would not discover its existence, and
he was teaching another attractive blonde how to drive it when tragedy
struck. The fourth Atlantic, a factory demonstrator, was the first
example built. It disappeared just before WWII.

Ettore Bugatti was an Italian who lived most of his life in
Alsace-Lorraine, a province in Eastern France. His father Carlo had
created astonishingly elegant, museum-quality furniture, and his brother
Rembrandt was an accomplished sculptor of animals. Briefly trained as an
apprentice engineer at a time when the automobile had newly seized the
imagination of everyone, Ettore possessed the dreamy soul of an artist.

From 1911-1939, Bugatti built uncompromised automobiles of great beauty
and sporting pretension. Often technically (even perversely) complex,
Bugatti cars were expensive, temperamental and often hauntingly
beautiful. Bugatti experimented with aerodynamics, favored expensive
de Ram shock absorbers, eschewed supercharging for years and insisted on
using cable-operated brakes long after hydraulics proved superior.

\title2 Another Sales Promotion

Although the depression of 1929 was slow to impact France due to the
country's high tariffs and restricted trade, the market for luxury
automobiles had finally dwindled by the early 1930s.
Ettore and Jean Bugatti knew that a very special model was imperative to
help their company survive, and the Type 57 was that car.

The concept for the Atlantic was first shown in 1935 at both the Paris
and London auto shows. Called the Competition Coupe Aerolithe (using the
French word for ``meteor''), it rode on a prototype chassis from a
Bugatti Type 57S and was powered by a normally aspirated, 3.3-liter,
DOHC straight-8. Historians are certain that two Aerolithes were built
as prototypes, but they did not exist simultaneously and neither
survives.

Although other manufacturers were experimenting with aerodynamics,
Bugatti's outrageously curvaceous Aerolithe proved to be a design
sensation. In production, it became known as the Aero, and Jean Bugatti
set out to make the body from Electron, an alloy of magnesium and
aluminum. When welding proved difficult, Jean Bugatti and assistant
Joseph Walter united the sections with rivets, which explains the
spinelike center rib dividing the svelte body, a theme repeated in its
teardrop-shaped fenders.

Production Atlantic bodies were aluminum. The rivets were no longer
needed, but they looked exotic, so the illusion of a riveted spine was
retained. Close-coupled, cramped, poorly ventilated and quite
impractical, the sexy lightweight coupe was nevertheless an enthusiast's
delight and one of only a handful of sports cars of the era that could
top 130 mph.

\title2 The Story of Four Cars

The first car (chassis 57374) sold was built in February 1936
and purchased by Lord Phillipe de Rothschild, then one of the world's
richest men as well as an enthusiastic amateur racer.
The second car (chassis 57473) (and the first to be called ``Atlantic'')
was bought eight months later by Monsieur and Madame Holzschuch.
At an undetermined time, the fenders were restyled, and new headlights
were faired in, perhaps by Figoni and Falaschi.
After a checkered history, this car was the one
eventually purchased by the amorous Monsieur Chatard and fated to be
badly mauled in the horrific accident at the railroad crossing.
The third and last customer car (chassis 57591) went to Mr.\ R.B.\ Pope
in March 1938. The tall Englishman usually wore a hat, and he insisted
the roof be made 1 inch taller. Ventilation slots were cut into the car
to facilitate driving in warm weather.

The ex-Rothschild Atlantic was fitted with a supercharger, along with a
period Cotal electromagnetic gearbox. After several owners, it was sold
in 1971 for the then-astronomical sum of \$59,000 to Dr.~Peter Williamson,
who still owns it. The former Pope Bugatti, also retrofitted with a
supercharger, went through just three owners before its purchase by
noted clothier and collector, Ralph Lauren. The factory-owned demo
(chassis 57453) is believed to be forever lost. The ex-Chatard coupe
languished for years while legal action raged over responsibility for
its demise. Finally released from bondage, it was rebodied twice, using
some original and salvaged parts. Arguments persist to this day over
whether it can still be considered a truly authentic Atlantic.

\title2 At the Wheel of \$10 Million

I've loved these cars since I was a teenager and read their story in
Ken Purdy's Kings of the Road, but I could never afford even a prosaic
Bugatti, let alone a fabulous Type 57SC Atlantic. In 1967, Mr. Pope
sold his car to Barrie Price for \pounds3,000 (about \$8,500), the same year
I paid \$1,600 for a new Volkswagen Beetle!

Luckily, I've had my chances to get close to the surviving Atlantics.
I saw Lauren's 57SC shortly after it arrived at Paul Russell's shop in
Essex, Massachusetts, for a frame-off restoration, and I was present at
Pebble Beach when it won Best of Show in 1990. Co-restorer Scott Sargent,
responsible for much of the work on the Williamson coupe, kindly gave me
a ride in the car last year, when the Saratoga Automobile Museum
featured Dr. Williamson's collection. At the same time, I drove
Williamson's Type 57SC Atalante, a slightly larger, heavier and yet
still elegant cousin of the Atlantic, of which just 17 were built.

And what a thrill! Bugatti's supercharged straight-8 is good for close
to 220 horsepower, and you can feel it. The long shift lever has
surprisingly short travel. Throttle response is immediate, and the
gear-driven overhead cams whirr and click with a delightful metallic
cacophony. The deep exhaust note bbbrrraaaaps with authority, and
despite the car's advanced age, skinny tires and stiff suspension, the
low-slung Type S still accelerates briskly and corners smartly.
Oversized Jaeger instruments mounted on a polished wood fascia feature
long, slender needles that wave like a conductor's baton.

Mindful of this car's immense value, which is approaching \$10 million,
one hesitates to drive quickly at first, but the spirits of
Ettore and Jean Bugatti will bristle if you simply toddle along.

A few Type 57 Atlantic replicas have been built on genuine Bugatti
chassis. Jay Leno owns an exact copy. And a German enthusiast replicated
the Aerolithe last year. You can't blame them. Ahead of its time, really
a racing berlinetta for the street, Bugatti's Type 57SC Atlantic is a
sports car for the ages. Famed author Ken W.\ Purdy called Bugatti
``the Prince of Motors.''

``Imagine a string-straight, poplar-lined Route Nationale in France on a
summer's day,'' Purdy wrote. ``That growing dot in the middle distance is
a sky-blue Bugatti coupe rasping down from Paris to Nice at 110 miles
an hour...''.

Just watch those railroad crossings.

\bye

